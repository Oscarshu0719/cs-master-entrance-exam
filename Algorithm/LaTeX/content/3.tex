\item \begin{theorem}{(89)} Longest Common Subsequence (LCS):\begin{itemize}
        \item \begin{equation}
            c[i, j] = \begin{cases}
                0 &, i = 0 \lor j = 0 \\
                c[i - 1, j - 1] + 1 &, i, j > 0 \land x_i = y_j \\
                \max(c[i, j - 1], c[i - 1, j]) &, i, j > 0 \land x_i \neq y_j
            \end{cases}
        \end{equation}
        \item $c[0 \cdots \len(X)][0 \cdots \len(Y)]$,$c[0, 0]$表示空字串,並初始化第一行及第一列為$0$。
        \item 字符不同時,標示左邊或上面較大值方向,數值相同時預設$\uparrow$;字符相同時標示$\nwarrow$。
    \end{itemize}
\end{theorem}

\item \begin{theorem}{(94)} Longest Common Substring:\begin{itemize}
        \item \begin{equation}
            c[i, j] = \begin{cases}
                0 &, i = 0 \lor j = 0 \lor x_i \neq y_j \\
                c[i - 1, j - 1] + 1 &, x_i = y_j
            \end{cases}
        \end{equation}
        \item $c[0 \cdots \len(X)][0 \cdots \len(Y)]$,$c[0, 0]$表示空字串,並初始化第一行及第一列為$0$。
    \end{itemize}
\end{theorem}

\item \begin{theorem}{(94)} Minimum Edit Distance:\begin{itemize}
        \item \begin{equation}
            c[i, j] = \min\begin{cases}
                c[i - 1, j] + 1 &, a_i \neq b_j \\
                c[i, j - 1] + 1 &, a_i \neq b_j \\
                c[i - 1, j - 1] + 1 &, a_i \neq b_j \\
                c[i - 1, j - 1] &, a_i = b_j \\
            \end{cases}
        \end{equation}
        \item 各情況依序表示刪除$\uparrow$、插入$\leftarrow$、替換$\nwarrow$以及匹配$\nwarrow^2$。
        \item $c[0 \cdots \len(X)][0 \cdots \len(Y)]$,$c[0, 0]$表示空字串,並初始化第$i$行為$i$並標示$\uparrow$,第$j$列為$j$並標示$\leftarrow$。
        \item 字符不同時,標示左邊(刪除)、上面(插入)與左上(替換)較小值方向;字符相同時標示$\nwarrow^2$。
    \end{itemize}
\end{theorem}

\item \begin{theorem}{(100)} Matrix-chain Multiplication:\begin{itemize}
        \item \begin{equation}
            m[i, j] = \begin{cases}
                0 &, i = j \\
                \min_{i \le k \le j - 1}\{m[i, k] + m[k + 1, j] + p_{i - 1}p_kp_j\} &, i < j
            \end{cases}
        \end{equation} 
        \item $p[0 \cdots \num(\text{Matrices})]$,存入矩陣大小。
        \item $m[1 \cdots \num(\text{Matrices})][1 \cdots \num(\text{Matrices})]$,初始化對角線上元素為$0$。
        \item $s[1 \cdots \num(\text{Matrices}) - 1][2 \cdots \num(\text{Matrices})]$,$s[i, j]$存入$m[i, j]$中最小值對應的$k$。
        \item 理解:$m[i, k]$為拆分的前部分,$m[k + 1, j]$為拆分的後部分,$p_{i - 1}p_kp_j$為前後部分相乘。
    \end{itemize}
\end{theorem}

\item \begin{theorem}{(111)} Optimal Binary Search Tree (OBST):\begin{itemize}
        \item \begin{equation}
            \begin{aligned}
                e[i, j] & = \begin{cases}
                    q_{i - 1} &, j = i - 1 \\
                    \min_{i \le r \le j}\{e[i, r - 1] + e[r + 1, j] + w[i, j]\} &, i \le j
                \end{cases} \\
                w[i, j] & = w[i, j - 1] + p_j + q_j
            \end{aligned}
        \end{equation} 其中,$p_j$為key(內部節點)機率,$q_j$為dummy key(外部節點)機率。
        \item $w[1 \cdots \num(\text{Key}) + 1][0 \cdots \num(\text{Key})]$,初始化對角線上元素$w[j + 1, j]$為$q_j$。
        \item $e[1 \cdots \num(\text{Key}) + 1][0 \cdots \num(\text{Key})]$,初始化對角線上元素$e[j + 1, j]$為$q_j$。
        \item $r[1 \cdots \num(\text{Key})][1 \cdots \num(\text{Key})]$,$r[i, j]$存入$e[i, j]$中最小值對應的$r$。
        \item 理解:$e[i, r - 1]$為左子樹,$e[r + 1, j]$為右子樹,$w[i, j]$為節點權重和,因為計算cost時是節點階層加一。
    \end{itemize}
\end{theorem}
