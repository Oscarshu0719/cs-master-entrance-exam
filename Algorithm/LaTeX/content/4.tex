\item \begin{theorem}{(171, 178, 183, 193, 195)} Shortest path: \begin{itemize}
        \item Floyd-Warshall:sparse時,也不能提升性能。
        \item Johnson's在sparse時,性能較Floyd-Warshall好;Reweight後圖上所有邊權重皆$>0$,且最短路徑與原圖相同。
        \item Bellman-Ford:\begin{equation}
            D[v, k] = \min \{D[v, k - 1], \min_{(u, v) \in \E} \{D[u, k - 1] + wt(u, v)\}\}
        \end{equation}
        \item Floyd-Warshall:\begin{equation}
            D^k[i, j] = \min\{D^{k - 1}[i, j], D^{k - 1}[i, k] + D^{k - 1}[k, j]\}
        \end{equation}
    \end{itemize}
\end{theorem}

\item \begin{theorem}{()} Minimum vertex cover (tree):\begin{equation}
        \begin{aligned}
            V(v) = \min \{& 1 + \Sum\{V(c), \ \forall c \in v.child\}, \\
            & \len\{v.child\} + \Sum\{V(g), \ \forall c \in v.child \ \forall g \in c.child\}
        \}   
        \end{aligned}
    \end{equation}
\end{theorem}

\item \begin{theorem}{()} Max-cut: \begin{itemize}
        \item NPC。
        \item 若所有邊權重皆負,則可乘上$-1$,變為Min-cut。
        \item 若為平面圖,可轉換為Chinese Postman Problem(若為無向圖,即Euler circuit,若為有向圖,則為NPC)。
    \end{itemize}
\end{theorem}
