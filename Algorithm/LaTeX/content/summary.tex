\item \begin{theorem}{()} \quad\quad
    \begin{itemize}
        \item (\textbf{FALSE}) For two functions $f(n)$ and $g(n)$, either $f(n) = O(g(n))$ or $f(n) = \Omega((f(n))$.
        Counterexample:\begin{equation}
            \begin{aligned}
                f(n) & = \begin{cases}
                    1, \ \text{if} \ n = 2k \\
                    0, \ \text{if} \ n = 2k + 1
                \end{cases} \\
                g(n) & = \begin{cases}
                    0, \ \text{if} \ n = 2k \\
                    1, \ \text{if} \ n = 2k + 1
                \end{cases} 
            \end{aligned}
        \end{equation}
        \item For any uniform cost RAM program $T(n) = \Omega(S(n))$, where $S(n)$ is the space an algorithm uses for an input of size $n$.
        \item The capacity of each edge of a flow network can be floating-point, and it can be solved by linear programming.
        \item A flow network of multiple sources can be reduced to a single source.
        \item (\textbf{FALSE}) The value of any flow of a flow network is bounded by the capacity of only at most $O(n)$ cuts.
        \item 2-coloring: $O(n^2)$, 3-coloring, 4-coloring: superpolynomial.
        \item Weighted-union heuristic: Append the \textbf{smaller} list onto the \textbf{longer} list, with ties broken arbitrarily.
        \item $n! \neq \Theta(n^n)$.
    \end{itemize}
\end{theorem}
