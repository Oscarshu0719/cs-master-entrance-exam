\item \begin{theorem}{(137)} 1's補數,在overflow時,需額外加$1$修正。
\end{theorem}

\item \begin{theorem}{(141)} \code{sltu}可以用來判斷負數,因為會大於正數。
\end{theorem}

\item \begin{theorem}{(144)} \code{add, addi, sub} overflow時會exception,但\code{addu, addiu, subu}則不會,因為較多用於計算address。
\end{theorem}

\item \begin{theorem}{(165, 167)} 無號數乘法:\begin{itemize}
        \item 傳統乘法:Multiplier右移,multiplicand初始放低位、左移且與product皆兩倍長。
        \item Hardware-friendly multiplication:Multiplicand一倍長,product兩倍長、初始放低位、加到高位且右移,捨去mulitplier。
    \end{itemize}
\end{theorem}

\item \begin{theorem}{(171, 173)} 有號數乘法:\begin{itemize}
        \item Booth's algorithm:\begin{itemize}
            \item 多一位mythical bit。
            \item Multiplicand一倍長,product兩倍長、初始為multiplier放低位且操作在高位、右移,無multiplier。
            \item $01$: $+$, $10$: $-$, $00, 11$: skip.
        \end{itemize}
        \item 快速乘法硬體:\begin{itemize}
            \item 沒有clock,屬於combinational unit,不消耗$1$個clock cycle。
            \item 容易optimize和pipeline。
        \end{itemize}
    \end{itemize}
\end{theorem}

\item \begin{theorem}{(177, 180)} 除法:\begin{itemize}
        \item 傳統除法:quotient一倍長且左移,divisor兩倍長、初始放高位且右移,remainder兩倍長且初始為dividend放低位。
        \item Hardware-friendly division:divisor一倍長,remainder兩倍長、初始為divident放低位、初始時左移$1$位、最後高位右移$1$位且左移,無quotient。
    \end{itemize}
\end{theorem}

\item \begin{theorem}{(183)} \quad\quad \begin{itemize}
        \item MIPS乘法:\begin{lstlisting}[language={[x86masm]Assembler}]
            mult $s2, $s3
            mfhi $s0
            mflo $s1
        \end{lstlisting} 
        \item MIPS除法:\begin{lstlisting}[language={[x86masm]Assembler}]
            # Lo = $s2 / $s3
            # Hi = $s2 % $s3
            div $s2, $s3
        \end{lstlisting}
    \end{itemize}
\end{theorem}

\item \begin{theorem}{(190)} 浮點數: \begin{itemize}
        \item \begin{equation}
            \begin{aligned}
                (-1)^{sign} \times & (0.fraction) \times 2^{(exponent - bias)} \\
                & bias = 2^{n - 1} - 1
            \end{aligned}
        \end{equation}
        \item Denormalized number:\begin{equation}
            (-1)^{sign} \times (0.fraction) \times 2^{-126}
        \end{equation}
        \begin{table}[H]
            \centering
            \begin{tabular}{|c|c|c|c|c|c|}
                \hline
                & Sign & Exponent & Fraction & Underflow & Overflow \\
                \Xhline{2\arrayrulewidth}
                Single precision & \multirow{2}{*}{$1$} & $8$ & $23$ & $>0, < 2 \times 10^{-38}$ & $> 2 \times 10^{38}$ \\
                \cline{1-1}\cline{3-6}
                Double precision &  & $11$ & $52$ & $>0, < 2 \times 10^{-308}$ & $> 2 \times 10^{308}$ \\
                \hline
            \end{tabular}
        \end{table}
        \begin{table}[H]
            \centering
            \begin{tabular}{|c|c|c|c|c|}
                \hline
                \multicolumn{2}{|c|}{Single precision} & \multicolumn{2}{c|}{Double precision} & Representation \\
                \Xhline{2\arrayrulewidth}
                Exponent & Fraction & Exponent & Fraction & \\
                \hline
                $0$ & $0$ & $0$ & $0$ & $\pm 0$ \\
                \hline
                $0$ & $\neq 0$ & $0$ & $\neq 0$ & $\pm$ denormalized number \\
                \hline
                $1 \sim 254$ & $\texttimes$ & $1 \sim 2046$ & $\texttimes$ & $\pm$ floating-point number \\
                \hline
                $255$ & $0$ & $2047$ & $0$ & $\pm \infty$ \\
                \hline
                $255$ & $\neq 0$ & $2047$ & $\neq 0$ & NaN \\
                \hline
            \end{tabular}
        \end{table}
    \end{itemize}
\end{theorem}

\item \begin{theorem}{(205)} 四捨五入:\begin{table}[H]
        \centering
        \begin{tabular}{|c|c|c|c|c|}
            \hline
            LSB & Guard & Round & Sticky & Action \\
            \Xhline{2\arrayrulewidth}
            $0$ & $1$ & $0$ & $0$ & $\downarrow$ \\
            \hline
            $0$ & $1$ & $0$ & $1$ & $\uparrow$ \\
            \hline
            $1$ & $1$ & $0$ & $0$ & $\uparrow$ \\
            \hline
            $1$ & $1$ & $0$ & $1$ & $\uparrow$ \\
            \hline
        \end{tabular}
    \end{table}
\end{theorem}

\item \begin{theorem}{(213)} 右移代替除法:若為負數需先加上$2^n - 1$再位移,其中$n$為要右移的位數。
\end{theorem}
