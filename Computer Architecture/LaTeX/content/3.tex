\item \begin{theorem}{(284)} 軟硬體影響性能:
    \begin{table}[H]
        \centering
        \begin{tabular}{|c|c|c|c|}
            \hline
             & IC & CPI & Clock rate \\
            \Xhline{2\arrayrulewidth}
            Algorithm & $\surd$ & $\surd$ & \\
            \hline
            Programming language & $\surd$ & $\surd$ & \\
            \hline
            Compiler & $\surd$ & $\surd$ & \\
            \hline
            ISA & $\surd$ & $\surd$ & $\surd$ \\
            \hline
            Computer organization & & $\surd$ & $\surd$ \\
            \hline
            VLSI & & & $\surd$ \\
            \hline
        \end{tabular}
    \end{table}
\end{theorem}

\item \begin{theorem}{(284)} \quad\quad
    \begin{table}[H]
        \centering
        \begin{tabular}{|c|c|c|c|}
            \hline
             & IC & CPI & Clock rate \\
            \Xhline{2\arrayrulewidth}
            RISC & 大 & 小 & 小 \\
            \hline
            CISC & 小 & 大 & 大 \\
            \hline
        \end{tabular}
    \end{table}
\end{theorem}

\item \begin{theorem}{(292)} MIPS: \begin{itemize}
        \item \begin{equation}
            \begin{aligned}
                MIPS & = \frac{\text{IC}}{\text{execution time} \times 10^6} \\
                & = \frac{\text{clock rate}}{\text{CPI} \times 10^6}
            \end{aligned}
        \end{equation}
        \item MIPS不考慮指令的能力。
        \item 在同一台電腦上的程式,MIPS也可能不同。
        \item MIPS可能與效能成反比。
    \end{itemize} 
\end{theorem}

\item \begin{theorem}{(306)} \quad\quad \begin{itemize}
        \item \begin{equation}
            SPECratio = \frac{\text{execution time referenced}}{\text{execution time measured}}
        \end{equation} SPECratio越大代表效能越好。
        \item 若$A$以$B$為基準取norm,值為$\frac{B}{A}$。
        \item 先取geometric mean再norm和先norm再geometric mean結果相同。
        \item Geometric mean和execution time、norm的基準皆無關。
        \item Geometric mean無法預測execution time。
    \end{itemize} 
\end{theorem}
