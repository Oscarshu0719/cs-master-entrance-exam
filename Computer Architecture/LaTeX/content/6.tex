\item \begin{theorem}{(25, 27, 38, 48, 124)} Cache:\begin{itemize}
        \item Write:\begin{itemize}
            \item Write-through:寫入cache和memory。
            \item Write-buffer:寫入cache和buffer且CPU繼續執行,當buffer滿時,CPU須等到buffer有空位。
            \item Write-back:只寫入cache。
            \item Write allocate:從memory拉進cache,在cache中修改。
            \item Write around (No write allocate):不拉進cache,只在memory中修改。
            \item 通常write-though配write around,write-back配write allocate。
        \end{itemize}
        \item Split cache通常有較差hit ratio,提升bandwidth,但不提升speed。
        \item 增加associativity:降低miss rate,增加hit time。
        \item L1 cache:注重減少hit time;L2 cache:注重減少miss ratio。
    \end{itemize}
\end{theorem}

\item \begin{theorem}{(71, 124)} TLB:\begin{itemize}
        \item fully associative、lower miss rate,且size較小,lower cost,並採用random。
        \item TLB miss result in TLB exception.
    \end{itemize}
\end{theorem}
