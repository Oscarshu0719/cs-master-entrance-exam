\item \begin{theorem}{(203)} \begin{equation}
        \text{Rotational time} = \frac{0.5}{\text{RPS}}
    \end{equation} 
\end{theorem}

\item \begin{theorem}{(207)} Falsh:\begin{itemize}
        \item NOT lost info if it loses power.
        \item Write is much slower than read.
        \item NOR flash:Read access time computation is similar to DRAM. 
        \item NAND flash:Read bandwidth computation is similar to disk. Much cheaper per GB.
    \end{itemize}
\end{theorem}

\item \begin{theorem}{(210, 213, 278, 289)} RAID:\begin{itemize}
        \item Dependability:服務品質。
        \item Reliabilitiy:時間區間內可以正確執行其功能的機率。
        \item Availability:某時間點可以正確執行其功能的機率。\begin{equation}
            \text{Availability} = \frac{\text{MTTF}}{\text{MTTF} + \text{MTTR}}
        \end{equation}
        \item Data stripping:Data分散到不同disks,一次data存取會使多個disks存取,提升performance。
        \item Redundancy:利用多餘disks提升availablity,但不能改善reliability。
        \item RAID 0:Stipping,非容錯,沒有多餘disks。
        \item RAID 1:Mirroring (shadowing),最貴,data總是有$2$copies,disks數量為RAID 0兩倍。
        \item RAID 2:Hamming code,Write需要讀取所有disks,從新計算Hamming code並寫入ECC disks,效率差。
        \item RAID 3:\begin{itemize}
            \item Bit-interleaved parity.
            \item 不做備份,花費較多時間恢復data,較少disks。
            \item 當$1$個disk出錯可救回來,多個則否。
            \item Availability cost為$\frac{1}{N}$,其中$N$為protection group disks數量。
        \end{itemize}
        \item RAID 4:Block-interleaved parity,只對protection group其中一disk做small reads。
        \item RAID 5:Distributed Block-interleaved parity,parity被分散到所有disks,Write就不會有單一disk瓶頸。
        \item Read latency:RAID 3最短;Write latency:RAID 0 或 RAID 3最短。
        \item RAID 3 has worst throuhput for small writes.
        \item RAID 3, 4, 5 have same throughput for large writes.
        \item RAID 4, 5 perform same for parallel small reads and small writes, but RAID 3 can NOT for both access.
    \end{itemize}
\end{theorem}