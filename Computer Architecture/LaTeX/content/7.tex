\item \begin{theorem}{(213, 278, 289)} RAID:\begin{itemize}
        \item Data stripping:Data分散到不同disks,一次data存取會使多個disks存取,提升performance,但不能改善reliability。
        \item Redundancy:提升availablity,但\textbf{不能改善reliability}。
        \item RAID 0:Block-stripping,非容錯,沒有多餘disks。
        \item RAID 1:Mirroring (shadowing),最貴,data總是有額外一個copy,$2n$ disks。
        \item RAID 2:Hamming code,Write需要讀取所有disks,從新計算Hamming code並寫入ECC disks,效率差,$2n - 1$ disks。
        \item RAID 3:\begin{itemize}
            \item Bit-interleaved parity.
            \item Reliability和RAID 2相同。
            \item 不做備份,花費較多時間恢復data,$n + 1$ disks。
            \item 當$1$個disk出錯可救回來,多個則否。
            \item Availability cost為$\frac{1}{N}$,其中$N$為protection group disks數量。
            \item Parity集中存放一個disk。
        \end{itemize}
        \item RAID 4:\begin{itemize}
            \item Block-interleaved parity。
            \item 只對protection group其中一disk做small reads。
            \item $n + 1$ disks,parity集中存放一個disk。
            \item 當$1$個disk出錯可救回來,多個則否。
        \end{itemize}
        \item RAID 5:\begin{itemize}
            \item Distributed Block-interleaved parity
            \item Write就不會有單一disk瓶頸。
            \item $n + 1$ disks,parity被分散到所有disks。
        \end{itemize}
        \item RAID 0+1:一個disk壞,需整組copy替換;而RAID 1+0,只需該disk的copy替換。
        \item Read latency:RAID 3最短;Write latency:RAID 0 或 RAID 3最短。
        \item RAID 3 has worst throughput for small writes.
        \item RAID 3, 4, 5 have same throughput for large writes.
        \item RAID 4, 5 perform same for parallel small reads and small writes, but RAID 3 can NOT for both access.
    \end{itemize}
\end{theorem}