\item \begin{theorem}{(309, 319)} Shared-memory Multiprocessor (SMP):\begin{itemize}
        \item UMA:通過bus傳輸,因此同質CPU;NUMA:Netwrok傳輸,can scale to larger sizes and have lower latency to local memory。
        \item Snooping解決UMA和NUMA中cache coherence problem:\begin{itemize}
            \item Write-invalidate:先將其他caches invalidate後,在update自己的cache,降低bus bandwidth需求。
            \item Write-update (broadcast):通過bus broadcast最新的值。
        \end{itemize}
    \end{itemize}
\end{theorem}

\item \begin{theorem}{(325)} Multithreading: \begin{itemize}
        \item Fine-grained multithreading:Switches between threads on each instruction pack, but slows down the execution of individual threads. 隱藏long/short stalls造成的throughput損失。
        \item Coarse-grained multithreading:Switches only on costly stalls, e.g. L2 cache misses, but cost at startup because of flush and refill on thread switches.
        \item Simultaneous Multithreading (SMT):Hardware multithreading. Multiple-issue, dynamically scheduled processor (superscalar) to exploit ILP and Thread-level parallelism (TLP).
    \end{itemize}
\end{theorem}

\item \begin{theorem}{(332)} GPU:\begin{itemize}
        \item NOT rely on multilevel caches, but rely on enough threads to hide the latency to memory.
        \item GPU towards bandwidth rather than latency.
        \item NO support for double precision floating-point arithmetic.
    \end{itemize}
\end{theorem}
