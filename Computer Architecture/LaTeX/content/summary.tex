\item \begin{theorem}{()} NAS vs SAN: \begin{itemize}
        \item NAS operates at \textbf{file} level while SAN operates at \textbf{block} level.
        \item CIFS/SMB and NFS are examples of NAS.
        \item SAN is often the preferred choice over NAS.
        \item Almost any machine running Microsoft Windows with LAN connectivity can be configured to access a NAS.
    \end{itemize}
\end{theorem}

\item \begin{theorem}{()} Log-structured file system: \begin{itemize}
        \item 將要write的data合成一串,再一次write。
        \item Read都在cache,因為cache夠大。
        \item Disk access的seek和rotation是bottleneck,sequential access比random access好。
    \end{itemize}
\end{theorem}

\item \begin{theorem}{()} \quad\quad \begin{itemize}
        \item Out-of-order execution in \textbf{cache} level do NOT fail.
        \item GPGPU usually runs \textbf{SPMT} (Single Program Multiple Thread), and GPU runs SIMT.
        \item L1 data cache is usually seperated from L1 instruction cache to \textbf{increase bandwidth}.
        \item Data cache is usually deployed at \textbf{MEM} stage.
        \item Increasing number of \textbf{used sticky bits} do NOT improve accuracy.
        \item \textbf{Memory hazard} do NOT cause stall, e.g. \code{sw} after \code{lw}.
        \item Branch target buffer is used by \textbf{CPU}.
        \item Program is a \textbf{passive} entity, process is an \textbf{avtive} entity.
        \item Branch prediction buffer is good to predict the \textbf{branch outcome}, but it does \textbf{NOT} help in predicting the \textbf{branch target}.
        \item Many routers are equipped with \textbf{firewall} and \textbf{VPN} functions.
        \item \textbf{Static} power dissipation occurs because of leakage current that flows even when a transistor is \textbf{off}.
        \item In hash-based page tables using linked list to solve collision, \textbf{each element} contains a frame number and a page number.
        \item Multiple-cycles CPU requires \textbf{minimum function units}.
        \item Control hazards can \textbf{NOT} be avoided.
        \item \code{jr} is R-type.
        \item Conversion from single-precision to double-precision causes loss of precision.
        \item Compiler identifies \textbf{basic blocks} for code optimization.
        \item Vector processors need \textbf{less bandwidth} than conventional processors.
    \end{itemize}
\end{theorem}
