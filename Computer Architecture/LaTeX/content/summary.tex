\item \begin{theorem}{()} NAS vs SAN: \begin{itemize}
        \item NAS operates at \textbf{file} level while SAN operates at \textbf{block} level.
        \item CIFS/SMB and NFS are examples of NAS.
        \item SAN is often the preferred choice over NAS.
        \item Almost any machine running Microsoft Windows with LAN connectivity can be configured to access a NAS.
    \end{itemize}
\end{theorem}

\item \begin{theorem}{()} Log-structured file system: \begin{itemize}
        \item 將要write的data合成一串,再一次write。
        \item Read都在cache,因為cache夠大。
        \item Disk access的seek和rotation是bottleneck,sequential access比random access好。
    \end{itemize}
\end{theorem}

\item \begin{theorem}{()} \quad\quad \begin{itemize}
        \item Meltdown: read arbitrary kernel memorya, and it does \textbf{NOT} rely on software vulnerabilities.
        \item Spectre: Making other applications to access arbitrary contents in memory.
        \item Both belongs to \textbf{side channel attacks}.
        \item Does \textbf{NOT} leave records in traditional log.
        \item Hard for antivirus software to detect them.
        \item Processors which are able to implementat out-of-order execution is risky.
        \item IA-64 is immune to Spectre and Meltdown. 
    \end{itemize}
\end{theorem}

\item \begin{theorem}{()} Power: \begin{itemize}
        \item CMOS does \textbf{NOT} consume power when it's \textbf{static} ($power_{static} = 0$), so it can decrease \textbf{frequency} to save power.
        \item \textbf{Static} power dissipation occurs because of leakage current that flows even when a transistor is \textbf{off}.
        \item Computers at \textbf{lower utilization} does \textbf{NOT} use less power proportionally.
        \item The main reason for the switch from high-performance uniprocessors to multiprocessors with simpler cores and lower clock rates in recent years is the \textbf{power limit} and \textbf{memory gap}.
    \end{itemize}
\end{theorem}

\item \begin{theorem}{()} Cache: \begin{itemize}
        \item L1 data cache is usually seperated from L1 instruction cache to \textbf{increase bandwidth}.
        \item Data cache is usually deployed at \textbf{MEM} stage.
        \item In modern preocessors, \textbf{L1} data and instruction caches are split, but L2 does \textbf{NOT}. Both L1 and L2 caches are \textbf{write-back}.
    \end{itemize}
\end{theorem}

\item \begin{theorem}{()} Branch prediction: \begin{itemize}
        \item Branch target buffer is used by \textbf{CPU}, which is checked at \textbf{IF} stage.
        \item Branch prediction buffer is good to predict the \textbf{branch outcome}, but it does \textbf{NOT} help in predicting the \textbf{branch target}.
        \item Indirect branch prediction: Dynamic: hybrid predictor; Static: Neural branch predictor.
        \item \textbf{Virtual program counter prediction} is often used to predict \textbf{conditional/unconditional indirect} branch, which treats indirect branches as \textbf{multiple conditional branches}.
    \end{itemize}
\end{theorem}

\item \begin{theorem}{()} Hazards: \begin{itemize}
        \item \textbf{Memory hazard} do NOT cause stall, e.g. \code{sw} after \code{lw}.
        \item Control hazards can \textbf{NOT} be avoided.
    \end{itemize}
\end{theorem}

\item \begin{theorem}{()} Page table: \begin{itemize}
        \item In hash-based page tables using \textbf{linked list} to solve collision, \textbf{each element} contains a frame number and a page number.
        \item MIPS uses \textbf{two} seperated page tables and two limit registers, one for \textbf{stack} and the other for \textbf{heap}.
    \end{itemize}
\end{theorem}

\item \begin{theorem}{()} Arithmetic: \begin{itemize}
        \item Increasing number of \textbf{used sticky bits} do NOT improve accuracy.
        \item Conversion from single-precision to double-precision causes loss of precision.
        \item Ripple Carry Adder: Critical path delay is $2N$ gate delay (carry out), and sum delay is $2N + 1$ gate delay (actual sum).
        \item Converting an integer variable to a \textbf{single} precision FP number will lose precision, but \textbf{double} precision does \textbf{NOT}.
    \end{itemize}
\end{theorem}

\item \begin{theorem}{()} Multi-threading: \begin{itemize}
        \item GPGPU usually runs \textbf{SPMT} (Single Program Multiple Thread), and GPU runs SIMT.
        \item Vector processors need \textbf{less bandwidth} than conventional processors.
        \item GPUs do \textbf{NOT} rely on \textbf{multilevel caches}.
    \end{itemize}
\end{theorem}

\item \begin{theorem}{()} \quad\quad \begin{itemize}
        \item Out-of-order execution in \textbf{cache} level do NOT fail.
        \item Program is a \textbf{passive} entity, process is an \textbf{avtive} entity.
        \item Multiple-cycles CPU requires \textbf{minimum function units}.
        \item Compiler identifies \textbf{basic blocks} for code optimization.
        \item To form the machine code, the value of label of branch instructions is computed by \textbf{linker} when the label is an \textbf{external} reference.
        \item \textbf{NOT} each computer support \textbf{direct addressing mode}.
        \item \textbf{Conflict} misses do \textbf{NOT} occur in \textbf{fully associative} caches.
        \item \textbf{MIPS} and \textbf{ARM} use \textbf{memory-mapped I/O}.
        \item Writes are much \textbf{slower} than reads for flash. NAND flash is \textbf{cheaper} than NOR flash.
        \item Difficulty to handle \textbf{exceptions} (from most difficult to simplest): \begin{table}[H]
            \centering
            \begin{tabular}{|c|}
                \hline
                Superscalar \\
                \hline
                Speculative \\
                \hline
                Out-of-order \\
                \hline
                Pipelined \\
                \hline
                Single-issue in-order processor \\
                \hline
                Hierarchical data caches \\
                \hline
            \end{tabular}
        \end{table}
        \item Difficulty to handle \textbf{interrupts} (from most difficult to simplest): \begin{table}[H]
            \centering
            \begin{tabular}{|c|}
                \hline
                GPGPU \\
                \hline
                Containers \\
                \hline
                Virtual machines \\
                \hline
                Hyper-threaded processor \\
                \hline
                Superscalar \\
                \hline
                Pipelined \\
                \hline
            \end{tabular}
        \end{table}
    \end{itemize}
\end{theorem}
