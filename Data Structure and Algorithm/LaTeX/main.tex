\documentclass[12pt]{article}

\usepackage{algorithm}
\usepackage{float}
\usepackage{algpseudocode}
\usepackage{algorithmicx}
\usepackage{algpseudocode}
\usepackage{answers}
\usepackage{authblk}
\usepackage{ctex}
\usepackage{setspace}
\usepackage{graphicx}
\usepackage{enumitem}
\usepackage{multicol}
\usepackage{mathrsfs}
\usepackage[margin=1in]{geometry} 
\usepackage{amsmath,amsfonts,amsthm,amssymb}
\usepackage{mathtools}
\usepackage{multirow}
\usepackage{xcolor}
\usepackage{makecell}
\usepackage{url}
\usepackage{fontspec}
\usepackage{hyperref}
\usepackage[bottom]{footmisc}
\usepackage{hologo}
\usepackage{wallpaper}
\usepackage{makecell}

\CenterWallPaper{.50}{img/zju_logo.png}
\hypersetup{
    colorlinks=true,
    linkcolor=blue,
	urlcolor=magenta,
	citecolor=red      
}

\newcommand{\ver}{\textmd{Version} 1.0.1} % Version number.

\newcommand{\mat}[1]{\boldsymbol{#1}}
\newcommand{\spc}[1]{\textit{#1}}
\renewcommand{\vec}[1]{\boldsymbol{#1}}

\newcommand{\N}{\mathbb{Z}^+}
\newcommand{\Z}{\mathbb{Z}}
\newcommand{\Q}{\mathbb{Q}}
\newcommand{\Qp}{\mathbb{Q}^+}
\renewcommand{\C}{\mathbb{C}}
\newcommand{\R}{\mathbb{R}}
\newcommand{\F}{\textit{F}}
\newcommand{\T}{\textit{T}}
\renewcommand{\P}{\textit{P}}
\newcommand{\J}[1]{\textit{J}_{\mat{#1}}}
\newcommand{\V}{\spc{V}}
\newcommand{\E}{\spc{E}}
\renewcommand{\L}{\mathcal{L}}

\newcommand{\inv}{^{\mathsf{-1}}}
\newcommand{\herm}{^{\mathsf{H}}}

\newcommand{\adj}[1]{\operatorname{adj}(\mat{#1})}
\DeclareMathOperator{\spn}{span}
\DeclareMathOperator{\rnk}{rank}
\DeclareMathOperator{\nul}{nullity}
\DeclareMathOperator{\Ker}{N} % Capital, since \ker is already defined.
\DeclareMathOperator{\lker}{Lker}
\DeclareMathOperator{\im}{Im}
\DeclareMathOperator{\cs}{CS}
\DeclareMathOperator{\rs}{RS}
\DeclareMathOperator{\tr}{tr}
\DeclareMathOperator{\cof}{cof}
\DeclareMathOperator{\am}{am}
\DeclareMathOperator{\gm}{gm}
\DeclareMathOperator{\len}{Length}
\DeclareMathOperator{\num}{Number}
\newcommand{\ii}{{i\mkern1mu}}
\newcommand{\dotp}[2]{<#1, #2>}
\newcommand{\proj}[2]{\operatorname{proj}_{#1}\vec{#2}}
\newcommand{\Mod}[1]{\ (\mathrm{mod}\ #1)}
\DeclarePairedDelimiter{\ceil}{\lceil}{\rceil}
\DeclarePairedDelimiter{\floor}{\lfloor}{\rfloor}

\newcommand{\tnr}{\fontspec{Times New Roman}}
\newcommand{\con}{\fontspec{Consolas}}
 
\newenvironment{theorem}[2][Theorem]{\begin{trivlist}
\item[\hskip \labelsep {\bfseries #1}\hskip \labelsep {\bfseries #2}]}{\end{trivlist}}

\renewcommand{\contentsname}{Table of Contents}
\renewcommand{\refname}{References}
\renewcommand{\tablename}{Table}
\renewcommand{\figurename}{Figure}

\begin{document}

\title{\Huge{\textbf{資料結構和演算法}} \\
	\LARGE{\textbf{Data Structure and Algorithm}}
}
\newcommand*{\affaddr}[1]{#1}
\newcommand*{\affmark}[1][*]{\textsuperscript{#1}}
\author{
	TZU-CHUN HSU\affmark[1] \\
	\affmark[1]\href{mailto:vm3y3rmp40719@gmail.com}{vm3y3rmp40719@gmail.com} \\
	\affaddr{\affmark[1]Department of Computer Science, Zhejiang University
	}
}

\date{\mbox{}\vfill\today\\ \ver}

\maketitle
\pagebreak

\addcontentsline{toc}{section}{Disclaimer}
\begin{center}
    \Huge{\texttt{Disclaimer}}\\
\end{center}

本文「演算法」為台灣研究所考試入學的「演算法」考科使用,內容主要參考洪捷先生的演算法參考書\cite{1},以及wjungle網友在PTT論壇上提供的演算法筆記\cite{2}。 \\
本文作者為\textsc{TZU-CHUN HSU},本文及其\hologo{LaTeX}相關程式碼採用\textbf{MIT協議},更多內容請訪問作者之\textsc{GitHub}分頁\href{https://github.com/Oscarshu0719}{Oscarshu0719}。 \\~\\

\con
MIT License

Copyright (c) 2020 TZU-CHUN HSU

Permission is hereby granted, free of charge, to any person obtaining a copy
of this software and associated documentation files (the "Software"), to deal
in the Software without restriction, including without limitation the rights
to use, copy, modify, merge, publish, distribute, sublicense, and/or sell
copies of the Software, and to permit persons to whom the Software is
furnished to do so, subject to the following conditions:

The above copyright notice and this permission notice shall be included in all
copies or substantial portions of the Software.

THE SOFTWARE IS PROVIDED "AS IS", WITHOUT WARRANTY OF ANY KIND, EXPRESS OR
IMPLIED, INCLUDING BUT NOT LIMITED TO THE WARRANTIES OF MERCHANTABILITY,
FITNESS FOR A PARTICULAR PURPOSE AND NONINFRINGEMENT. IN NO EVENT SHALL THE
AUTHORS OR COPYRIGHT HOLDERS BE LIABLE FOR ANY CLAIM, DAMAGES OR OTHER
LIABILITY, WHETHER IN AN ACTION OF CONTRACT, TORT OR OTHERWISE, ARISING FROM,
OUT OF OR IN CONNECTION WITH THE SOFTWARE OR THE USE OR OTHER DEALINGS IN THE
SOFTWARE.

\tnr
\pagebreak

\include{content/Summary}
\bibliographystyle{plain}
\bibliography{content/bibliography}
\addcontentsline{toc}{section}{References}


\end{document}
