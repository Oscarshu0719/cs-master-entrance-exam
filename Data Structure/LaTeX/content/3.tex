\item \begin{theorem}{(58, 59)} \quad\quad
    \begin{itemize}
        \item Infix:Compiler需scan多次,耗時。
        \item Postfix:Compiler\textbf{左到右}scan一次即可。
        \item Prefix:Compiler\textbf{右到左}scan一次即可,但效率比Postfix差,Infix轉Postfix只需\textbf{1}個stack,而Infix轉Prefix需要\textbf{2}個stack。
        \item Infix轉Postfix(Prefix):\begin{itemize}
            \item 手算:\begin{enumerate}
                \item 加上完整括號。
                \item 運算元取代最近的右(左)括號。
                \item 刪去左(右)括號。
                \item 從左至右即是Infix。
            \end{enumerate}
            \item 演算法:\begin{enumerate}
                \item 數字直接print。
                \item 運算元依照優先級比較;若小於等於stack中的運算元,pop直到大於為止;若大於或相同於stack中的運算元,push;若為$)$,pop直到$($為止,但不print。
                \begin{table}[H]
                    \centering
                    \begin{tabular}{|c|c|}
                        \hline
                        \multicolumn{2}{|c|}{Priority} \\
                        \Xhline{3\arrayrulewidth}
                        Priority & Operator \\
                        \Xhline{2\arrayrulewidth}
                        1 & $($ out of stack \\
                        \hline
                        2 & $\uparrow$ out of stack \\
                        \hline
                        3 & $\uparrow$ in stack \\
                        \hline
                        4 & $*$, $/$ \\
                        \hline
                        5 & $+$, $-$ \\
                        \hline
                        6 & empty stack, $($ in stack \\
                        \hline
                    \end{tabular}
                \end{table}
            \end{enumerate}
        \end{itemize}
    \end{itemize}
\end{theorem}

\item \begin{theorem}{(80, 81)} Stack與Queue相互製作:\begin{itemize}
        \item 利用Stack製作Queue:\begin{itemize}
            \item Enqueue:利用Push代替。
            \item Dequeue:額外使用一個Stack,將原本Stack全部Pop並且Push到另一個Stack,最後Pop掉top。
        \end{itemize}
        \item 利用Queue製作Stack:\begin{itemize}
            \item Push:利用Enqueue代替。
            \item Pop:除了rear皆Dequeue並且Enqueue進Queue,最後Dequeue掉front(原本的rear)。
        \end{itemize}
    \end{itemize}
\end{theorem}
