\item \begin{theorem}{(221, 223, 224, 225, 227)} Hashing: \begin{itemize}
        \item Uniform hashing function:使資料量$n$大致平均分布在所有$B$個bucket,每個bucket內資料量大約$\frac{n}{B} = \alpha$,則\begin{itemize}
            \item 成功搜尋平均比較次數為$\frac{1 + 2 + \cdots + \alpha}{\alpha} = \frac{1 + \alpha}{2} \approx 1 + \frac{\alpha}{2}$。
            \item 失敗搜尋平均比較次數為$\alpha$。
        \end{itemize}
        \item Linear probing:易發生primary clustering problem,即相同hashing address的data易儲存在附近,增加searching time。
        \item Quadratic probing:Overflow發生時,改變hashing function為\begin{equation}
            (H(x) \pm i^2) \ \% \ B, \ \forall i = 1, 2, \ \cdots, \ceil{\frac{B - 1}{2}}
        \end{equation} 其中$B$為bucket數,$i$找到有空bucket或是所有格皆滿為止。解決primary clustering problem,但易發生secondary clustering problem,即相同hashing address的data overflow probe的位置距規律性,增加searching time。
        \item Double hashing:Overflow發生時,改變hashing function為\begin{equation}
            \begin{aligned}
                & (H(x) + i \times H'(x)) \ \% \ B, \ \forall i = 1, 2, \ \cdots \\
                & H'(x) = R - (x \ \% \ R), \ R \ \text{is prime}
            \end{aligned}
        \end{equation} 其中$B$為bucket數,$i$找到有空bucket或是所有格皆滿為止。解決secondary clustering problem,但不保證table充分利用。
    \end{itemize}
\end{theorem}