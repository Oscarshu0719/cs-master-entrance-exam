\item \begin{theorem}{(129, 130, 131)} Min-max heap: \label{min-maxheap}\begin{itemize}
        \item Complete。
        \item Root為最小值。
        \item 最大值在第二層其中一個。
        \item 越下層min-level越大,越下層max-level越小。
        \begin{algorithm}[H]
            \begin{algorithmic}[1]
                \Function{InsertMinMaxHeap}{MinMaxHeap $s$, Element $x$}
                    \State Put $x$ at the last position $n$, which has parent $p$.
                    \If {$p$ is at min-level} 
                        \If {$s[n].data < s[p].data$}
                            \State \Call{Swap}{$s[n]$, $s[p]$}
                            \State \Call{VerifyMin}{$s$, $p$, $x$}
                        \Else
                            \State \Call{VerifyMax}{$s$, $n$, $x$}
                        \EndIf
                    \Else \Comment{$p$ is at max-level.}
                        \If {$s[n].data > s[p].data$}
                            \State \Call{Swap}{$s[n]$, $s[p]$}
                            \State \Call{VerifyMax}{$s$, $p$, $x$}
                        \Else
                            \State \Call{VerifyMin}{$s$, $n$, $x$}
                        \EndIf
                    \EndIf
                \EndFunction
            \end{algorithmic}
        \end{algorithm} 
        \begin{algorithm}[H]
            \begin{algorithmic}[1]
                \Function{DeleteMinMinMaxHeap}{MinMaxHeap $s$}
                    \State Copy the data of the last node to the root and remove the last node.
                    \If {Root has no children}
                        \State Exit.
                    \ElsIf {Root has no grandchildren}
                        \If {Children $k$ is smaller than the root}
                            \State \Call{Swap}{root, $s[k]$}
                        \EndIf
                    \ElsIf {Min grandchildren $k$ and its parent $p$}
                        \If {root $> s[k]$}
                            \State \Call{Swap}{root, $s[k]$}
                            \If {root $> s[p]$}
                                \State \Call{Swap}{root, $s[p]$}
                                \State Recursively run the previous process.
                            \EndIf
                        \EndIf
                    \EndIf
                \EndFunction
            \end{algorithmic}
        \end{algorithm}
    \end{itemize}
\end{theorem}

\item \begin{theorem}{(133, 134, 135)} Deap(Double-ended heap): \label{deap}\begin{itemize}
        \item Complete。
        \item root不存data,root左子樹是min-heap,右子樹是max-heap。
        \item root左子樹中一節點必須$<$右子樹中對應的節點。
        \begin{algorithm}[H]
            \begin{algorithmic}[1]
                \Function{InsertDeap}{Deap $s$, Element $x$}
                    \State Put $x$ at the last position $n$.
                    \If {$n$ is at min-heap} 
                        \State $j$ is the corresponding position in the max-heap.
                        \If {$s[n].data > s[j].data$}
                            \State \Call{Swap}{$s[n]$, $s[j]$}
                            \State \Call{InsertMaxHeap}{$s$, $j$, $x$}
                        \Else
                            \State \Call{InsertMinHeap}{$s$, $n$, $x$}
                        \EndIf
                    \Else \Comment{$n$ is at max-heap.}
                        \State $j$ is the corresponding position in the min-heap.
                        \If {$s[n].data < s[j].data$}
                            \State \Call{Swap}{$s[n]$, $s[j]$}
                            \State \Call{InsertMinHeap}{$s$, $j$, $x$}
                        \Else
                            \State \Call{InsertMaxHeap}{$s$, $n$, $x$}
                        \EndIf
                    \EndIf
                \EndFunction
            \end{algorithmic}
        \end{algorithm} 
        \begin{algorithm}[H]
            \begin{algorithmic}[1]
                \Function{DeleteMinDeap}{Deap $s$}
                    \State Replace the data of the left child of the root with the smaller of its children and recursively run the same process to its subtree, making an empty node $n$ at the last level. 
                    \State Copy the data of the last node as $x$ and remove the node.
                    \State \Call{InsertDeap}{$x$} to position $n$.
                \EndFunction
            \end{algorithmic}
        \end{algorithm}
    \end{itemize}
\end{theorem}

\item \begin{theorem}{(136, 137)} SMMH(Symmetric min-max heap): \label{smmh}\begin{itemize}
        \item Complete。
        \item root不存data。
        \item 左兄弟節點$\le$右兄弟節點。
        \item 對一節點$x$,祖父節點的左子節點$\le x$,祖父節點的右子節點$\ge x$。
        \item 以一節點為root,則該子樹最小值(不含root)為左子節點,最大值(不含root)為右子節點。
        \begin{algorithm}[H]
            \begin{algorithmic}[1]
                \Function{InsertSMMH}{SMMH $s$, Element $x$}
                    \State Put $x$ at the last position.
                    \State Recursively swap those nodes which break the rules.
                \EndFunction
            \end{algorithmic}
        \end{algorithm}
        \begin{algorithm}[H]
            \begin{algorithmic}[1]
                \Function{DeleteMinSMMH}{SMMH $s$}
                    \State Copy the data of the last node as $x$ and remove the node.
                    \State Replace the left child of the root with the smaller of the leftmost grandchild and the third grandchild of the root and replace the chosen the node with $x$.
                    \State Recursively swap those nodes which break the rules.
                \EndFunction
            \end{algorithmic}
        \end{algorithm}
    \end{itemize}
\end{theorem}

\item \begin{theorem}{()} \quad\quad \begin{table}[H]
        \centering
        \begin{tabular}{|c|c|}
            \hline
            \multicolumn{2}{|c|}{Min-max heap, Deap, and SMMH} \\
            \Xhline{3\arrayrulewidth}
            Operation & Time complexity \\
            \Xhline{2\arrayrulewidth}
            Insert & $O(\log n)$ \\
            \hline
            Delete min/max & $O(\log n)$ \\
            \hline
            Find min/max & $O(1)$ \\
            \hline
        \end{tabular}
    \end{table} 
\end{theorem}

\item \begin{theorem}{(144)} Huffman's algorithm:$O(n\log n)$,採用Greedy,但可以求出最佳解。
\end{theorem}

\item \begin{theorem}{(145, 151)} AVL tree: \label{avltree}\begin{itemize}
        \item Height balanced BST.
        \item 平衡係數:左子樹高度減去右子樹高度。
        \item 左右子樹高度相差不超過$1$,即平衡係數只能為$-1$, $0$, $1$。
        \item 若不符合條件,根據父節點和祖父節點類型($LL$, $LR$, $RL$, $RR$)調整樹。
        \item 高度為$h$的AVL tree且節點數為$n$,則\begin{equation}
            F_{n + 2} - 1 \le n \le 2^h - 1
        \end{equation} 其中$F$是費氏數列,最大值為Full。 
    \end{itemize}
\end{theorem}

\item \begin{theorem}{(154, 155, 156, 158, 161)} $m$-way search tree: \label{m-wayst}\begin{itemize}
        \item 節點表示data block,從左到右為小到大,每個節點有$m - 1$個key。
        \item 用於external search/sort,資料量大時,需要分批載入search,因為無法全部放memory。
        \item B tree:\begin{itemize}
            \item Balanced $m$-way search tree。
            \item 所有failure nodes都在同一層。
            \item $m = 3$,$2$-$3$ tree;$m = 4$,$2$-$3$-$4$ tree。
            \item 若order $m$、高度$h$且節點數為$n$,則\begin{equation}
                1 + 2\frac{\ceil{\frac{m}{2}}^{h - 1} - 1}{\ceil{\frac{m}{2}} - 1} \le n \le 2\ceil{\frac{m}{2}}^{h - 1} - 1
            \end{equation}
            \item \begin{equation}
                \begin{aligned}
                    2 \le \deg(v) \le m &, v \ \text{is root} \\
                    \ceil{\frac{m}{2}} \le \deg(v) \le m &, v \ \text{is NOT root or failure nodes}
                \end{aligned}
            \end{equation}
            \begin{algorithm}[H]
                \begin{algorithmic}[1]
                    \Function{InsertBTree}{BTree $s$, Element $x$}
                        \State Put $x$ at proper position, which is at position $n$.
                        \While {$n$ overflow}
                            \State Choose the $\ceil{\frac{m}{2}}$-th key of $n$ (started from $1$), move it to its parent, and split $n$.
                            \State $n$ := $n.parent$
                        \EndWhile
                    \EndFunction
                \end{algorithmic}
            \end{algorithm}
            \begin{algorithm}[H]
                \begin{algorithmic}[1]
                    \Function{DeleteBTree}{BTree $s$, Element $x$}
                        \State $n$ := \Call{SearchBTree}{$s$, $x$}
                        \If {$n$ is leaf}
                            \State Delete $n$.
                            \While {$n$ underflow} 
                                \If {Can be rotated}
                                    \State Rotate.
                                    \State Break.
                                \Else
                                    \State Combine.
                                    \State $n$ := $n.parent$
                                \EndIf
                            \EndWhile
                        \Else \Comment{Non-leaf}
                            \State Replace $n$ with the max key of the left subtree, which is at position $m$.
                            \While {$m$ underflow} \Comment{Same as leaf deletion.}
                                \If {Can be rotated}
                                    \State Rotate.
                                    \State Break.
                                \Else
                                    \State Combine.
                                    \State $m$ := $m.parent$
                                \EndIf
                            \EndWhile
                        \EndIf
                    \EndFunction
                \end{algorithmic}
            \end{algorithm}
        \end{itemize}
    \end{itemize}
\end{theorem}

\item \begin{theorem}{(162)} Red-black tree: \label{rbt}\begin{itemize}
        \item BST.
        \item root和NIL皆黑色,紅色節點的兩個子節點必定是黑色。
        \item root到不同樹葉節點路徑上皆有相同數量黑色節點。
        \item 若一Red-black tree高度為$h$且節點數為$n$的,則\begin{equation}
            h \le 2\log (n + 1)
        \end{equation}
        \begin{algorithm}[H]
            \begin{algorithmic}[1]
                \Function{InsertRedBlackTree}{RedBlackTree $s$, Element $x$}
                    \State $x.color$ := $red$.
                    \State \Call{InsertBST}{$x$}.\
                    \While {$x.parent.color = red$}
                        \If {$x.parent$ is left child}
                            \If {$x.parent.sibling.color = red$}
                                \State $x.parent.color$ := $black$
                                \State $x.parent.sibling.color$ := $black$
                                \State $x.parent.parent.color$ := $red$
                                \State $x$ := $x.parent.parent$
                            \ElsIf {$x$ is right child}
                                \State $x$ := $x.parent$
                                \State \Call{LeftRotate}{$s$, $x$}
                            \Else
                                \State $x.parent.color$ := $black$
                                \State $x.parent.parent.color$ := $red$
                                \State \Call{RightRotate}{$s$, $x.parent.parent$}
                            \EndIf
                        \Else
                            \State Similar to the process above, just change \Call{LeftRotate}{} and \Call{RightRotate}{}.
                        \EndIf
                    \EndWhile
                    \State $s.root$ := $black$
                \EndFunction
            \end{algorithmic}
        \end{algorithm}
        \begin{algorithm}[H]
            \begin{algorithmic}[1]
                \Function{DeleteRedBlackTree}{RedBlackTree $s$, Element $x$}
                    \State $org-color$ := $x.color$
                    \If {$x$ is leaf} 
                        \State Set link from $x.parent$ to $x$ as NIL.
                    \ElsIf {$x$ has $1$ child}
                        \State Replace $x$ with its child. 
                    \Else \Comment{$x$ has $2$ children}
                        \State Replace $x$ with largest in left subtree or smallest in right subtree.
                    \EndIf
                    \If {$org-color = black$}
                        \State \Call{DeleteFixRedBlackTree}{$s$, $x$}
                    \EndIf
                \EndFunction
            \end{algorithmic}
        \end{algorithm}
        \begin{algorithm}[H]
            \begin{algorithmic}[1]
                \Function{DeleteFixRedBlackTree}{RedBlackTree $s$, Element $x$}
                    \While {$x \neq s.root \land x.color = black$}
                        \If {$x$ is left child}
                            \State $w$ := $x.sibling$
                            \If {$w.color = red$}
                                \State $w.color$ := $black$
                                \State $x.parent.color$ := $red$
                                \State \Call{LeftRotate}{$s$, $x.parent$}
                                \State $w$ := $x.sibling$
                            \ElsIf {$w.lchild.color = black \land w.rchild.color = black$}
                                \State $w.color$ := $red$
                                \State $x$ := $x.parent$
                            \ElsIf {$w.rchild.color = black$}
                                \State $w.lchild.color$ := $black$
                                \State $w.color$ := $red$
                                \State \Call{RightRotate}{$s$, $w$}
                                \State $w$ := $x.sibling$
                            \Else
                                \State $w.color$ := $x.parent.color$
                                \State $x.parent.color$ := $black$
                                \State $w.rchild.color$ := $black$
                                \State \Call{LeftRotate}{$s$, $x.parent$}
                                \State $x$ = $s.root$
                            \EndIf
                        \Else
                            \State Similar to the process above, just change \Call{LeftRotate}{} and \Call{RightRotate}{}.
                        \EndIf
                    \EndWhile
                    \State $x.color = black$
                \EndFunction
            \end{algorithmic}
        \end{algorithm}
    \end{itemize}
\end{theorem}
    
\item \begin{theorem}{(170, 171)} Splay Tree: \label{splaytree}\begin{itemize}
        \item BST.
        \item 每一次splay運算都將splay起點最終變為root。
        \item Rotation和AVL tree不同。
        \begin{algorithm}[H]
            \begin{algorithmic}[1]
                \Function{InsertSplayTree}{SplayTree $s$, Element $x$}
                    \State $n$ := \Call{InsertBST}{$x$}
                    \State \Call{Splay}{$n$}
                \EndFunction
            \end{algorithmic}
        \end{algorithm}
        \begin{algorithm}[H]
            \begin{algorithmic}[1]
                \Function{SearchSplayTree}{SplayTree $s$, Element $x$}
                    \State $n$ := \Call{SearchBST}{$x$}
                    \State \Call{Splay}{$n$}
                \EndFunction
            \end{algorithmic}
        \end{algorithm}
        \begin{algorithm}[H]
            \begin{algorithmic}[1]
                \Function{DeleteSplayTree}{SplayTree $s$, Element $x$}
                    \State $n$ := \Call{SearchBST}{$x$}
                    \State \Call{Splay}{$n$}
                    \State Remove $n$ and get its left and right subtrees $T_L$ and $T_R$.
                    \State $max$ := \Call{FindMaxBST}{$T_L$}
                    \State \Call{Splay}{$max$}
                    \State $max.rchild$ := $T_R$
                \EndFunction
            \end{algorithmic}
        \end{algorithm}
        \begin{table}[H]
            \centering
            \begin{tabular}{|c|c|c|}
                \hline
                \multicolumn{3}{|c|}{AVL tree, B tree, and Splay tree} \\
                \Xhline{3\arrayrulewidth}
                & AVL tree and B tree & Splay tree \\
                \Xhline{2\arrayrulewidth}
                Worst & $O(\log n)$ & $O(n)$ \\
                \hline
                Amortized & $O(\log n)$ & $O(\log n)$ \\
                \hline
            \end{tabular}
        \end{table}
    \end{itemize}
\end{theorem}

\item \begin{theorem}{()}
    \begin{table}[H]
        \centering
        \begin{tabular}{|c|c|}
            \hline
            \multicolumn{2}{|c|}{AVL tree, B tree, Red-black tree, and Splay tree} \\
            \Xhline{3\arrayrulewidth}
            Operation & Time complexity \\
            \Xhline{2\arrayrulewidth}
            Insert $x$ & \multirow{3}{*}{$O(\log_m n)$} \\
            \cline{1-1}
            Delete $x$ & \\
            \cline{1-1}
            Search $x$ & \\
            \hline
        \end{tabular}
    \end{table}
\end{theorem}

\item \begin{theorem}{(172, 173, 174)} Leftist heap: \label{leftistheap}\begin{itemize}
        \item \begin{equation}
            shortest(x) = \begin{cases}
                0 &, x \ \text{is external node} \\
                1 + \min\{shortest(x.lchild), shortest(x.rchild)\} &, x \ \text{is internal node}
            \end{cases}
        \end{equation}
        \item $\forall n \in$ leftist tree,$shortest(n.lchild) \ge shortest(n.rchild)$。
        \item Min(Max)-leftist heap:leftist tree and min(max)-tree.
        \item 一$n$個節點的leftist tree,root距離$\le \log (n + 1) - 1$。
        \begin{algorithm}[H]
            \begin{algorithmic}[1]
                \Function{MergeLeftistHeap}{LeftistHeap $s$, $t$}
                    \If {$s.data < t.data)$}
                        \State \Call{MergeLeftistHeap}{$s.rchild$, $t$}
                    \Else 
                        \State \Call{MergeLeftistHeap}{$t.rchild$, $s$}
                    \EndIf
                    \State Check the $shortest$ value of each node, if breaking the rule, swap the node and its sibling.
                \EndFunction
            \end{algorithmic}
        \end{algorithm}
        \begin{algorithm}[H]
            \begin{algorithmic}[1]
                \Function{DeltetMinLeftistHeap}{LeftistHeap $s$}
                    \State Remove root and get its left and right subtrees $T_L$ and $T_R$.
                    \State \Call{MergeLeftistHeap}{$T_L$, $T_R$}
                \EndFunction
            \end{algorithmic}
        \end{algorithm}
        \begin{algorithm}[H]
            \begin{algorithmic}[1]
                \Function{InsertLeftistHeap}{LeftistHeap $s$, Element $x$}
                    \State Let $x$ be a tree $n$.
                    \State \Call{MergeLeftistHeap}{$s$, $n$}
                \EndFunction
            \end{algorithmic}
        \end{algorithm}
        \begin{table}[H]
            \centering
            \begin{tabular}{|c|c|c|}
                \hline
                \multicolumn{3}{|c|}{Heap and Leftist heap} \\
                \Xhline{3\arrayrulewidth}
                Operation & Heap & Leftist heap \\
                \Xhline{2\arrayrulewidth}
                Insert $x$ & \multicolumn{2}{c|}{\multirow{2}{*}{$O(\log n)$}} \\
                \cline{1-1}
                Delete min & \multicolumn{2}{c|}{} \\
                \hline
                Merge one or two heaps & $O(n)$ & $O(\log n)$ \\
                \hline
            \end{tabular}
        \end{table}
    \end{itemize}
\end{theorem}

\item \begin{theorem}{(174, 175, 176)} Binomial heap: \label{binomialheap}\begin{itemize}
        \item root level為$0$。
        \item $B_k$為高度為$k$的binomial tree,由兩個高度$k - 1$的$B_{k - 1}$組成,其中$B_0$只有root一個節點。
        \item $B_k$第$i$level的節點數為$\binom{k}{i}$,總共$2^k$個節點。
        \item Binomial heap:一組binomial tree且皆為min-tree組成的forest。
        \begin{algorithm}[H]
            \begin{algorithmic}[1]
                \Function{MergeBinomialHeap}{BinomialHeap $s$, $t$}
                    \State Merge all trees with same height recursively by choosing the smaller root as new root.
                \EndFunction
            \end{algorithmic}
        \end{algorithm}
        \begin{algorithm}[H]
            \begin{algorithmic}[1]
                \Function{DeleteMinBinomialHeap}{BinomialHeap $s$}
                    \State Delete the smallest root from tree $p$ and get new trees $u$, and the others are $q$.
                    \State \Call{MergeBinomialHeap}{$q$, $u$}
                \EndFunction
            \end{algorithmic}
        \end{algorithm}
        \begin{algorithm}[H]
            \begin{algorithmic}[1]
                \Function{InsertBinomialHeap}{BinomialHeap $s$, Element $x$}
                    \State Let $x$ be a tree $n$.
                    \State \Call{MergeBinomialHeap}{$s$, $n$}
                \EndFunction
            \end{algorithmic}
        \end{algorithm}
    \end{itemize}
\end{theorem}

\item \begin{theorem}{(179)} Fibonacci heap: \label{fibheap}\begin{itemize}
        \item Binomial heap的superset,又稱Extended binomial heap。
        \item 比binomial heap多DeleteNode和DecreaseKey。
        \item 與binomial heap差異:\begin{itemize}
            \item insert與delete皆不合併。
            \item 所有節點用一個double-linked circular linked list連結起來,同時紀錄左右兄弟、父節點。
            \item DecreaseKey若使該節點小於其父節點,則將該子樹獨立出來。
        \end{itemize}
        \begin{table}[H]
            \centering
            \begin{tabular}{|c|c|c|}
                \hline
                \multicolumn{3}{|c|}{Binomial heap and Fibonacci heap} \\
                \Xhline{3\arrayrulewidth}
                Operation & Binomial heap & Fibonacci heap \\
                \Xhline{2\arrayrulewidth}
                Insert $x$ & $O(\log n), O(1)^*$ & $O(1)^*$ \\
                \hline
                Delete $x$/min & $O(\log n)$ & $O(\log n)^*$ \\
                \hline
                Merge & $O(\log n)$ & $O(1)^*$ \\
                \hline
                Decrease key & $O(\log n)$ & $O(1)^*$ \\
                \hline
                Find min & $O(\log n)$ & $O(1)$ \\
                \hline
                Remark & \makecell{Find min can be\\down to $O(1)$.} & \makecell{Decrease key is faster\\than binomial heap} \\
                \hline
            \end{tabular}
        \end{table}
    \end{itemize}
\end{theorem}
