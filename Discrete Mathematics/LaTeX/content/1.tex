\item \begin{theorem}{(1.6)} 
    \begin{subequations}
        \begin{align}
            A \oplus B & = (A \cup B) - (A \cap B) = (A - B) \cup (B - A) \\
            A - B & = A \cap \overline{B}
        \end{align}
    \end{subequations}
\end{theorem}

\item \begin{theorem}{(1.16)} \begin{subequations}
    \begin{align}
        \P(A \cup B) & \neq \P(A) \cup \P(B) \\
        \P(A \cap B) & = \P(A) \cap \P(B)
    \end{align}
    \end{subequations}
\end{theorem}

\item \begin{theorem}{(1.42)} \begin{equation}
    \floor*{-x} = - \ceil*{x}, \ \ceil*{-x} = -\floor*{x}
    \end{equation}
\end{theorem}

\item \begin{theorem}{(1.50)} 若$a \equiv b \Mod{n}, \ c \equiv d \Mod{n}$,則
    \begin{subequations}
    \begin{align}
        ac & \equiv bd \Mod{n} \\ 
        ab \Mod{n} & = ((a \Mod{n})(b \Mod{n})) \Mod{n}
    \end{align}
    \end{subequations}
\end{theorem}

\item \begin{theorem}{(1.55)} 若$a, b, c \in \N$,則$ax + by = c$有整數解$\iff$$\gcd(a, b) | c$。
\end{theorem}

\item \begin{theorem}{(1.58, 1.60)} (質數)
    \begin{itemize}
        \item 若$a \in \Z, n \in \N$,且$\gcd(a, n) = 1$,則$a^{-1} \Mod{n}$存在。
        \item 若$p$為質數,$a \in \Z$,則$a^{-1} \equiv a \Mod{p}$即$a^2 \equiv 1 \Mod{p}$$\iff$$a \equiv \pm 1 \Mod{p}$。
    \end{itemize}
\end{theorem}

\item \begin{theorem}{(1.62)} 若$a, b, c, n \in \Z$,若$\gcd(c, n) = 1$,則$ac \equiv bc \Mod{n}$$\iff$$a \equiv b \Mod{n}$。
\end{theorem}

\item \begin{theorem}{(1.61, 1.62)} (質數)
    \begin{itemize}
        \item Wilson's theorem:
        若$p$為質數,則
        \begin{equation}
            (p - 1)! \equiv -1 \Mod{p}
        \end{equation}
        \item Fermat's little theorem:
        若$p$為質數,$m \in Z$,且$\gcd(m, p) = 1$,則
        \begin{equation}
            m^{p - 1} \equiv 1 \Mod{p}
        \end{equation}
    \end{itemize}
\end{theorem}

\item \begin{theorem}{(1.64, 1.65, 1.66)} (質數)
    \begin{itemize}
        \item 若$n \in \N$,則Euler $\phi$-函數$\phi(n)$為$\{1, 2, \ \cdots, n - 1\}$中與$n$互質的元素個數,又稱Euler's totient function。
        \item 若$n = p_1^{e_1}p_2^{e_2}\cdots p_k^{e_k}$為$n$的質因數分解,則$\phi(n) = n\Pi_{j = 1}^{k}(1 - \frac{1}{p_j})$。
        \item 若$p \in \N$,則$\phi(p) = p - 1$$\iff$$p$為質數。
        \item 若$m \in \Z, n \in \N$,且$\gcd(m, n) = 1$,則$m^{\phi(n)} \equiv 1 \Mod{n}$。
        \item 若$p$為質數,$m \in Z$,且$\gcd(m, p) = 1$,則$m^{-1} \equiv m^{p - 2} \Mod{p}$
    \end{itemize}
\end{theorem}

\item \begin{theorem}{(1.66)} 中國餘數定理(Chinese Remainder Theorem,CRT)中,模的數之間必須互質。
\end{theorem}
