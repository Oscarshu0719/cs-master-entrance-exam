\item \begin{theorem}{(2.8, 2.10)} \quad\quad
    \begin{itemize}
        \item 若$A, B, C$為三集合,$R \subseteq A \times B, S \subseteq B \times C$為二關係,則$\mat{M}_{S \circ R} = \mat{M}_{RS} = \mat{M}_{R}\mat{M}_{S}$。
        \item 若$R$為一關係,則$\mat{M}_{R^{-1}} = \mat{M}_{R}^\intercal$。
    \end{itemize}
\end{theorem}

\item \begin{theorem}{(2.18, 2.25)} \quad\quad
    \begin{itemize}
        \item 非對稱性(asymmetric):若$A$為集合,$R \subseteq A \times A$為一關係,則$\forall a, b \in A$,$aRb \Rightarrow \lnot(bRa)$。
        \item 反對稱性(antisymmetric):若$A$為集合,$R \subseteq A \times A$為一關係,則$\forall a, b \in A$,$aRb \land bRa \Rightarrow a = b$。
        \item 非對稱性\textbf{不}允許對角線上元素為$1$,反對稱性則允許對角線上元素為$1$。
        \item 若$A$為集合,$R \subseteq A \times A$為一關係,則$R$具遞移性$\iff$$R^n \subseteq R, \ \forall n \in \N$。
        \item 若$A$為集合,$R, S \subseteq A \times A$為二關係,則
        \begin{itemize}
            \item 若$R, S$具反身性,則$R \cap S$與$R \cup S$亦具反身性。
            \item 若$R, S$具對稱性,則$R \cap S$與$R \cup S$亦具對稱性。
            \item 若$R, S$具遞移性,則$R \cap S$亦具遞移性。
        \end{itemize}
    \end{itemize}
\end{theorem}

\item \begin{theorem}{(2.42)} 若$P_i$表示$i \ge 1$個元素上相異的等價關係個數,假設$P_0 = 1$,則$P_n = \sum_{i = 0}^{n - 1}\binom{n - 1}{i}P_i$。
\end{theorem}

\item \begin{theorem}{(2.52, 2.53, 2.55)} \quad\quad
    \begin{itemize}
        \item 若$|A| = n$,則$\mat{M}_{t(R)} = \mat{M}_{R} \lor \mat{M}_{R}^2 \lor \cdots \lor \mat{M}_{R}^n$。
        \item $t(s(r(R)))$\textbf{未必}等於$t(R) \cup r(R) \cup r(R)$。
        \item 若$R_1, R_2$為二等價關係,其分割分別為$\pi_1$與$\pi_2$,則
            \begin{itemize}
                \item $R_1 \cap R_2$為等價關係,且分割為$\pi_1 \cdot \pi_2$。
                \item $R_1 \cup R_2$\textbf{未必}為等價關係,因為$R_1 \cup R_2$\textbf{未必}具遞移性,而$t(R_1 \cup R_2)$為等價關係,且分割為$\pi_1 + \pi_2$。
            \end{itemize}
    \end{itemize}
\end{theorem}

\item \begin{theorem}{(2.63, 2.66, 2.67, 2.69, 2.70, 2.71, 2.74)} \quad\quad
    \begin{itemize}
        \item 定義域中所有值都必須有對應到對應域的一個值。
        \item 若$f: \ A \rightarrow B$,$A_1, A_2 \subseteq A$,則
            \begin{itemize}
                \item 若$A_1 \subseteq A_2$,則$f(A_1) \subseteq f(A_2)$。
                \item $f(A_1 \cup A_2) = f(A_1) \cup f(A_2)$。
                \item $f(A_1 \cap A_2) \subseteq f(A_1) \cap f(A_2)$。
                \item $f$為一對一$\iff$$f(A_1 \cap A_2) = f(A_1) \cap f(A_2)$
            \end{itemize}
        \item 若$f: \ A \rightarrow B, g: \ B \rightarrow C$,則
            \begin{itemize}
                \item 若$f, g$為一對一,則$g \circ f$亦是一對一,反之不然。
                \item 若$f, g$為映成,則$g \circ f$亦是映成,反之不然。
                \item 若$f, g$可逆,則$g \circ f$亦可逆,反之不然。
                \item 若$g \circ f$為一對一,則$f$亦是一對一。
                \item 若$g \circ f$為映成,則$g$亦是映成。
            \end{itemize}
        \item 若$f: \ A \rightarrow B$,$B_1, B_2 \subseteq B$,則
            \begin{itemize}
                \item 若$B_1 \subseteq B_2$,則$f^{-1}(B_1) \subseteq f^{-1}(B_2)$。
                \item $f^{-1}(B_1 \cup B_2) = f^{-1}(B_1) \cup f^{-1}(B_2)$。
                \item $f^{-1}(B_1 \cap B_2) = f^{-1}(B_1) \cap f^{-1}(B_2)$。
            \end{itemize}
    \end{itemize}
\end{theorem}

\item \begin{theorem}{(2.96, 2.99, 2.100, 2.101, 2.102)} \quad\quad
    \begin{itemize}
        \item 若$A$為一集合,若$A = \emptyset$或$\exists n \in \N$使得$A \sim \{1, 2, \ \cdots, n\}$,則稱$A$為有限集。
        \item 若$A$為一集合,且$A$為有限集或$A \sim \N$,則稱$A$為可數集。
        \item 若$A, B$為可數集,則$A \times B$為可數集。
        \item 若$\{A_i\}_{i \in \N}$為可數集,則$\bigcup_{i \in \N}A_i$為可數集。
        \item $\Qp, \Q, \N \times \N$皆為可數集。
        \item $(0, 1) \sim \R, \R, \overline{\Q}, \C \sim \R$皆為不可數集。
        \item $|\mathbb{Z}| = |\N| = |\Q| < |\overline{Q}| = |\R| = |\C|$
    \end{itemize}
\end{theorem}
