\item \begin{theorem}{(2.101)} 證明$(0, 1)$為不可數集。 \\
    \begin{proof}
        $f: \N \rightarrow (0, 1)$ is bijective,令$f(i) = r_i, \ \forall i = 1, 2, 3, \cdots$其中 \\ \begin{equation}
            \begin{cases}
                r_1 = 0.r_{11}r_{12}\cdots \\
                r_2 = 0.r_{21}r_{22}\cdots \\
                \quad \quad \quad \vdots \\
                r_i = 0.r_{i1}r_{i2}\cdots \\
            \end{cases}
        \end{equation} 取 \begin{equation}
            s = 0.s_1s_2\cdots, \ s_i = \begin{cases}
                4 &, r_{ii} \neq 4 \\
                5 &, r_{ii} = 4
            \end{cases}
        \end{equation} $s_i \in (0, 1)$但$\nexists \ i \in \N$ s.t. $f(i) = s$,因此$(0, 1)$為不可數集。
    \end{proof}
\end{theorem}

\item \begin{theorem}{(.53)} \begin{equation}
        A = \{1, \ 2, \ \cdots, \ 2n\}
    \end{equation} 在$A$取$N + 1$個數,\begin{equation}
        \exists \ a, b \quad \text{s.t.} \ a | b \lor b | a
    \end{equation}
    \begin{proof}
        \begin{equation}
            \forall \ x \in A, \ x = 2^k \times y, \ k \in \Z, \ y = 2l + 1, \ l \in \Z
        \end{equation} 又$A$中只有$n$個奇數,則取$n + 1$個數時, \begin{equation}
            \begin{aligned}
                \exists \ a, b \quad & \text{s.t.} \ a = 2^{k_1} \times y, \ b = 2^{k_2} \times y \\
                & a | b \lor b | a
            \end{aligned}
        \end{equation}
    \end{proof}
\end{theorem}
