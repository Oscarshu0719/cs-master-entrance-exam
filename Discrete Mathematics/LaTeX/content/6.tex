\item \begin{theorem}{(6.6, 6.7, 6.8, 6.9, 6.10, 6.11, 6.12, 6.16, 6.18)} \quad\quad
    \begin{itemize}
        \item 不含重複\textbf{邊}的路(walk)稱作路線(trail)。封閉路線(trail)稱作迴路(circuit)。
        \item 不含重複\textbf{點}的路(walk)稱作路徑(path)。封閉路徑(path)稱作環路(cycle),其中重複的點不包含起終點,且至少含$3$邊。
        \item 單方向(unilaterally)連通圖:$\forall x, y \in \V, \ x \neq y$,在$x, y$之間存在一條由某一點到另一點的有向路徑。
        \begin{itemize}
            \item 強連通圖必為單方向連通圖,反之不然。
            \item 單方向連同圖必微弱連通圖,反之不然。
        \end{itemize}
        \item 誘導子圖(Induced subgraph):子圖所含原圖之點,皆保留其原有的邊。
        \item 一無向圖,且未必為連通圖,其極大連通誘導子圖稱連通分量圖(Connected component)。
        \item 若在一連通圖上去除一點,使該圖變為不連通圖,則稱該點為切點(cut point或articulation point)。
        \item 若在一連通圖上去除一邊集,使該圖變為不連通圖,且若去除該邊集的真子集不使該圖變為不連通圖,則稱該邊集為切集(cut set)。若切集只含一邊,則稱該邊為切邊(cut edge)或橋(bridge)。
        \item 雙連通圖:無迴圈的連通無向圖且\textbf{不含切點}。
        \item 一無迴圈的連通無向圖,且未必為雙連通圖,其極大雙連通誘導子圖稱雙連通分量圖(Biconnected component)。
        \item 補圖須包含所有點。
        \item 度數為一的點稱懸吊點(pendant)。
        \item 迴圈的度數為$2$。
        \item 一無向圖,所有點的度數皆為$k$,則稱$k$-規則($k$-regular)圖。
        \item 若$G = (\V, \E), \ |\V| = n$,則\begin{equation}
            |\E| \ge \binom{n - 1}{2} + 1
        \end{equation}
        時,$G$為連通圖。
        \item 團(clique):完全子圖,任兩點皆有邊相連。
        \item 獨立集(independent set):補圖的團,任兩點無邊相連。
        \item 支配集(dominating set):支配集中的點與圖上所有點皆有邊相連。
        \item 覆蓋(covering):覆蓋中的點與圖上所有邊相連。
    \end{itemize}
\end{theorem}

\item \begin{theorem}{(6.44, 6.45, 6.47, 6.55)} \quad\quad
    \begin{itemize}
        \item 一簡單無向圖,若所有點的度數至數為$k$,則圖上必含一個長度至少為$k + 1$的環路(cycle)。
        \item 若$G = (\V, \E), \ |\V| \ge 1$為一連通無向圖,則\begin{equation}
            |\E| \ge |\V| - 1
        \end{equation}
        \item 若$A$為一鄰接矩陣,則$A^r$的第$(i, j)$項為$v_i$到$v_j$長度為$r$路的個數。
        \item 一無向圖,為雙分圖$\iff$圖中不含奇數長度的環路。
        \item 若$A$為一鄰接矩陣,則$\frac{1}{6}\tr(A^3)$為圖上三角形個數。
    \end{itemize}
\end{theorem}

\item \begin{theorem}{(6.57, 6.59, 6.60, 6.62)} \quad\quad
    \begin{itemize}
        \item 若存在一路線經過圖中每一\textbf{邊}恰一次稱有尤拉路線(Euler trail)。
        \item 圖中有尤拉迴路$\iff$為連接圖且所有點的度數為偶數。
        \item $K_n$有尤拉迴路$\iff$$n$為奇數。
        \item $K_{m, n}$有尤拉迴路$\iff$$m, n$為偶數。
        \item 圖中有尤拉路線$\iff$為連通圖且圖中恰含$0$個或$2$個點度數為奇數。
        \item 圖中有尤拉迴路$\iff$為強連通圖且所有點的出度數與入度數相同。
    \end{itemize}
\end{theorem}

\item \begin{theorem}{(6.64, 6.68, 6.71, 6.72, 6.73, 6.74, 6.84)} \quad\quad
    \begin{itemize}
        \item 若存在一路境經過圖中每一\textbf{點}恰一次稱有漢米爾頓路徑(Hamiltonian path)。
        \item 漢米爾頓環路必要條件:
        \begin{itemize}
            \item 度數為$2$的點,與其相鄰的邊必在漢米爾頓環路中。
            \item 度數$\ge 2$的點,除了前一條件已知在漢米爾頓環路中的邊之外,其他的邊必不在漢米爾頓環路中。
        \end{itemize}
        \item 有向完全圖必有有向漢米爾頓路徑。
        \item 若$G = (\V, \E), \ |\V| = n \ge 3$為一無迴圈無向圖,
        \begin{itemize}
            \item 若$\deg(x) + \deg(y) \ge n - 1, \ \forall x, y \in \V, x \neq y$,則$G$有漢米爾頓路徑。
            \item 若$\deg(x) + \deg(y) \ge n, \ \forall x, y \in \V, x, y\text{不相鄰}$,則$G$有漢米爾頓環路。
            \item 若$\deg(v) \ge \frac{n - 1}{2}, \ \forall v \in \V$,則$G$有漢米爾頓路徑。
            \item 若$\deg(v) \ge \frac{n}{2}, \ \forall v \in \V$,則$G$有漢米爾頓環路。
        \end{itemize}
        \item $K_n, n \ge 3$必有漢米爾頓環路。
        \item 若一圖有漢米爾頓環路,則該圖中任兩點至少有兩條路徑相連。
        \item 一連通雙分圖,若圖中有漢米爾頓環路,則兩邊的頂點數相同。$K_{m, n}, \ \forall m, n \ge 2$有漢米爾頓環路$\iff$$m = n$。
        \item 一連通雙分圖,若圖中有漢米爾頓路徑,則兩邊的頂點數相差$\le 1$。$K_{m, n}, \ \forall m, n \ge 2$有漢米爾頓路徑$\iff$$|m - n| \le 1$。
        \item $K_n$有$\frac{(n - 1)!}{2}$個相異漢米爾頓環路。
        \item $K_n$,$n$為奇數,有$\le \frac{n - 1}{2}$個\textbf{不共邊}的漢米爾頓環路。
        \item $K_{n, n}$有$\frac{1}{2}n!(n - 1)!$個相異漢米爾頓環路。
        \item 若$G = (\V, \E), \ |\V| = n$,則\begin{equation}
            |\E| \ge \binom{n - 1}{2} + 2
        \end{equation}
        時,$G$有漢米爾頓環路。
    \end{itemize}
\end{theorem}

\item \begin{theorem}{(6.88, 6.90, 6.93, 6.94, 6.97, 6.98)} \quad\quad
    \begin{itemize}
        \item 一無迴圈無向圖,在圖中一邊中加上一頂點,使其變為二邊,稱基本區分(elementary subdivision)。
        \item 二無迴圈無向圖,若二圖同構或皆可由某個圖經由數次基本區分得到,稱二圖同胚(homeomorphic)。
        \item Kuratowski's theorem:平面圖$\iff$\textbf{不}含子圖與$K_5$或$K_{3, 3}$同胚。
        \item 當一個邊在一個區域出現二次,該邊的度數為二。
        \item Euler formula:若$G = (\V, \E), |\V| = v, |\E| = e, r\text{為區域個數}, M\text{為分量圖數}$為平面圖,則$v - e + r = 1 + M$。
        \item 若$G = (\V, \E), |\V| = v, |\E| = e \ge 2, r\text{為區域個數}$為無迴圈簡單\textbf{連通}平面圖,則\begin{itemize}
            \item \begin{equation}
                \frac{3}{2}r \le e \le 3v - 6
            \end{equation}
            \item 若$G$不含任何三角形,則\begin{equation}
                e \le 2v - 4
            \end{equation}
            \item 若每個環路$\ge k \ge 3$邊組成,則\begin{equation}
                e \le \frac{k}{k - 2}(v - 2)
            \end{equation}
        \end{itemize}
        \item 雙分圖不含三角形。
        \item 一無迴圈簡單平面圖必含一個度數$\le 5$的頂點。
    \end{itemize}
\end{theorem}

\item \begin{theorem}{(6.109, 6.110, 6.113, 6.114, 6.115, 6.116)} \quad\quad
    \begin{itemize}
        \item $\chi(K_n) = n$,$\chi(W_n) = 1 + \chi(C_n)$,$\chi(C_n) = \begin{cases}
            2, & n = 2k \\ 3, & n = 2k + 1
        \end{cases}$。
        \item 若$K_n$為一圖$G$的子圖,則$\chi(G) \ge n$。
        \item 雙分圖$\iff$$2$-著色$\iff$圖中不含奇數長的迴圈。
        \item 任意平面圖皆為$4$-著色,反之不然。
        \item 若$G = (\V, \E)$為無向圖,$\lambda$為顏色數,則稱$P(G, \lambda)$為著色多項式,表示至多使用$\lambda$種顏色著色的不同方法數,且$\chi(G) = \min\{\lambda | P(G, \lambda) > 0\}$。\begin{itemize}
            \item $P(G, \lambda)$常數項為$0$。
            \item $P(G, \lambda)$係數和為$0$。
            \item $P(G, \lambda)$最高次項係數為$1$。
        \end{itemize}
        \item 若$G = (\V, \E)$為連通無向圖,$e = \{a, b\} \in \E$,$G \cdot e$為$G - e$中將$a, b$二頂點黏合後的圖,則\begin{equation}
            P(G, \lambda) = P(G - e, \lambda) - P(G \cdot e, \lambda)
        \end{equation}
        \item 若$G = (\V, \E)$為連通無向圖,$e = \{a, b\} \notin \E$,$G \cdot e$為$G + e$中將$a, b$二頂點黏合後的圖,則\begin{equation}
            P(G, \lambda) = P(G + e, \lambda) + P(G \cdot e, \lambda)
        \end{equation}
        \item 若$G = (\V, \E)$為無向圖,且$G_1, G_2$為$G$子圖、$G = G_1 \cup G_2$、$G_1 \cap G_2 = K_n$,則\begin{equation}
            P(G, \lambda) = \frac{P(G_1, \lambda)P(G_2, \lambda)}{\lambda(\lambda - 1)\cdots(\lambda - (n - 1))}
        \end{equation}
    \end{itemize}
\end{theorem}
