\item \begin{theorem}{(6.6, 6.7, 6.8, 6.9, 6.10, 6.11, 6.12, 6.16, 6.18)} \quad\quad
    \begin{itemize}
        \item 若$G = (\V, \E), \ |\V| = n$,則\begin{equation}
            |\E| \ge \binom{n - 1}{2} + 1
        \end{equation}
        時,$G$為連通圖。
    \end{itemize}
\end{theorem}

\item \begin{theorem}{(6.44, 6.47, 6.55)} \quad\quad
    \begin{itemize}
        \item 一簡單無向圖,若所有點的度數至數為$k$,則圖上必含一個長度至少為$k + 1$的環路(cycle)。
        \item 若$A$為一鄰接矩陣,則$\frac{1}{6}\tr(A^3)$為圖上三角形個數。
    \end{itemize}
\end{theorem}

\item \begin{theorem}{(6.57, 6.59, 6.60, 6.62)} \quad\quad
    \begin{itemize}
        \item 圖中有尤拉迴路$\iff$為連接圖且所有點的度數為偶數。
        \item $K_n$有尤拉迴路$\iff$$n$為奇數。
        \item $K_{m, n}$有尤拉迴路$\iff$$m, n$為偶數。
        \item 圖中有尤拉路線$\iff$為連通圖且圖中恰含$0$個或$2$個點度數為奇數。
        \item 圖中有尤拉迴路$\iff$為強連通圖且所有點的出度數與入度數相同。
    \end{itemize}
\end{theorem}

\item \begin{theorem}{(6.64, 6.68, 6.71, 6.72, 6.73, 6.74, 6.84)} \quad\quad
    \begin{itemize}
        \item 有向完全圖必有有向漢米爾頓\textbf{路徑}。
        \item 若$G = (\V, \E), \ |\V| = n \ge 3$為一無迴圈無向圖,
        \begin{itemize}
            \item 若\begin{equation}
                \begin{aligned}
                    \deg(x) + \deg(y) & \ge n - 1, \ \forall x, y \in \V, x \neq y \ \lor \\
                    \deg(v) & \ge \frac{n - 1}{2}, \ \forall v \in \V
                \end{aligned}
            \end{equation}
            ,則$G$有漢米爾頓\textbf{路徑}。
            \item 若\begin{equation}
                \begin{aligned}
                    \deg(x) + \deg(y) & \ge n, \ \forall x, y \in \V, x, y\ \text{不相鄰} \ \lor \\
                    \deg(v) & \ge \frac{n}{2}, \ \forall v \in \V
                \end{aligned}
            \end{equation}
            ,則$G$有漢米爾頓\textbf{環路}。
        \end{itemize}
        \item $K_n, n \ge 3$必有漢米爾頓\textbf{環路}。
        \item 若一圖有漢米爾頓環路,則該圖中任兩點至少有兩條路徑相連。
        \item 一連通雙分圖,若圖中有漢米爾頓\textbf{環路},則兩邊的頂點數相同。
        \item 一連通雙分圖,若圖中有漢米爾頓\textbf{路徑},則兩邊的頂點數相差$\le 1$。
        \item $K_n$有$\frac{(n - 1)!}{2}$個相異漢米爾頓環路。
        \item $K_n$,$n$為奇數,有$\le \frac{n - 1}{2}$個\textbf{不共邊}的漢米爾頓環路。
        \item $K_{n, n}$有$\frac{1}{2}n!(n - 1)!$個相異漢米爾頓環路。
        \item 若$G = (\V, \E), \ |\V| = n$,則\begin{equation}
            |\E| \ge \binom{n - 1}{2} + 2
        \end{equation}
        時,$G$有漢米爾頓\textbf{環路}。
    \end{itemize}
\end{theorem}

\item \begin{theorem}{(6.88, 6.90, 6.93, 6.94, 6.97, 6.98)} \quad\quad
    \begin{itemize}
        \item Euler formula:若$G = (\V, \E), |\V| = v, |\E| = e, r\text{為區域個數}, M\text{為分量圖數}$,且$G$為平面圖,則$v - e + r = 1 + M$。
        \item 若$G = (\V, \E), |\V| = v, |\E| = e \ge 2, r\text{為區域個數}$,且$G$為無迴圈簡單\textbf{連通}平面圖,則\begin{itemize}
            \item \begin{equation}
                \frac{3}{2}r \le e \le 3v - 6
            \end{equation}
            \item 若$G$\textbf{不含任何三角形},則\begin{equation}
                e \le 2v - 4
            \end{equation}
            \item 若每個環路$\ge k \ge 3$邊組成,則\begin{equation}
                e \le \frac{k}{k - 2}(v - 2)
            \end{equation}
        \end{itemize}
        \item 雙分圖不含三角形。
        \item 一無迴圈簡單平面圖必含一個度數$\le 5$的頂點。
    \end{itemize}
\end{theorem}

\item \begin{theorem}{(6.109, 6.110, 6.113, 6.114, 6.115, 6.116)} \quad\quad
    \begin{itemize}
        \item $\chi(K_n) = n$,$\chi(W_n) = 1 + \chi(C_n)$,$\chi(C_n) = \begin{cases}
            2, & n = 2k \\ 3, & n = 2k + 1
        \end{cases}$。
        \item 若$K_n$為一圖$G$的子圖,則$\chi(G) \ge n$。
        \item 若$G = (\V, \E)$為無向圖,$\lambda$為顏色數,則稱$P(G, \lambda)$為著色多項式,表示至多使用$\lambda$種顏色著色的不同方法數,且$\chi(G) = \min\{\lambda | P(G, \lambda) > 0\}$。\begin{itemize}
            \item $P(G, \lambda)$常數項為$0$。
            \item $P(G, \lambda)$係數和為$0$。
            \item $P(G, \lambda)$最高次項係數為$1$。
        \end{itemize}
    \end{itemize}
\end{theorem}
