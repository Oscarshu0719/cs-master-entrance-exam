\item \begin{theorem}{(7.5, 7.9, 7.15, 7.16, 7.18, 7.20, 7.22, 7.24, 7.28)} \quad\quad
    \begin{itemize}
        \item 樹:$|\E| = |\V| - 1$。
        \item 森林:$|\E| = |\V| - \kappa(G)$,其中$\kappa(G)$表示樹的個數。
        \item \begin{itemize}
            \item 滿(full)二叉樹:非樹葉節點都有二個兒子。
            \item 完全(complete)二叉樹:所有樹葉節點的階層皆與樹高相同。
            \item 平衡(balanced)二叉樹:若樹高為$h$,則所有樹葉節點的階層皆為$h$或$h - 1$。
        \end{itemize}
        \item 若$T = (\V, \E), |\V| = n$為$m$-元樹,其中$i, l$分別表示內部節點與樹葉個數,則\begin{itemize}
            \item \begin{equation}
                n \le mi + 1
            \end{equation}
            \item \begin{equation}
                l \le (m - 1)i + 1
            \end{equation}
            當$T$為\textbf{滿}$m$-元樹時,等號成立。
        \end{itemize}
        \item 若$T = (\V, \E), |\V| = n$為\textbf{滿}$m$-元樹,其中$l, h$分別表示樹葉個數與樹高,則\begin{equation}
            \begin{aligned}
                (m - 1)(h - 1) + m & \le l \le m^h \\
                mh+ 1 & \le n \le \frac{m^{h + 1} - 1}{m - 1} \\
                h & \ge \ceil{log_{m}l} \quad T\text{為平衡樹時,等號成立}
            \end{aligned}
        \end{equation}
        \item 一二元樹,$n_0$表示樹葉個數,$n_2$表示二個兒子的節點個數,則\begin{equation}
            n_0 = n_2 + 1
        \end{equation}
        \item 一滿$m$-元樹,$i$為內部節點個數,$I, E$分別表示內部及外部路徑長,則\begin{equation}
            E = (m - 1)I + mi
        \end{equation}
    \end{itemize}
\end{theorem}

\item \begin{theorem}{(7.30, 7.33, 7.34, 7.35, 7.37, 7.44, 7.46, 7.49)} \quad\quad
    \begin{itemize}
        \item 若$G$為無向圖,則$G$為連通圖$\iff$$G$有生成樹。
        \item $K_n$相異生成樹個數為$n^{n - 2}$。
        \item $K_{m, n}$相異生成樹個數為$m^{n - 1}n^{m - 1}$。
        \item 若$G = (\V, \E)$為無向圖,且$e = \{a, b\} \in \E$,$N(G)$為$G$的相異生成樹個數,則\begin{equation}
            N(G) = N(G - e) + N(G \cdot e)
        \end{equation}
        \item 生成樹的任一邊稱分枝(branch),是原圖的邊但不為生成樹的邊稱弦(chord)。若$G = (\V, \E), |\V| = v, |\E| = e$,則\begin{itemize}
            \item 生成樹必含$v - 1$分支與$e - v + 1$弦。
            \item 若將任意弦加入生成樹,則新圖必含一環路,稱該環路為基本環路(fundamental cycle)。
            \item 若生成樹切除任意分支,則新圖變不連通,稱$G$中該切集為基本切集(fundamental cut set)。
        \end{itemize}
        \item 一無向連通圖,其任意切集與環路必含\textbf{偶數}個共同邊。
        \item Prim's只能找相鄰的邊,Kruskal's則無限制。
        \item Kruskal's時間複雜度為$O(m\log m)$,其中$m$為圖的\textbf{邊}數;Prim's時間複雜度為$O(n^2)$,其中$n$為圖的\textbf{點}數。
        \item 最小生成樹\textbf{未必}唯一,只有在所有邊權重皆不同時,最小生成樹唯一。
    \end{itemize}
\end{theorem}
