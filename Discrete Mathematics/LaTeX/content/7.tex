\item \begin{theorem}{(7.16, 7.24)} \quad\quad
    \begin{itemize}
        \item 若$T = (\V, \E), |\V| = n$為$m$-元樹,其中$i, l$分別表示內部節點與樹葉個數,則\begin{itemize}
            \item \begin{equation}
                n \le mi + 1
            \end{equation}
            \item \begin{equation}
                l \le (m - 1)i + 1
            \end{equation}
            當$T$為\textbf{滿}$m$-元樹時,等號成立。
        \end{itemize}
        \item 一滿$m$-元樹,$i$為內部節點個數,$I, E$分別表示內部及外部路徑長,則\begin{equation}
            E = (m - 1)I + mi
        \end{equation}
    \end{itemize}
\end{theorem}

\item \begin{theorem}{(7.33, 7.34, 7.37)} \quad\quad
    \begin{itemize}
        \item $K_n$相異生成樹個數為$n^{n - 2}$。
        \item $K_{m, n}$相異生成樹個數為$m^{n - 1}n^{m - 1}$。
        \item 若$G = (\V, \E)$為無向圖,且$e = \{a, b\} \in \E$,$N(G)$為$G$的相異生成樹個數,則\begin{equation}
            N(G) = N(G - e) + N(G \cdot e)
        \end{equation}
        \item 一無向連通圖,其任意切集與環路必含\textbf{偶數}個共同邊。
    \end{itemize}
\end{theorem}
