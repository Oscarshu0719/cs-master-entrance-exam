\item \begin{theorem}{(1.10)} (解)
    \begin{itemize}
        \item 若$\mat{A}^2 = \mat{A}$,\textbf{未必}保證$\mat{A} = \mat{I} \lor \mat{A} = \mat{O}$。
        \item 若$\mat{X}^n = \mat{A}$,\textbf{未必}保證$\mat{X}$有$n$個解。
    \end{itemize}
\end{theorem}

\item \begin{theorem}{(1.17)} ($\tr$)
    $\tr(\mat{AB}) = \tr(\mat{BA})$,可非方陣。
\end{theorem}

% \item \begin{theorem}{(1.55)} (解)
%     $\vec{x}_0$為$\mat{A}\vec{x} = \vec{b}$之一解,則$\mat{A}\vec{x} = \vec{b}$的解集合為$\{\vec{x}_0 + \vec{u} | \vec{u} \in \spc{U}\}$,
%     其中$\spc{U} = \{\vec{u} | \mat{A}\vec{u} = \vec{0}\}$,即$\spc{U}$為$\mat{A}$的null space,$\vec{x}_0$為特解、$\vec{u}$為齊次解。此式說明每一個解可表示為特解加上齊次解。
% \end{theorem}

\item \begin{theorem}{(1.67, 1.92)} (可逆,特徵,解,方陣)
    若$\mat{A}$為\textbf{方陣},則以下等價
    \begin{itemize}
        \item $\mat{A}$可逆。
        \item $\mat{A}\vec{x} = \vec{0}$只有$\vec{0}$唯一解。
        \item $\mat{A}$列等價於$\mat{I}$。
        \item $\vec{x}\mat{A} = \vec{0}$只有$\vec{0}$唯一解。
        \item $\mat{A}$行等價於$\mat{I}$。
        \item $0$不為$\mat{A}$的特徵根。
    \end{itemize}
\end{theorem}

% \item \begin{theorem}{(1.76, 1.80, 1.81)} (分解)
%     \begin{itemize}
%         \item LU分解:$\mat{A} = \mat{LU}$,其中$\mat{L}$為下三角方陣,$\mat{U}$為\textbf{pivot未必為$1$}的列梯形形式。
%         並非所有矩陣都能LU分解,需不經列交換得到列梯形形式$\mat{U}$。
%         \item LDU分解:$\mat{A} = \mat{LDU}$,其中$\mat{L}$為下三角方陣,$\mat{D}$為對角方陣,$\mat{U}$為列梯形形式。
%         \item $P^\intercal LU$分解:$\mat{A} = \mat{P^\intercal LU}$,其中$\mat{P}$為排列方陣,$\mat{L}$為下三角方陣,$\mat{U}$為\textbf{pivot未必為$1$}的列梯形形式。
%         即允許列交換的LU分解。
%     \end{itemize}
% \end{theorem}
