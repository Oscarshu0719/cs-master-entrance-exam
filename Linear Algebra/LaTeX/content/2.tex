% \item \begin{theorem}{(2.17)} (方陣,$\det$)
%     若$\mat{A}$為方陣,
%     \begin{equation}
%         \sum_{j = 1}^{n} a_{ij} \cof(a_{kj}) = 
%         \begin{cases}
%             \det(\mat{A}), & i = k \\
%             0, & i \neq k
%         \end{cases}
%     \end{equation}
%     \begin{equation}
%         \sum_{i = 1}^{n} a_{ij} \cof(a_{ik}) = 
%         \begin{cases}
%             \det(\mat{A}), & j = k \\
%             0, & j \neq k
%         \end{cases}
%     \end{equation}
% \end{theorem}

\item \begin{theorem}{(2.26)} (方陣,$\det$)
    若$\mat{A}, \mat{B}$\textbf{不}為方陣,$\det(\mat{AB}) = \det(\mat{A})\det(\mat{B})$\textbf{未必}成立。
\end{theorem}

\item \begin{theorem}{(2.43, 2.44, 2.45, 2.47)} (可逆,$\det$)
    \begin{itemize}
        \item $\mat{A} \cdot \adj{A} = \det(\mat{A}) \cdot \mat{I}$
        \item $\mat{A}\inv = \frac{1}{\det(\mat{A})}\adj{A}$
        \item $\adj{A}\inv = \frac{1}{\det(\mat{A})}\mat{A} = \adj{A\inv}$
        \item $\mat{A}$可逆$\iff$$\adj{A}$可逆。
    \end{itemize}
\end{theorem}

\item \begin{theorem}{(2.48)} (可逆,$\det$)Cramer's rule: \\
    若$\mat{A}$可逆且$\vec{x}$為$\mat{A}\vec{x} = \vec{b}$的唯一解,$\mat{A}_i(\vec{b})$為將$\mat{A}$的第$i$行改為$\vec{b}$,$\Delta = \det(\mat{A}), \Delta_i = \det(\mat{A}_i(\vec{b}))$,則
    \begin{equation}
        \vec{x}_i = \frac{\Delta_i}{\Delta}
    \end{equation}
\end{theorem}
