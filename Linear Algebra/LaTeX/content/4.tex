\item \begin{theorem}{(4.10)} 旋轉矩陣(從左到右分別為繞$x, y, z$軸旋轉的旋轉矩陣):
	\begin{equation}
		{\left[ 
		\begin{array}{ccc}
			1 & 0 & 0 \\
			0 & \cos\theta & -\sin\theta \\
			0 & \sin\theta & \cos\theta
		\end{array} 
		\right]},
		{\left[ 
		\begin{array}{ccc}
			\cos\theta & 0 & \sin\theta \\
			0 & 1 & 0 \\
			-\sin\theta & 0 & \cos\theta
		\end{array} 
		\right]},
		{\left[ 
		\begin{array}{ccc}
			\cos\theta & -\sin\theta & 0 \\
			\sin\theta & \cos\theta & 0 \\
			0 & 0 & 1 \\
		\end{array} 
		\right]}
	\end{equation}
\end{theorem}

\item \begin{theorem}{(4.30, 4.32)} 若$\T \in \L(\V, \V')$滿足
	\begin{enumerate}
		\item 線性映射。
		\item 一對一。
		\item 映成。
	\end{enumerate}
	則稱$\V$與$\V'$同構(isomorphic),記作$\V \cong \V'$。因為任意n維向量皆與$\F^{n \times 1}$同構,因此\textbf{同維即同構}。
\end{theorem}

\item \begin{theorem}{(4.73, 4.74)} (線性獨立)若$\T \in \L(\V, \V')$,則
	\begin{itemize}
		\item $\T$必保相依。
		\item $\T$保獨立,即若$\spc{S}$為線性獨立,則$\T(\spc{S})$亦是線性獨立$\iff$$\T$為一對一
		\item $\T$保生成,即若$\spc{S}$為$\V$生成集,則$\T(\spc{S})$亦是$\V$生成集$\iff$$\T$為映成
	\end{itemize}
\end{theorem}

\item \begin{theorem}{(4.67, 4.71, 4.93, 4.96, 4.97, 7.69, 7.70, 8.163)} (可逆,解)若$\T(x) = \mat{A}\vec{x}, \T \in \L(\V, \V'), \mat{A} \in \F^{m \times n}$,則
	\begin{itemize}
		\item $\mat{A}$為一對一$\iff$
		$\mat{A}$有左反矩陣$\iff$
		$\rnk(\mat{A}) = \dim(\V) = n \le \dim(\V') = m, \ \Ker(\mat{A}) = \{\vec{0}\}$$\iff$
		$n \le m$$\iff$
		$\mat{A}\vec{x} = \vec{0}$只有$\vec{0}$唯一解$\iff$
		$\mat{A}$行獨立,列生成$\F^{1 \times n}$$\iff$
		$\mat{A}\vec{x} = \vec{b}$至多一解$\iff$
		$\mat{A}\herm \mat{A}$可逆$\iff$
		$\mat{A}^+ = (\mat{A}^\intercal\mat{A})\inv\mat{A}^\intercal$
		\item $\mat{A}$為映成$\iff$
		$\mat{A}$有右反矩陣$\iff$
		$\rnk(\mat{A}) = \dim(\V') = m \le \dim(\V) = n$$\iff$
		$m \le n$$\iff$
		$\mat{A}$列獨立,行生成$\F^{m \times 1}$$\iff$
		$\mat{A}\vec{x} = \vec{b}$至少一解$\iff$
		$\mat{A}\mat{A}\herm$可逆$\iff$
		$\mat{A}^+ = \mat{A}^\intercal(\mat{A}\mat{A}^\intercal)\inv$
		\item $\T$可逆$\iff$$\T$為雙射
	\end{itemize}
\end{theorem}

\item \begin{theorem}{(4.90)} (可逆)
	\begin{itemize}
		\item $\cs(\mat{AB}) \subseteq \cs(\mat{A})$
		\item $\rs(\mat{AB}) \subseteq \rs(\mat{B})$
		\item $\Ker(\mat{B}) \subseteq \Ker(\mat{AB})$。當$\mat{A}$可逆時,$\Ker(\mat{B}) = \Ker(\mat{AB})$
		\item $\lker(\mat{A}) \subseteq \lker(\mat{AB})$
	\end{itemize}
\end{theorem}

\item \begin{theorem}{((4-3)4.3)} 
	若$\T_1$為繞x軸旋轉的旋轉矩陣,$\T_2$為繞z軸旋轉的旋轉矩陣,則$\T_1\T_2 \neq \T_2\T_1$。
\end{theorem}
