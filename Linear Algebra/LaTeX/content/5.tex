\item \begin{theorem}{(5.5, 5.35, 5.37, 6.35, 6.51)} (相似,$\tr$,$\det$,特徵,方陣)若$\mat{A}, \mat{B}$為方陣且$\mat{A} \sim \mat{B}$,則$\mat{A}$與$\mat{B}$的
	\begin{itemize}
		\item $\tr$
		\item $\det$
		\item $\rnk$
		\item $\nul$
		\item 特徵多項式
		\item 特徵根
		\item 喬丹型
	\end{itemize}
	皆\textbf{相等},反之不然。但\textbf{特徵向量}不保證相同,且僅\textbf{喬丹型}為充要條件。
\end{theorem}

\item \begin{theorem}{(5.10, 5.11, 5.34, 6.7, 6.22, 6.34)} (方陣,特徵)若$\T \in \L(\V, \V)$,$\spc{W}_i, \ \forall i = 1, \ \cdots, k$為$\V$子空間且$\spc{W}_i$為$\T$-不變子空間,$\lambda$為$\T$的特徵根,則
	\begin{itemize}
		\item $\{\vec{0}\}$
		\item $\V$
		\item $\Ker(\T^k), \ k \in \N$
		\item $\im(\T^k), \ k \in \N$
		\item $\spc{W}_1 \cap \spc{W}_2 \cap \cdots \cap \spc{W}_k$
		\item $\spc{W}_1 + \spc{W}_2 + \cdots + \spc{W}_k$
		\item 特徵空間$\V(\lambda) = \Ker(\T - \lambda\mat{I})$
		\item $\T$-循環子空間$C_{\vec{v}}(\T) = \spn\{\vec{v}, \T(\vec{v}), \T^2(\vec{v}), \ \cdots\}$
		\item 廣義特徵空間$\textit{K}(\lambda) = \bigcup\limits_{i = 1}^{\infty}\Ker((\T - \lambda\mat{I})^i)$
	\end{itemize}
	皆為$\T$-不變子空間。
\end{theorem}

\item \begin{theorem}{(5.38)} (特徵,方陣)若$\mat{A}, \mat{B}$為方陣,則$\mat{AB}$與$\mat{BA}$有相同的
	\begin{itemize}
		\item 特徵根
		\item 特徵多項式
	\end{itemize}
	若$\mat{A}, \mat{B}$\textbf{不}為方陣,只能保證$\mat{AB}$與$\mat{BA}$有相同的\textbf{非零}特徵根。 \\
	\begin{proof} \begin{equation}
			\begin{bmatrix}
				\mat{I} & \mat{B} \\
				\mat{O} & \mat{I}
			\end{bmatrix}
			\begin{bmatrix}
				\mat{O} & \mat{O} \\
				\mat{A} & \mat{AB}
			\end{bmatrix}
			\begin{bmatrix}
				\mat{I} & \mat{-B} \\
				\mat{O} & \mat{I}
			\end{bmatrix} = 
			\begin{bmatrix}
				\mat{BA} & \mat{O} \\
				\mat{A} & \mat{O}
			\end{bmatrix} = \mat{T} = \mat{P}\mat{S}\mat{P}\inv
		\end{equation},所以$\mat{S} \sim \mat{T}$,有\begin{equation}
			\begin{aligned} 
				& \det(\begin{bmatrix}
					-x\mat{I} & \mat{O} \\
					\mat{A} & \mat{AB} - x\mat{I}
				\end{bmatrix}) = 
				\det(\begin{bmatrix}
					\mat{BA} - x\mat{I} & \mat{O} \\
					\mat{A} & -x\mat{I}
				\end{bmatrix}) \\
				\Rightarrow & \det(-x\mat{I})\det(\mat{AB} - x\mat{I}) = \det(\mat{BA} - x\mat{I})\det(-x\mat{I}) \\
				\Rightarrow & \det(\mat{AB} - x\mat{I}) = \det(\mat{BA} - x\mat{I})
			\end{aligned}
		\end{equation} 則$\mat{AB}$與$\mat{BA}$有相同特徵多項式。
	\end{proof}
\end{theorem}

\item \begin{theorem}{(5.40, 8.13)} (特徵,方陣,可逆)若$\mat{A}$為方陣,則
	\begin{itemize}
		\item $\mat{A}\inv$,若$\mat{A}$可逆
		\item $\mat{A}^m, \ \forall m \in \N$
		\item $\alpha\mat{A}$
		\item $\mat{A} + \alpha\mat{I}$
		\item $f(\mat{A}), \ f(x) \in \P$
		\item $\mat{A}\herm$,若$\mat{A}$為正規矩陣,即$\mat{A}\mat{A}\herm = \mat{A}\herm\mat{A}$。
	\end{itemize}
	特徵向量\textbf{不}改變。
\end{theorem}

\item \begin{theorem}{(5.81)} (對角化,方陣)若$\T, \textit{U} \in \L(\V, \V)$皆可對角化,則\begin{equation}
		\T, \textit{U} \ \text{可同步對角化} \ \iff \T\textit{U} = \textit{U}\T	
	\end{equation}
\end{theorem}

\item \begin{theorem}{(5.100, 5.102)} (方陣,特徵,對角化,$\tr$)若$\T \in \L(\V, \V)$,且$\T^2 = \T$,稱$\T$為$\V$上的冪等(idempotent)算子,則
	\begin{itemize}
		\item $\V = \Ker(\T) \oplus \im(\T)$
		\item $\V(0) = \Ker(\T), \V(1) = \im(\T)$
	\end{itemize}
\end{theorem}

\item \begin{theorem}{(5.99, 6.13)} (方陣)若$\T \in \L(\V, \V)$,以下等價
	\begin{itemize}
		\item $\im(\T) = \im(\T^2)$
		\item $\rnk(\T) = \rnk(\T^2)$
		\item $\nul(\T) = \nul(\T^2)$
		\item $\Ker(\T) = \Ker(\T^2) \iff \V = \ker(\T) \oplus \im(\T)$
	\end{itemize}
\end{theorem}
