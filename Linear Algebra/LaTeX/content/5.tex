\item \begin{theorem}{(5.5, 5.35, 5.37, 6.35, 6.51)} (相似,$\tr$,$\det$,特徵,方陣)若$\mat{A}, \mat{B}$為方陣且$\mat{A} \sim \mat{B}$,則$\mat{A}$與$\mat{B}$的
	\begin{itemize}
		\item $\tr$
		\item $\det$
		\item $\rnk$
		\item $\nul$
		\item 特徵多項式
		\item 特徵根
		\item 喬丹型
	\end{itemize}
	皆\textbf{相等},反之不然。但\textbf{特徵向量}不保證相同,且僅\textbf{喬丹型}為充要條件。
\end{theorem}

\item \begin{theorem}{(5.6)} (相似,方陣)若$\mat{A}, \mat{B}$為方陣且$\mat{A} \sim \mat{B}$,則
	\begin{itemize}
		\item $\mat{A}^\intercal \sim \mat{B}^\intercal$
		\item $\mat{A}^k \sim \mat{B}^k, \ \forall k \in \N$
		\item $\alpha\mat{A} \sim \alpha\mat{B}, \ \forall \alpha \in \F$
		\item $\mat{A} + \alpha \mat{I} \sim \mat{B} + \alpha \mat{I}$
		\item $f(\mat{A}) \sim f(\mat{B}), \ \forall f(x) \in F[x]$
	\end{itemize}
\end{theorem}

\item \begin{theorem}{(5.10, 5.11, 5.34, 6.7, 6.22, 6.34)} (方陣,特徵)若$\T \in \L(\V, \V)$,$\spc{W}_i, \ \forall i = 1, \ \cdots, k$為$\V$子空間且$\spc{W}_i$為$\T$-不變子空間,$\lambda$為$\T$的特徵跟,則
	\begin{itemize}
		\item $\{\vec{0}\}$
		\item $\V$
		\item $\Ker(\T^k), \ k \in \N$
		\item $\im(\T^k), \ k \in \N$
		\item $\spc{W}_1 \cap \spc{W}_2 \cap \cdots \cap \spc{W}_k$
		\item $\spc{W}_1 + \spc{W}_2 + \cdots + \spc{W}_k$
		\item 特徵空間$\V(\lambda) = \Ker(\T - \lambda\mat{I})$
		\item $\T$-循環子空間$C_{\vec{v}}(\T) = \spn\{\vec{v}, \T(\vec{v}), \T^2(\vec{v}), \ \cdots\}$
		\item 廣義特徵空間$\textit{K}(\lambda) = \bigcup_{i = 1}^{\infty}\Ker((\T - \lambda\mat{I})^i)$
	\end{itemize}
	皆為$\T$-不變子空間。
\end{theorem}

% \item \begin{theorem}{(5.10, 5.11)} (方陣)若$\T \in \L(\V, \V)$,$\spc{W}_i, \ \forall i = 1, \ \cdots, k$為$\T$-不變子空間且$\V = \spc{W}_1 \oplus \spc{W}_2 \oplus \cdots \oplus \spc{W}_k$,
% 	$\beta_i$為$\spc{W}_i$基底,則$\beta = \beta_1 \cup \beta_2 \cup \cdots \cup \beta_k$為$\V$基底,且
% 	\begin{equation}
% 		\begin{pmatrix}
% 			\begin{matrix} 
% 				\mat{A}_1 &  \\
% 				& \mat{A}_2 
% 			\end{matrix} & \Large{\mat{O}} \\
% 			\Large{\mat{O}} & \begin{matrix} 
% 				\ddots &  \\ 
% 				& \mat{A}_k 
% 			\end{matrix}
% 		\end{pmatrix}
% 	\end{equation}
% 	,其中$\mat{A}_i = [\T_{\spc{W}_i}]_{\beta_i}$。
% \end{theorem}

\item \begin{theorem}{(5.23)} (特徵)特徵多項式:
	\begin{equation}
		p_{\mat{A}}(x) = (-1)^nx^n + (-1)^{n - 1}\tr(\mat{A})x^{n - 1} + \cdots + \det(\mat{A})
	\end{equation}
\end{theorem}

\item \begin{theorem}{(5.34)} (特徵,線性獨立,方陣)若$\mat{A}$為方陣,$\lambda_i, \ \forall i = 1, \ \cdots, k$為$\mat{A}$相異特徵根,則
	\begin{itemize}
		\item $\V(\lambda_1), \V(\lambda_2), \ \cdots, \V(\lambda_k)$為獨立子空間。
		\item 若$\vec{v}_i$為$\lambda_i$的特徵向量,則$\vec{v}_1, \vec{v}_2, \ \cdots, \vec{v}_k$線性獨立。
	\end{itemize}
\end{theorem}

\item \begin{theorem}{(5.38)} (特徵,方陣)若$\mat{A}, \mat{B}$為方陣,則$\mat{AB}$與$\mat{BA}$有相同的
	\begin{itemize}
		\item 特徵根
		\item 特徵多項式
	\end{itemize}
	若$\mat{A}, \mat{B}$\textbf{不}為方陣,只能保證$\mat{AB}$與$\mat{BA}$有相同的\textbf{非零}特徵根。
\end{theorem}

\item \begin{theorem}{(5.40, 8.13)} (特徵,方陣,可逆)若$\mat{A}$為方陣,則
	\begin{itemize}
		\item $\mat{A}\inv$,若$\mat{A}$可逆
		\item $\mat{A}^m, \ \forall m \in \N$
		\item $\alpha\mat{A}$
		\item $\mat{A} + \alpha\mat{I}$
		\item $f(\mat{A}), \ f(x) \in \P$
		\item 若$\mat{A}$為正規矩陣,即$\mat{A}\mat{A}\herm = \mat{A}\herm\mat{A}$。
	\end{itemize}
	特徵向量\textbf{不}改變。
\end{theorem}

\item \begin{theorem}{(5.55, 5.57)} (對角化,特徵,線性獨立,方陣)若$\mat{A} \in \F^{n \times n}$為方陣,則
	$\mat{A}$可對角化$\iff$$\mat{A}$有$n$個線性獨立特徵向量;當$< n$個線性獨立特徵向量時,稱$\mat{A}$有缺陷(defective),即$\mat{A}$不可對角化。
\end{theorem}

\item \begin{theorem}{(5.59, 5.61)} (方陣,特徵)若$\T \in \L(\V, \V)$,$\lambda$為$\T$特徵根,則
	\begin{itemize}
		\item $\am(\lambda)$為$\lambda$在特徵多項式$p_{\T}(x)$的重根數。
		\item $\gm(\lambda) = \dim(\V(\lambda)) = \nul(\T - \lambda\mat{I}) = \dim(\V) - \rnk(\T - \lambda\mat{I})$
		\item $1 \le \gm(\lambda) \le \am(\lambda) \le \dim(\V)$
	\end{itemize}
\end{theorem}

\item \begin{theorem}{(5.64)} (方陣,$\tr$,$\det$)若$\mat{A} \in \F^{n \times n}$為方陣且$\lambda_i, \ \forall i = 1, \ \cdots, n$為$\mat{A}$特徵根,則
	\begin{itemize}
		\item $\det(\mat{A}) = \Pi_{i = 1}^{n}\lambda_i$(含重根)
		\item $\tr(\mat{A}) = \sum_{i = 1}^{n}\lambda_i$(含重根)
	\end{itemize}
\end{theorem}

\item \begin{theorem}{(5.71, 5.75)} (方陣,特徵,對角化)若$\T \in \L(\V, \V)$,$\dim(\V) = n$,$\lambda_i, \ \forall i = 1, \ \cdots, r$為相異特徵根,則以下等價
	\begin{itemize}
		\item $\T$可對角化。
		\item $\gm(\lambda_i) = \am(\lambda_i)$
		\item $\V = \V(\lambda_1) \oplus \V(\lambda_2) \oplus \cdots \oplus \V(\lambda_r)$
	\end{itemize}
	若有$n$個相異特徵根,即$r = n$,則$\T$可對角化,反之不然。
\end{theorem}

\item \begin{theorem}{(5.79)} (方陣,對角化)若$\T, \textit{U} \in \L(\V, \V)$,存在$\V$基底$\beta$使得$[\T]_{\beta} = \mat{D}_1, [\textit{U}]_{\beta} = \mat{D}_2$,且$\mat{D}_1, \mat{D}_2$為對角矩陣,則
	\begin{itemize}
		\item $\T, \textit{U}$稱可同步對角化。\label{sim_diag}
		\item $\T\textit{U} = \textit{U}\T$
	\end{itemize}
	反之不然,因為$\T$或$\textit{U}$\textbf{未必}能對角化。當$\T$與$\textit{U}$皆能對角化時,Theorem \ref{sim_diag}的兩個敘述等價。
\end{theorem}

\item \begin{theorem}{(5.97, 5.100, 5.102, 5.103)} (方陣,特徵,對角化,$\tr$)若$\T \in \L(\V, \V)$,且$\T^2 = \T$,稱$\T$為$\V$上的冪等(idempotent)算子,則
	\begin{itemize}
		\item $\V = \Ker(\T) \oplus \im(\T)$
		\item $\T$可能特徵根只有$0$和$1$。
		\item $\V(0) = \Ker(\T), \V(1) = \im(\T)$
		\item $\T$可對角化,且存在$\V$基底$\beta$使得$[\T]_{\beta} = 
		{\left[ 
		\begin{array}{cc}
			\Large{\mat{I}_r} & \Large{\mat{O}} \\
			\Large{\mat{O}} & \Large{\mat{O}_{n - r}}
		\end{array} 
		\right]}$,其中$r = \rnk(\T)$。
		\item $\rnk(\T) = \tr(\T)$
	\end{itemize}
\end{theorem}

\item \begin{theorem}{(5.99)} (方陣)若$\T \in \L(\V, \V)$,以下等價
	\begin{itemize}
		\item $\im(\T) = \im(\T^2)$
		\item $\rnk(\T) = \rnk(\T^2)$
		\item $\nul(\T) = \nul(\T^2)$
		\item $\Ker(\T) = \Ker(\T^2)$
	\end{itemize}
\end{theorem}

\item \begin{theorem}{(5.100)} (方陣)Sylvester第二定理: \\
	若$\T \in \L(\V, \V)$,以下等價
	\begin{itemize}
		\item $\V = \Ker(\T) \oplus \im(\T)$
		\item $\V = \Ker(\T) + \im(\T)$
		\item $\Ker(\T) \cap \im(\T) = \{\vec{0}\}$
	\end{itemize}
\end{theorem}

\item \begin{theorem}{(5.148, 5.150, 5.151, 5.153, 5.155)} (方陣,特徵,對角化)若$\mat{P} \in \F^{n \times n}$為方陣且$\mat{P}$為Markov矩陣,則
	\begin{itemize}
		\item 若$\lambda$為$\mat{P}$的特徵根,則$|\lambda| \le 1$,且$\mat{P}$必定含$1$特徵根。
		\item 若$\mat{Q} \in \F^{n \times n}$為方陣,則$\mat{PQ}$仍為Markov矩陣。
		\item 若存在$k \in \N$使得$\mat{P}^k$中元素皆$> 0$,則稱$\mat{P}$為正則轉移矩陣,則
		\begin{itemize}
			\item 若$\lambda$為$\mat{P}$特徵根且$\lambda \neq 1$,則$|\lambda| < 1$。
			\item 若$\mat{P}$可對角化,則$\am(1) = 1$。
			\item $\lim_{k \rightarrow \infty} \mat{P}^k = 
			{\left[ 
			\begin{array}{cccc}
				\vec{x} & \vec{x} & \cdots & \vec{x}
			\end{array} 
			\right]}$
			,其中$\vec{x}$為穩定狀態向量。
		\end{itemize}
	\end{itemize}
\end{theorem}

\item \begin{theorem}{(5.153)} (方陣,特徵)若$\mat{P} \in \F^{n \times n}$為方陣且$\mat{P}$為Markov矩陣,$\vec{x}$為機率向量,有
	\begin{equation}
		\mat{P}\vec{x} = \vec{x}
	\end{equation}
	則$\vec{x}$為$\mat{P}$的一個穩定狀態向量,且$\vec{x} \in \Ker(\mat{P} - \mat{I})$,$\vec{x}$為$\mat{P}$相對於$1$的特徵向量。
\end{theorem}
