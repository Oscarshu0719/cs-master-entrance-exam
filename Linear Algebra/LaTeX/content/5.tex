\item \begin{theorem}{(5.5, 5.35, 5.37, 6.35, 6.51)} (相似,$\tr$,$\det$,特徵,方陣)若$\mat{A}, \mat{B}$為方陣且$\mat{A} \sim \mat{B}$,則$\mat{A}$與$\mat{B}$的
	\begin{itemize}
		\item $\tr$
		\item $\det$
		\item $\rnk$
		\item $\nul$
		\item 特徵多項式
		\item 特徵根
		\item 喬丹型
	\end{itemize}
	皆\textbf{相等},反之不然。但\textbf{特徵向量}不保證相同,且僅\textbf{喬丹型}為充要條件。
\end{theorem}

\item \begin{theorem}{(5.6)} (相似,方陣)若$\mat{A}, \mat{B}$為方陣且$\mat{A} \sim \mat{B}$,則
	\begin{itemize}
		\item $\mat{A}^\intercal \sim \mat{B}^\intercal$
		\item $\mat{A}^k \sim \mat{B}^k, \ \forall k \in \N$
		\item $\alpha\mat{A} \sim \alpha\mat{B}, \ \forall \alpha \in \F$
		\item $\mat{A} + \alpha \mat{I} \sim \mat{B} + \alpha \mat{I}$
		\item $f(\mat{A}) \sim f(\mat{B}), \ \forall f(x) \in F[x]$
	\end{itemize}
\end{theorem}

\item \begin{theorem}{(5.10, 5.11, 5.34, 6.7, 6.22, 6.34)} (方陣,特徵)若$\T \in \L(\V, \V)$,$\spc{W}_i, \ \forall i = 1, \ \cdots, k$為$\V$子空間且$\spc{W}_i$為$\T$-不變子空間,$\lambda$為$\T$的特徵根,則
	\begin{itemize}
		\item $\{\vec{0}\}$
		\item $\V$
		\item $\Ker(\T^k), \ k \in \N$
		\item $\im(\T^k), \ k \in \N$
		\item $\spc{W}_1 \cap \spc{W}_2 \cap \cdots \cap \spc{W}_k$
		\item $\spc{W}_1 + \spc{W}_2 + \cdots + \spc{W}_k$
		\item 特徵空間$\V(\lambda) = \Ker(\T - \lambda\mat{I})$
		\item $\T$-循環子空間$C_{\vec{v}}(\T) = \spn\{\vec{v}, \T(\vec{v}), \T^2(\vec{v}), \ \cdots\}$
		\item 廣義特徵空間$\textit{K}(\lambda) = \bigcup\limits_{i = 1}^{\infty}\Ker((\T - \lambda\mat{I})^i)$
	\end{itemize}
	皆為$\T$-不變子空間。
\end{theorem}

\item \begin{theorem}{(5.38)} (特徵,方陣)若$\mat{A}, \mat{B}$為方陣,則$\mat{AB}$與$\mat{BA}$有相同的
	\begin{itemize}
		\item 特徵根
		\item 特徵多項式
	\end{itemize}
	若$\mat{A}, \mat{B}$\textbf{不}為方陣,只能保證$\mat{AB}$與$\mat{BA}$有相同的\textbf{非零}特徵根。
\end{theorem}

\item \begin{theorem}{(5.40, 8.13)} (特徵,方陣,可逆)若$\mat{A}$為方陣,則
	\begin{itemize}
		\item $\mat{A}\inv$,若$\mat{A}$可逆
		\item $\mat{A}^m, \ \forall m \in \N$
		\item $\alpha\mat{A}$
		\item $\mat{A} + \alpha\mat{I}$
		\item $f(\mat{A}), \ f(x) \in \P$
		\item $\mat{A}\herm$,若$\mat{A}$為正規矩陣,即$\mat{A}\mat{A}\herm = \mat{A}\herm\mat{A}$。
	\end{itemize}
	特徵向量\textbf{不}改變。
\end{theorem}

\item \begin{theorem}{(5.79)} (方陣,對角化)若$\T, \textit{U} \in \L(\V, \V)$,存在$\V$基底$\beta$使得$[\T]_{\beta} = \mat{D}_1, [\textit{U}]_{\beta} = \mat{D}_2$,且$\mat{D}_1, \mat{D}_2$為對角矩陣,則
	\begin{itemize}
		\item $\T, \textit{U}$稱可同步對角化。\label{sim_diag}
		\item $\T\textit{U} = \textit{U}\T$
	\end{itemize}
	反之不然,因為$\T$或$\textit{U}$\textbf{未必}皆能對角化。當$\T$與$\textit{U}$皆能對角化時,Theorem \ref{sim_diag}的兩個敘述等價。
\end{theorem}

\item \begin{theorem}{()} (方陣,對角化)若$\mat{A}$可對角化,則$\rnk(\mat{A}) = $不為$0$的特徵根個數。
\end{theorem}

\item \begin{theorem}{(5.97, 5.100, 5.102, 5.103)} (方陣,特徵,對角化,$\tr$)若$\T \in \L(\V, \V)$,且$\T^2 = \T$,稱$\T$為$\V$上的冪等(idempotent)算子,則
	\begin{itemize}
		\item $\V = \Ker(\T) \oplus \im(\T)$
		\item $\T$可能特徵根只有$0$和$1$。
		\item $\V(0) = \Ker(\T), \V(1) = \im(\T)$
		\item $\T$可對角化,且存在$\V$基底$\beta$使得$[\T]_{\beta} = 
		{\left[ 
		\begin{array}{cc}
			\Large{\mat{I}_r} & \Large{\mat{O}} \\
			\Large{\mat{O}} & \Large{\mat{O}_{n - r}}
		\end{array} 
		\right]}$,其中$r = \rnk(\T) = \tr(\T)$。
	\end{itemize}
\end{theorem}

\item \begin{theorem}{(5.99)} (方陣)若$\T \in \L(\V, \V)$,以下等價
	\begin{itemize}
		\item $\im(\T) = \im(\T^2)$
		\item $\rnk(\T) = \rnk(\T^2)$
		\item $\nul(\T) = \nul(\T^2)$
		\item $\Ker(\T) = \Ker(\T^2)$
	\end{itemize}
\end{theorem}

\item \begin{theorem}{(5.100)} (方陣)Sylvester第二定理: \\
	若$\T \in \L(\V, \V)$,以下等價
	\begin{itemize}
		\item $\V = \Ker(\T) \oplus \im(\T)$
		\item $\V = \Ker(\T) + \im(\T)$
		\item $\Ker(\T) \cap \im(\T) = \{\vec{0}\}$
	\end{itemize}
\end{theorem}

\item \begin{theorem}{(5.148, 5.150, 5.151, 5.153, 5.155)} (方陣,特徵,對角化)若$\mat{P} \in \F^{n \times n}$為方陣且$\mat{P}$為正則轉移矩陣,則
	\begin{itemize}
		\item 若$\mat{P}$可對角化,則$\am(1) = 1$。
		\item $\lim_{k \rightarrow \infty} \mat{P}^k = 
		{\left[ 
		\begin{array}{cccc}
			\vec{x} & \vec{x} & \cdots & \vec{x}
		\end{array} 
		\right]}$
		,其中$\vec{x}$為穩定狀態向量,即$\vec{x} \in \Ker(\mat{P} - \mat{I})$。
	\end{itemize}
\end{theorem}
