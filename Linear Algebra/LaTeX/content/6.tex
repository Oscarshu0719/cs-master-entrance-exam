\item \begin{theorem}{(6.3, 6.15, 6.17, 6.71)} (方陣,$\det$,特徵,線性獨立)若$\T \in \L(\V, \V)$,存在$k \in N$使得$\T^{k} = \mat{O}$,則稱$\T$為冪零(nilpotent)算子,最小正整數$k$為$\T$的指標,則
	\begin{itemize}
		\item $\det(\T) = 0$
		\item $\T^n = \mat{O}$
		\item $\T$為冪零算子$\iff$$\T$特徵根\textbf{皆}$0$。
		\item $\exists \vec{v} \in \V \lor \vec{v} \in \Ker(\T^k) - \Ker(\T^{k - 1})$且$\vec{v} \neq \vec{0}$,
		$\{\vec{v}, \T(\vec{v}), \ \cdots, \T^{k - 1}(\vec{v})\}$線性獨立。
	\end{itemize}
\end{theorem}

\item \begin{theorem}{(6.9, 6.11)} (方陣)若$\T \in \L(\V, \V)$,則
	\begin{itemize}
		\item 核集鏈定理:
			\begin{itemize}
				\item $\{\vec{0}\} \subseteq \Ker(\T) \subseteq \Ker(\T^2) \subseteq \cdots \subseteq \V$
				\item $\exists \min k \in \N, \ \Ker(\T^{k}) = \Ker(\T^{k + 1}) = \cdots$
				\item $\spc{W} = \bigcup_{i = 1}^{\infty} \Ker(\T^i) = \Ker(\T^k)$為最大冪零區。
			\end{itemize}
		\item 像集鏈定理:
			\begin{itemize}
				\item $\V \supseteq \im(\T) \supseteq \im(\T^2) \supseteq \cdots \supseteq \{\vec{0}\}$
				\item $\exists \min k \in \N, \ \im(\T^{k}) = \im(\T^{k + 1}) = \cdots$
				\item $\spc{W} = \bigcap_{i = 1}^{\infty} \im(\T^i) = \im(\T^k)$為最大可逆區。
			\end{itemize}
	\end{itemize}
\end{theorem}

\item \begin{theorem}{(6.12)} (方陣)Fitting lemma: \\
	若$\T \in \L(\V, \V)$,則存在$k \in N$使得$\V = \Ker(\T^k) \oplus \im(\T^k)$。
\end{theorem}

\item \begin{theorem}{(6.20, 6.22, 6.23, 6.24)} (方陣)若$\T \in \L(\V, \V), \vec{v} \in \V$,稱\\$C_{\vec{v}}(\T) = \spn\{\vec{v}, \T(\vec{v}), \T^2(\vec{v}), \ \cdots\}$為由$\vec{v}$生成的$\T$-循環子空間,
	且$C_{\vec{v}}(\T)$為$\T$-不變子空間,則:
	\begin{itemize}
		\item 若$\vec{v} \in \Ker(\T^k) - \Ker(\T^{k - 1})$,則稱$\beta = \{\vec{v}, \T(\vec{v}), \ \cdots, \T^{k - 1}(\vec{v})\}$為循環基底,且$\beta$為$\spc{W} = C_{\vec{v}}(\T)$的基底有
		$[\T_{\spc{W}}]_{\beta} = \mat{S}_k$為下移矩陣,$\dim(C_{\vec{v}}(\T)) = k$,$\T^k(\vec{v}) = 0$。
		\begin{itemize}
			\item 當$\beta = \{\T^{k - 1}(\vec{v}), \T^{k - 2}(\vec{v}), \ \cdots, \T(\vec{v}), \vec{v}\}$時,$[\T_{\spc{W}}]_{\beta} = \mat{S}^\intercal_k$為上移矩陣。
			\item 反之,$\dim(C_{\vec{v}}(\T)) = k$\textbf{不能}保證$\vec{v} \in \Ker(\T^k) - \Ker(\T^{k - 1})$。
			\item 反之,$\dim(C_{\vec{v}}(\T)) = k$保證$\beta = \{\vec{v}, \T(\vec{v}), \ \cdots, \T^{k - 1}(\vec{v})\}$為$C_{\vec{v}}(\T)$的基底。
			\item 反之,$\dim(C_{\vec{v}}(\T)) = k$\textbf{不能}保證$\T^k(\vec{v}) = 0$。\\
				令$\T^k(\vec{v}) = -a_0\vec{v} + \sum_{i = 1}^{k - 1}(-a_{i})\T^{i}(\vec{v})$,有
				\begin{equation}
					\mat{A} = [\T_{\spc{W}}]_{\beta} = 
					{\left[ 
					\begin{array}{ccccc}
						0 & 0 & \cdots & 0 & -a_0 \\
						1 & 0 & \cdots & 0 & -a_1 \\
						0 & 1 & \cdots & 0 & -a_2 \\
						\vdots & \vdots & \ddots & \vdots & \vdots \\
						0 & 0 & \cdots & 1 & -a_{k - 1} \\
					\end{array} 
					\right]}
				\end{equation}
				有$p_{\mat{A}}(x) = (-1)^k(a_0 + a_1x + \cdots + a_{k - 1}x^{k - 1} + x^k)$,稱$\mat{A}$為$(a_0 + a_1x + \cdots + a_{k - 1}x^{k - 1} + x^k)$的友(companion)矩陣。
				當$\T^k(\vec{0}) = \vec{0}$時,$a_0 = a_1 = \cdots = a_{k - 1} = 0, \mat{A} = [\T_{\spc{W}}]_{\beta} = \spc{S}_k$。
		\end{itemize}
	\end{itemize}
\end{theorem}

\item \begin{theorem}{(6.24)} (方陣)循環分解定理(Cyclic decomposition theorem): \\
	若$\T \in \L(\V, \V)$為指標為$k$的冪零算子,則存在$\vec{v}_1, \vec{v}_2, \ \cdots, \vec{v}_r \in \V$使得
	\begin{equation}
		\V = C_{\vec{v}_1}(\T) \oplus C_{\vec{v}_2}(\T) \oplus \cdots \oplus C_{\vec{v}_r}(\T)
	\end{equation}
	其中$\vec{v}_i \in \Ker(\T^{n_i}) - \Ker(\T^{n_i - 1})$,則
	\begin{itemize}
		\item $\dim(C_{\vec{v}_i}(\T)) = n_i$,$n_1 \ge n_2 \ge \cdots \ge n_r$,稱$\{n_1, n_2, \ \cdots, n_r\}$為不變集(invariant set)。
		\item $r = \dim(\Ker(\T))$
		\item $k = n_1$
	\end{itemize}
\end{theorem}

\item \begin{theorem}{(6.33, 6.34, 6.35, 6.36,  6.38)} (方陣,特徵)若$\T \in \L(\V, \V), \dim(\V) = n$,$\lambda$為$\T$特徵根,
	稱
	\begin{equation}
		\textit{K}(\lambda) = \bigcup_{i = 1}^{\infty}\Ker((\T - \lambda\mat{I})^i)
	\end{equation}
	為$\T$相對於$\lambda$的廣義特徵空間,且$\dim(\textit{K}(\lambda)) = m$,則
	\begin{itemize}
		\item $\textit{K}(\lambda) = \{\vec{v} \in \V | (\T - \lambda\mat{I})^p(\vec{v}) = 0, \exists p \in \N\} = \Ker((\T - \lambda\mat{I})^n) \supseteq \V(\lambda)$
		\item $\textit{K}(\lambda) \subseteq \V$,為$\T$-不變子空間。
		\item $\lambda$為$p_{\T}(x) = 0$的$m$重根,則$\am(\lambda) = m$。
		\item $(\T - \lambda\mat{I})_{\textit{K}(\lambda)}$為冪零算子。
		\item 存在$\vec{v}_1, \vec{v}_2, \ \cdots, \vec{v}_k \in \textit{K}(\lambda)$使得
		\begin{equation}
			\textit{K}(\lambda) = C_{\vec{v}_1}(\T - \lambda\mat{I}) \oplus C_{\vec{v}_2}(\T - \lambda\mat{I}) \oplus \cdots \oplus C_{\vec{v}_k}(\T - \lambda\mat{I})
		\end{equation}
		\item 若$\lambda_1, \lambda_2, \ \cdots, \lambda_r$為$\T$相異特徵根,且$p_{\T}(x)$在$\F$可分解,則
		\begin{equation}
			\V = \textit{K}(\lambda_1) \oplus \textit{K}(\lambda_2) \oplus \cdots \oplus \textit{K}(\lambda_r)
		\end{equation}
	\end{itemize}
\end{theorem}

\item \begin{theorem}{(6.68)} (方陣,特徵)Cayley-Hamilton theorem: \\
	若$\mat{A} \in \F^{n \times n}$為方陣,$f(x) = p_{\mat{A}}(x) = \det(\mat{A} - x\mat{I})$,則$f(\mat{A}) = \mat{O}$。
\end{theorem}

\item \begin{theorem}{(6.69, 6.115, 6.116)} (方陣,特徵,對角化)若$\T \in \L(\V, \V)$,$\spc{W}$為$\T$-不變子空間,則
	\begin{itemize}
		\item $\T_\spc{W}$的特徵多項式整除$\T$的特徵多項式。
		\item $\T_\spc{W}$的極小多項式整除$\T$的極小多項式。
		\item 若$\T$可對角化,則$\T_{\spc{W}}$也可對角化。
	\end{itemize}
\end{theorem}

\item \begin{theorem}{(6.99, 6.100, 6.111)} (方陣,特徵,對角化)若$\mat{A} \in \F^{n \times n}$為方陣,$m(x) \in \P$滿足
	\begin{enumerate}
		\item $m(x)$為首一(monic)多項式,即$\deg(m(x)) \ge 1$,且最高項係數為$1$。
		\item $m(\mat{A}) = \mat{O}$
		\item $f(x) \in \P, \ f(\mat{A}) = \mat{O}$,則$\deg(f(x)) \le \deg(m(x))$。
	\end{enumerate}
	稱$m(x)$為極小(minimal)多項式,則
	\begin{itemize}
		\item $f(x) \in \P, \ f(\mat{A}) = \mat{O}$,則$m(x) | f(x)$。
		\item $m(x) | p_{\mat{A}}(x)$
		\item $m(x)$唯一。
		\item 若$\lambda_1, \lambda_2, \ \cdots, \lambda_r$為$\T$相異特徵根,則
		\begin{itemize}
			\item $(\T - \lambda\mat{I})_{\textit{K}(\lambda_i)}$有指標$k_i$,即點圖的第一列點數,則
			$m(x) = (x - \lambda_1)^{k_1}(x - \lambda_2)^{k_2}\cdots(x - \lambda_r)^{k_r}$
			\item $\T$可對角化$\iff$$m(x) = (x - \lambda_1)(x - \lambda_2)\cdots(x - \lambda_r)$
		\end{itemize}
	\end{itemize}
\end{theorem}
