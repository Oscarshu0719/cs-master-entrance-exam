\item \begin{theorem}{(6.17)} (方陣,$\det$,特徵,線性獨立)若$\T \in \L(\V, \V)$為冪零算子,且最小正整數$k$為$\T$的指標,則
	$\exists \vec{v} \in \V \lor \vec{v} \in \Ker(\T^k) - \Ker(\T^{k - 1})$且$\vec{v} \neq \vec{0}$,
	$\{\vec{v}, \T(\vec{v}), \ \cdots, \T^{k - 1}(\vec{v})\}$線性獨立。
\end{theorem}

\item \begin{theorem}{(6.9, 6.11)} (方陣)若$\T \in \L(\V, \V)$,則
	\begin{itemize}
		\item $\{\vec{0}\} \subseteq \Ker(\T) \subseteq \Ker(\T^2) \subseteq \cdots \subseteq \V$
		\item $\spc{W} = \bigcup\limits_{i = 1}^{\infty} \Ker(\T^i) = \Ker(\T^k)$為最大冪零區。
		\item $\V \supseteq \im(\T) \supseteq \im(\T^2) \supseteq \cdots \supseteq \{\vec{0}\}$
		\item $\spc{W} = \bigcap\limits_{i = 1}^{\infty} \im(\T^i) = \im(\T^k)$為最大可逆區。
	\end{itemize}
\end{theorem}

\item \begin{theorem}{(6.12)} (方陣)若$\T \in \L(\V, \V)$,則$\exists \ k \in N$使得$\V = \Ker(\T^k) \oplus \im(\T^k)$。
\end{theorem}

\item \begin{theorem}{(6.16)} (特徵)冪零矩陣特徵根全都是$0$。
\end{theorem}

\item \begin{theorem}{(6.22, 6.23, 6.24)} (方陣)若$\T \in \L(\V, \V), \vec{v} \in \V$,則
	\begin{itemize}
		\item $\dim(C_{\vec{v}}(\T)) = k$\textbf{不能}保證$\vec{v} \in \Ker(\T^k) - \Ker(\T^{k - 1})$。
		\item $\dim(C_{\vec{v}}(\T)) = k$保證$\beta = \{\vec{v}, \T(\vec{v}), \ \cdots, \T^{k - 1}(\vec{v})\}$為$C_{\vec{v}}(\T)$的基底。
		\item $\dim(C_{\vec{v}}(\T)) = k$\textbf{不能}保證$\T^k(\vec{v}) = 0$。
	\end{itemize}
\end{theorem}

\item \begin{theorem}{(6.69, 6.115, 6.116)} (方陣,特徵,對角化)若$\T \in \L(\V, \V)$,$\spc{W}$為$\T$-不變子空間,則
	\begin{itemize}
		\item $\T_\spc{W}$的特徵多項式整除$\T$的特徵多項式。
		\item $\T_\spc{W}$的極小多項式整除$\T$的極小多項式。
		\item 若$\T$可對角化,則$\T_{\spc{W}}$也可對角化。
	\end{itemize}
\end{theorem}

\item \begin{theorem}{(6.111)} (方陣,特徵)若$\T \in \L(\V, \V)$,且$\lambda_1, \ \cdots, \ \lambda_r$為相異特徵根,則\begin{equation}
		\T \ \text{可對角化} \ \iff m_{\T}(x) = (x - \lambda_1)\cdots(x - \lambda_r)
	\end{equation} 
\end{theorem}

\item \begin{theorem}{((6-55)6.88)} (方陣,特徵)若$\mat{A}, \mat{B} \in \R^{n \times n}$為實方陣,當$\mat{AB} = \mat{BA}$時,\begin{equation}
		e^{A}e^{B} = e^{A + B}
	\end{equation} 可通過泰勒展開式證明。
\end{theorem}
