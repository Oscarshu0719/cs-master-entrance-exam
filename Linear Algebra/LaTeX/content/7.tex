\item \begin{theorem}{(7.3, 7.6)} (方陣,$\tr$)若$\vec{u}, \vec{v} \in \V$,$\mat{A}, \mat{B} \in \F^{n \times n}$為方陣,則
	\begin{itemize}
		\item $\dotp{\vec{u}}{\vec{v}} = \overline{\dotp{\vec{v}}{\vec{u}}}$
		\item $\dotp{\vec{u}}{\alpha \vec{v}} = \overline{\alpha}\dotp{\vec{u}}{\vec{v}}$
		\item $\dotp{\mat{A}}{\mat{B}} = \tr(\mat{AB}\herm) = \sum_{i = 1}^{n}\sum_{j = 1}^{n}a_{ij}\overline{b_{ij}}$
	\end{itemize}
\end{theorem}

\item \begin{theorem}{(7.34)} (正交,線性獨立)若$\spc{S} = \{\vec{1}, \vec{2}, \ \cdots, \vec{k}\} \subseteq \V$為不含$\vec{0}$的正交集,
	則$\spc{S}$線性獨立,反之不然。
\end{theorem}

% \item \begin{theorem}{(7.44)} (分解,正交)QR分解: \\
% 	若$\mat{A} = \mat{QR} \in \F^{m \times n}$,其中$\cs(\mat{Q})$為單範正交集,$\mat{R}$為上三角矩陣。
% \end{theorem}

\item \begin{theorem}{(7.64, 7.65, 7.66)} (正交)若$\spc{W}$為$\V$子空間,$\beta = \{\vec{v}_1, \vec{v}_2, \ \cdots, \vec{v}_k\}$為$\spc{W}$的\textbf{正交}基底,$\vec{v} \in \V$,則
	稱
	\begin{equation}
		\P(\vec{v}) = \proj{\spc{W}}{v} = \sum_{i = 1}^{k}\frac{\dotp{\vec{v}}{\vec{v}_i}}{\dotp{\vec{v}_i}{\vec{v}_i}}\vec{v}_i
	\end{equation}
	為$\V$在$\spc{W}$上的正交投影向量,則
	\begin{itemize}
		\item $\P$為線性映射。
		\item $\forall \vec{v} \in \spc{W}, \ \P(\vec{v}) = \vec{v}$
		\item $\im(\P) = \spc{W}$
		\item $\P^2 = \P$
		\item 
		\begin{equation}
			||\P(\vec{v})||^2 = \sum_{i = 1}^{k}\frac{|\dotp{\vec{v}}{\vec{v}_i}|^2}{||\vec{v_i}||^2}, \ \forall \vec{v} \in \V
		\end{equation}
		\item Bessel's Inequality:
		\begin{equation}
			||\P(\vec{v})|| \le ||\vec{v}||, \ \forall \vec{v} \in \V
		\end{equation}
		\item Parseval's equality:
		\begin{equation}
			||\P(\vec{v})|| = ||\vec{v}||, \ \forall \vec{v} \in \spc{W}
		\end{equation}
		\item 
		\begin{equation}
			||\vec{v} - \P(\vec{v})|| \le  ||\vec{v} - \vec{w}||, \ \forall \vec{w} \in \spc{W}, \vec{v} \in \V
		\end{equation}
	\end{itemize}
\end{theorem}

\item \begin{theorem}{(7.68)} (正交)
	\begin{equation}
		\{\frac{1}{\sqrt{2}}, \cos x, \cos 2x, \ \cdots, \cos nx, \sin x, \sin 2x, \ \cdots, \sin nx\}
	\end{equation}
	為單範正交集。
\end{theorem}

\item \begin{theorem}{(7.68, 7.69)} (實數)
	\begin{itemize}
		\item $\Ker(\mat{A}\herm \mat{A}) = \Ker(\mat{A})$
		\item $\rnk(\mat{A}\herm \mat{A}) = \rnk(\mat{A})$
		\item $\lker(\mat{A} \mat{A}\herm) = \lker(\mat{A})$
		\item 若$\mat{A} \in \R^{m \times n}$,$\rnk(\mat{A}^\intercal \mat{A}) = \rnk(\mat{A}\mat{A}^\intercal)$
	\end{itemize}
\end{theorem}

\item \begin{theorem}{(7.73, 7.79)} (正交)若$\mat{A} \in \F^{m \times n}$,$\spc{W} = \cs(\mat{A}), \vec{b} \in \F^{m \times 1}$,則
	\begin{itemize}
		\item $\vec{x} \in \F^{n \times 1}$使得$||\mat{A}\vec{x} - \vec{b}||$最小$\iff$$\mat{A}\herm\mat{A}\vec{x} = \mat{A}\herm\vec{b}$
		稱正規方程式(normal equation),正規方程式必有解,則
		\begin{itemize}
			\item 若$\mat{A}$\textbf{不}為行獨立,正規方程式有\textbf{無限多解},但\textbf{$\proj{\spc{W}}{b}$必定唯一}。
			\item 若$\mat{A}$為行獨立,$\proj{\spc{W}}{b} = \mat{A}\vec{x} = \mat{A}(\mat{A}\herm\mat{A})\inv\mat{A}\herm\vec{b}$。
		\end{itemize}
		% \item 若$\rnk(\mat{A}) = n$,且$\mat{A} = \mat{QR}$為$\mat{A}$的QR分解,則
		% $\vec{x} \in \F^{n \times 1}$使得$||\mat{A}\vec{x} - \vec{b}||$最小$\iff$$\mat{R}\vec{x} = \mat{Q}\herm \vec{b}$
	\end{itemize}
\end{theorem}

\item \begin{theorem}{(7.80, 7.81)} (正交)若$\mat{A} \in \F^{m \times n}$\textbf{行獨立},$\P = \mat{A}(\mat{A}\herm\mat{A})\inv\mat{A}\herm$為正交投影矩陣,則
	\begin{itemize}
		\item $\P^2 = \P$且$\P\herm = \P$
		\item $\cs(\P) = \cs(\mat{A})$
		\item $\rnk(\P) = \rnk(\mat{A}) = n$
	\end{itemize}
	\item 反之,若$\P^2 = \P$且$\P\herm = \P$,則$\P$為投影在$\cs(\P)$上的正交投影矩陣。
\end{theorem}

\item \begin{theorem}{(7.81)} (正交)若$\mat{Q}$的行向量為\textbf{單範正交集},$\spc{W} = \cs(\mat{Q})$,則
	\begin{equation}
		\proj{\spc{W}}{b} = \mat{Q}\mat{Q}\herm\vec{b}
	\end{equation}
	即投影在$\mat{Q}$的正交投影矩陣為$\mat{Q}\mat{Q}\herm$。
\end{theorem}

\item \begin{theorem}{(7.98)} (正交)若$\spc{S} \subseteq \V$,稱
	\begin{equation}
		\spc{S}^{\perp} = \{\vec{v} \in \V | \dotp{\vec{v}}{\vec{s}} = 0, \ \forall \vec{s} \in \spc{S}\}
	\end{equation}
	為$\spc{S}$的正交補空間,則
	\begin{itemize}
		\item $\{\vec{0}\}^{\perp} = \V$且$\V^{\perp} = \{\vec{0}\}$
		\item $\spc{S}^{\perp} \subseteq \V$
		\item $\spc{S} \subseteq \spc{S}^{\perp\perp}$
		\item $\spc{S} \cap \spc{S}^{\perp} = 
		\begin{cases}
			\{\vec{0}\}, & \text{if} \ \{\vec{0}\} \in \spc{S} \\
			\emptyset, & \text{if} \ \{\vec{0}\} \not\in \spc{S}
		\end{cases}$
	\end{itemize}
\end{theorem}

\item \begin{theorem}{(7.100, 7.103, 7.104)} (正交)若$\spc{W} \subseteq \V$,$\P$為$\V$在$\spc{W}$上的投影算子,則
	\begin{itemize}
		\item $\Ker(\P) = \spc{W}^{\perp}$
		\item $\V = \spc{W} \oplus \spc{W}^{\perp}$
		\item $\dim(\V) = \dim(\spc{W}) + \dim(\spc{W}^{\perp})$
		\item $\spc{W} = \spc{W}^{\perp\perp}$
		\item $\vec{v} - \P(\vec{v})$為$\vec{v}$在$\spc{W}^{\perp}$上的正交投影向量,即$\proj{\spc{W}^{\perp}}{v} = \vec{v} - \proj{\spc{W}}{v}$。
		\item $\mat{I} - \P$為$\V$在$\spc{W}^{\perp}$上的正交投影算子。
		\item $\vec{v}$到$\spc{W}$的距離$ = ||\vec{v} - \proj{\spc{W}}{v}|| = ||\proj{\spc{W}^{\perp}}{v}||$。
	\end{itemize}
\end{theorem}

\item \begin{theorem}{(7.108)} (正交,實數)$\mat{A} \in \R^{m \times n}$,則
	\begin{itemize}
		\item $\cs(\mat{A}^{\intercal})^{\perp} = \Ker(\mat{A})$
		\item $\cs(\mat{A})^{\perp} = \Ker(\mat{A}^{\intercal})$
		\item $\Ker(\mat{A})^{\perp} = \cs(\mat{A}^{\intercal})$
		\item $\Ker(\mat{A}^{\intercal})^{\perp} = \cs(\mat{A})$
	\end{itemize}
\end{theorem}

\item \begin{theorem}{(7.110, 7.111)} (正交,實數)若$\mat{A} \in \R^{m \times n}, \vec{v} \in \R^{m}, \vec{u} \in \R^{n}$,則
	\begin{itemize}
		\item 若$\rnk(\mat{A}) = n$,則
			\begin{itemize}
				\item $\mat{A}(\mat{A}^{\intercal}\mat{A})\inv\mat{A}^{\intercal}$為$\R^m$投影在$\cs(\mat{A})$的正交投影矩陣。
				\item $\mat{I} - \mat{A}(\mat{A}^{\intercal}\mat{A})\inv\mat{A}^{\intercal}$為$\R^m$投影在$\Ker(\mat{A}^\intercal)$的正交投影矩陣。
			\end{itemize}
		\item 若$\rnk(\mat{A}) = m$,則
			\begin{itemize}
				\item $\mat{A}^\intercal(\mat{A}\mat{A}^{\intercal})\inv\mat{A}$為$\R^n$投影在$\cs(\mat{A}^\intercal)$的正交投影矩陣。
				\item $\mat{I} - \mat{A}^\intercal(\mat{A}\mat{A}^{\intercal})\inv\mat{A}$為$\R^n$投影在$\Ker(\mat{A})$的正交投影矩陣。
			\end{itemize}
	\end{itemize}
\end{theorem}

\item \begin{theorem}{(7.112)} (解,實數)
	\begin{itemize}
		\item 若$\mat{A} \in \R^{m \times n}, \vec{b} \in \R^{m \times 1}$,則$\mat{A}\vec{x} = \vec{b}$有解$\iff$$\forall \vec{y}, \mat{A}^\intercal\vec{y} = \vec{0}$,則$\vec{b}^\intercal\vec{y} = \vec{0}$。
		\item 若$\mat{A}\vec{x} = \vec{b}$有解,則
			\begin{itemize}
				\item \textbf{唯一}$\exists \vec{s} \in \cs(\mat{A}\herm)$為$\mat{A}\vec{x} = \vec{b}$之極小解,即$||\vec{x}||_2$為所有解中最小。
				\item 若$\vec{u}$滿足$(\mat{AA}\herm)\vec{u} = \vec{b}$,則$\vec{s} = \mat{A}\herm\vec{u}$。
			\end{itemize}
	\end{itemize}
\end{theorem}
