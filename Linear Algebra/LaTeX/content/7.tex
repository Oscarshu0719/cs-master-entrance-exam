\item \begin{theorem}{(7.68, 7.69, 7.70, (7-97)7.174)} (實數,可逆)
	\begin{itemize}
		\item $\Ker(\mat{A}\herm \mat{A}) = \Ker(\mat{A})$
		\item $\rnk(\mat{A}\herm \mat{A}) = \rnk(\mat{A})$
		\item $\lker(\mat{A} \mat{A}\herm) = \lker(\mat{A})$
		\item 若$\mat{A} \in \R^{m \times n}$,$\rnk(\mat{A}^\intercal \mat{A}) = \rnk(\mat{A}\mat{A}^\intercal)$
		\item $\cs(\mat{A}^\intercal\mat{A}) = \cs(\mat{A}^\intercal)$
		\item $\mat{A}$行獨立$\iff$$\mat{A}\herm\mat{A}$可逆
		\item $\mat{A}$列獨立$\iff$$\mat{A}\mat{A}\herm$可逆
	\end{itemize}
\end{theorem}

\item \begin{theorem}{(7.79, 7.81)} (正交)若$\mat{A} \in \F^{m \times n}$,$\spc{W} = \cs(\mat{A}), \vec{b} \in \F^{m \times 1}$,則
	\begin{itemize}
		\item 若$\mat{A}$行獨立,且$\mat{A} = \mat{QR}$為$\mat{A}$的QR分解,則
		$\vec{x} \in \F^{n \times 1}$使得$||\mat{A}\vec{x} - \vec{b}||$最小$\iff$$\mat{R}\vec{x} = \mat{Q}\herm \vec{b}$。
		\item 若$\mat{Q}$的行向量為\textbf{單範正交集},$\spc{W} = \cs(\mat{Q})$,則
		\begin{equation}
			\proj{\spc{W}}{b} = \mat{Q}\mat{Q}\herm\vec{b}
		\end{equation}
	\end{itemize}
\end{theorem}

\item \begin{theorem}{(7.80, 7.81)} (正交,方陣)若$\mat{A} \in \F^{n \times n}$為方陣,且$\mat{A}$\textbf{行獨立},則
	\begin{itemize}
		\item $\P^2 = \P$為冪等方陣且$\P\herm = \P$$\iff$$\P$為正交投影矩陣
		\item $\cs(\P) = \cs(\mat{A})$
		\item $\rnk(\P) = \rnk(\mat{A}) = n$
	\end{itemize}
\end{theorem}

\item \begin{theorem}{(7.100, 7.103, 7.104)} (正交)若$\spc{W} \subseteq \V$,$\P$為$\V$在$\spc{W}$上的投影算子,則$\Ker(\P) = \spc{W}^{\perp}$。
\end{theorem}

\item \begin{theorem}{(7.112)} (解,實數)若$\mat{A}\vec{x} = \vec{b}$有解,則
	\begin{itemize}
		\item \textbf{唯一}$\exists \vec{s} \in \cs(\mat{A}\herm)$為$\mat{A}\vec{x} = \vec{b}$之極小解,即$||\vec{x}||_2$為所有解中最小。
		\item 若$\vec{u}$滿足$(\mat{AA}\herm)\vec{u} = \vec{b}$,則$\vec{s} = \mat{A}\herm\vec{u}$。
	\end{itemize}
\end{theorem}

\item \begin{theorem}{(.181)} 證明Cauchy-Schwarz inequality:\begin{equation}
		|<\vec{u}, \vec{v}>| \le ||\vec{u}|| \times ||\vec{v}||
	\end{equation} \begin{proof}
		用數學歸納法證明: \\
		若$\vec{v} = \vec{0}$,成立。 \\
		若$\vec{v} \neq \vec{0}$,取\begin{equation}
			\alpha = \frac{<\vec{u}, \vec{v}>}{<\vec{v}, \vec{v}>}
		\end{equation} 則 \begin{equation}
			\begin{aligned}
				0 & \le ||\vec{u} - \alpha\vec{v}||^2 \\
				& = <\vec{u}, \vec{u}> - \overline{\alpha}<\vec{u}, \vec{v}> - \alpha<\vec{v}, \vec{u}> + \alpha\overline{\alpha}<\vec{v}, \vec{v}> \\
				& = <\vec{u}, \vec{u}> - \overline{\frac{<\vec{u}, \vec{v}>}{<\vec{v}, \vec{v}>}}<\vec{u}, \vec{v}> - \frac{<\vec{u}, \vec{v}>}{<\vec{v}, \vec{v}>}<\vec{v}, \vec{u}> \\
				& + \frac{<\vec{u}, \vec{v}>}{<\vec{v}, \vec{v}>}\overline{\frac{<\vec{u}, \vec{v}>}{<\vec{v}, \vec{v}>}}<\vec{v}, \vec{v}> \\
				& = ||\vec{u}||^2 - \frac{|<\vec{u}, \vec{v}>|^2}{||\vec{v}||^2} \\
				\Rightarrow & \frac{|<\vec{u}, \vec{v}>|^2}{||\vec{v}||^2} \le ||\vec{u}||^2 \\
				\Rightarrow & |<\vec{u}, \vec{v}>|^2 \le ||\vec{u}||^2 \times ||\vec{v}||^2 \\
			\end{aligned}
		\end{equation}
	\end{proof}
\end{theorem}