\item \begin{theorem}{(8.14, 8.15, 8.17, 8.18, 8.20, 8.21, 8.22, 8.23, 8.54, 8.55, 8.56)} (方陣,正交,特徵,$\tr$,$\det$,可逆,實數,複數)若$\T \in \L(\V, \V)$,則
	\begin{table}[H]
		\renewcommand{\arraystretch}{2}
		\begin{tabular}{|c|c|c|c|}
			\hline
			定義 & $\lambda$ & $a_{ii}$ & $\det$ \\
			% \hline
			\Xhline{3\arrayrulewidth}
			self-adjoint $\T^* = \T$ & \multirow{3}{*}{$\in \R$} 
			& \multirow{3}{*}{$\in \R$} & \multirow{3}{*}{$\in \R$} \\
			Hermitian (over $\C$) $\mat{A}\herm = \mat{A}$ & & & \\
			\cline{1-1}
			symmetric (over $\R$) $\mat{A}^\intercal = \mat{A}$ & & & \\

			% \hline
			\Xhline{3\arrayrulewidth}
			skew self-adjoint $\T^* = -\T$ & \multirow{2}{*}{$0$或純虛數} 
			& \multirow{2}{*}{$0$或純虛數} & \multirow{2}{*}{$\begin{cases}
				\in \R, & \text{if} \ n \in 2k \\
				0\text{或純虛數}, & \text{if} \ n \in 2k + 1 \\
			\end{cases}$} \\
			skew Hermitian (over $\C$) $\mat{A}\herm = -\mat{A}$ & & & \\
			\hline
			skew symmetric (over $\R$) $\mat{A}^\intercal = -\mat{A}$ & $0$
			& $0$ & $\begin{cases}
				\in \R, & \text{if} \ n \in 2k \\
				0, & \text{if} \ n \in 2k + 1 \\
			\end{cases}$ \\

			% \hline
			\Xhline{3\arrayrulewidth}
			positive definite $\dotp{\T(\vec{x})}{\vec{x}} > 0, \ \forall \vec{x} \neq \vec{0}$ & \multirow{2}{*}{$> 0$} 
			& \multirow{2}{*}{$> 0$} & \multirow{2}{*}{$> 0$} \\
			$\dotp{\mat{A}\vec{x}}{\vec{x}} = \vec{x}\herm\mat{A}\vec{x} > 0, \ \forall \vec{x} \neq \vec{0}$ & & & \\
			
			% \hline
			\Xhline{3\arrayrulewidth}
			positive semi-definite $\dotp{\T(\vec{x})}{\vec{x}} \ge 0, \ \forall \vec{x}$ & \multirow{2}{*}{$\ge 0$} 
			& \multirow{2}{*}{$\ge 0$} & \multirow{2}{*}{$\ge 0$} \\
			$\dotp{\mat{A}\vec{x}}{\vec{x}} = \vec{x}\herm\mat{A}\vec{x} \ge 0, \ \forall \vec{x}$ & & & \\
			
			% \hline
			\Xhline{3\arrayrulewidth}
			unitary (over $\C$) $\T^*\T = \mat{I}$ & \multirow{2}{*}{$|\lambda| = 1$} 
			& \multirow{2}{*}{$\texttimes$} & \multirow{2}{*}{$|\det(\mat{A})| = 1$} \\
			$\mat{A}\herm\mat{A} = \mat{I}$ & & & \\

			% \hline
			\Xhline{3\arrayrulewidth}
			orthogonal (over $\R$) $\T^*\T = \mat{I}$ & \multirow{2}{*}{$|\lambda| = 1$} 
			& \multirow{2}{*}{$\texttimes$} & \multirow{2}{*}{$\pm 1$} \\
			$\mat{A}^\intercal\mat{A} = \mat{I}$ & & & \\

			\hline
		\end{tabular}
		\renewcommand{\arraystretch}{1}
	\end{table}
\end{theorem}

\item \begin{theorem}{(8.28)} (實數,複數)Polar identity:
	\begin{itemize}
		\item 若$\V$ over $\R$,則\begin{equation}
			\dotp{\vec{u}}{\vec{v}} = \frac{1}{4}||\vec{u} + \vec{v}||^2 - \frac{1}{4}||\vec{u} - \vec{v}||^2
		\end{equation}
		\item 若$\V$ over $\C$,則\begin{equation}
			\dotp{\vec{u}}{\vec{v}} = \frac{1}{4}\sum_{k = 1}^{4}i^k||\vec{u} + i^k\vec{v}||^2
		\end{equation}
	\end{itemize}
\end{theorem}

\item \begin{theorem}{(8.35, (8-5)8.10)} (方陣,正交)若$\mat{A} \in \F^{n \times n}$為么正或是正交方陣,則\begin{itemize}
		\item $\cs(\mat{A})$和$\rs(\mat{A})$皆為單範正交集。
		\item 若$\mat{A}$不為方陣,則$\rs(\mat{A})$\textbf{未必}為單範正交集。
	\end{itemize}
\end{theorem}

\item \begin{theorem}{()} (方陣,$\tr$)若$\mat{A}, \mat{B} \in \F^{n \times n}$為方陣,且$\mat{A}$和$\mat{B}$么正相等,則$\tr(\mat{A}\herm\mat{A}) = \tr(\mat{B}\herm\mat{B})$。
\end{theorem}

\item \begin{theorem}{(8.60, 8.61, 8.62, 8.63, 8.68, 8.69)} (方陣,正交,相似,對角化,實數,複數)
	\begin{itemize}
		\item 若$\mat{A}, \mat{B} \in \C^{n \times n}$為複數方陣,且$\mat{A}$與$\mat{B}$么正相似,則以下$\mat{A}$與$\mat{B}$的性質等價
			\begin{itemize}
				\item 正規
				\item Hermitian
				\item 斜Hermitian
				\item 正定
				\item 半正定
				\item 么正
			\end{itemize}
		\item 若$\mat{A}, \mat{B} \in \R^{n \times n}$為實方陣,且$\mat{A}$與$\mat{B}$正交相似,則以下$\mat{A}$與$\mat{B}$的性質等價
		\begin{itemize}
			\item 對稱
			\item 斜對稱
			\item 正交
		\end{itemize}
		\item 若$\mat{A} \in \C^{n \times n}$為複數方陣,則$\mat{A}$為正規且上三角方陣$\iff$$\mat{A}$為對角方陣。
		\item 若$\mat{A} \in \C^{n \times n}$為複數方陣,則$\mat{A}$為正規方陣$\iff$$\mat{A}$可么正對角化。
		\item 若$\mat{A} \in \R^{n \times n}$為實方陣,則$\mat{A}$為對稱方陣$\iff$$\mat{A}$可正交對角化。
	\end{itemize}
\end{theorem}

\item \begin{theorem}{(8.89)} (方陣) Cholesky分解必須對稱且正定。
\end{theorem}

\item \begin{theorem}{(8.94)} (實數,複數)
	\begin{itemize}
		\item 若$\mat{A} \in \C^{m \times n}$為複數矩陣,則$\mat{A}\herm\mat{A}$與$\mat{A}\mat{A}\herm$皆正半定。
		\item 若$\mat{A} \in \R^{m \times n}$為實矩陣,則$\mat{A}^\intercal\mat{A}$與$\mat{A}\mat{A}^\intercal$皆正半定。
	\end{itemize}
\end{theorem}

\item \begin{theorem}{(8.142)} (方陣,正交) Householder矩陣為對稱且正交矩陣。
\end{theorem}

\item \begin{theorem}{(8.157, (8-47)8.94)} (正交,方陣,特徵)若$\mat{A} = \mat{U\Sigma V}\herm$為$\mat{A}$的奇異值分解,則
	\begin{itemize}
		\item \begin{equation}
			\sum_{i = 1}^{r} \sigma^2 = \sum_{i = 1}^{m}\sum_{j = 1}^{n}|a_{ij}|^{2}
		\end{equation}
		\item $\cs(\mat{V})$為$\mat{A}\herm\mat{A}$的特徵向量且為單範正交集。
		\item $\cs(\mat{U})$為$\mat{A}\mat{A}\herm$的特徵向量且為單範正交集。
		\item 若$\rnk(\mat{A}) = r$為非零奇異值個數,則 \begin{itemize}
			\item $\vec{v}_1, \vec{v}_2, \ \cdots, \vec{v}_r$為$\cs(\mat{A}\herm)$單範正交基底。
			\item $\vec{v}_{r + 1}, \ \cdots, \vec{v}_n$為$\Ker(\mat{A})$單範正交基底。
			\item $\vec{u}_1, \vec{u}_2, \ \cdots, \vec{u}_r$為$\cs(\mat{A})$單範正交基底。
			\item $\vec{u}_{r + 1}, \ \cdots, \vec{u}_m$為$\Ker(\mat{A}\herm)$單範正交基底。 
		\end{itemize}
	\end{itemize}
\end{theorem}
