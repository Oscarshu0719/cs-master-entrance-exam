\item \begin{theorem}{(8.3, 8.4, 8.5, 8.6, 8.8)} (方陣,可逆,特徵)若$\T \in \L(\V, \V)$,存在$\T^* \in \L(\V, \V)$使得
	\begin{equation}
		\dotp{\T(\vec{u})}{\vec{v}} = \dotp{\vec{u}}{\T^*(\vec{v})}
	\end{equation}
	稱$\T^*$為$\T$的伴隨(adjoint)算子,則
	\begin{itemize}
		\item $\T^*$唯一。
		\item 若$\beta = \{\vec{v}_1, \vec{v}_2, \ \cdots, \vec{v}_n\}$為$\V$\textbf{單範正交基底},則$[\T^*]_{\beta} = ([\T]_\beta)\herm$。
		\item 若$\mat{A} \in \F^{n \times n}$為方陣,則$\dotp{\mat{A}\vec{x}}{\vec{y}} = \dotp{\vec{x}}{\mat{A}\herm\vec{y}}$。
		\item $(\alpha\T + \beta\textit{U})^* = \overline{\alpha}\T^* + \overline{\beta}\textit{U}^*$
		\item $(\T^*)^* = \T$
		\item $(\T\textit{U})^* = \textit{U}^*\T^*$
		\item 若$\T$可逆,則$\T^*$可逆,$(\T^*)\inv = (\T\inv)^*$。
		\item 若$\lambda$為$\T$特徵根,則$\overline{\lambda}$為$\T^*$特徵根。
	\end{itemize}
\end{theorem}

\item \begin{theorem}{(8.13)} (方陣,特徵)若$\T \in \L(\V, \V)$,滿足
	\begin{equation}
		\T^*\T = \T\T^*
	\end{equation}
	稱$\T$為正規(normal)算子,則
	\begin{itemize}
		\item $||\T(\vec{v})|| = ||\T^*(\vec{v})||, \ \forall \vec{v} \in \V$
		\item $\T - c\mat{I}, \ \forall c \in \V$仍是正規算子。
		\item 若$\vec{v}$為$\T$相對於$\lambda$的特徵向量,則$\vec{v}$為$\T^*$相對於$\overline{\lambda}$的特徵向量。
		\item 若$\vec{v}_1, \vec{v}_2$分別是$\T$相對於$\lambda_1, \lambda_2$相異特徵根的特徵向量,則$\vec{v}_1 \perp \vec{v}_2$。
	\end{itemize}
\end{theorem}

\item \begin{theorem}{(8.14, 8.15, 8.17, 8.18, 8.20, 8.21, 8.22, 8.23, 8.54, 8.55, 8.56)} (方陣,正交,特徵,$\tr$,$\det$,可逆,實數,複數)若$\T \in \L(\V, \V)$,則
	\begin{table}[H]
		% \centering
		\renewcommand{\arraystretch}{2}
		\begin{tabular}{|c|c|c|c|}
			\hline
			定義 & $\lambda$ & $a_{ii}$ & $\det$ \\
			% \hline
			\Xhline{3\arrayrulewidth}
			self-adjoint $\T^* = \T$ & \multirow{3}{*}{$\in \R$} 
			& \multirow{3}{*}{$\in \R$} & \multirow{3}{*}{$\in \R$} \\
			Hermitian (over $\C$) $\mat{A}\herm = \mat{A}$ & & & \\
			\cline{1-1}
			symmetric (over $\R$) $\mat{A}^\intercal = \mat{A}$ & & & \\

			% \hline
			\Xhline{3\arrayrulewidth}
			skew self-adjoint $\T^* = -\T$ & \multirow{2}{*}{$0$或純虛數} 
			& \multirow{2}{*}{$0$或純虛數} & \multirow{2}{*}{$\begin{cases}
				\in \R, & \text{if} \ n \in 2k \\
				0\text{或純虛數}, & \text{if} \ n \in 2k + 1 \\
			\end{cases}$} \\
			skew Hermitian (over $\C$) $\mat{A}\herm = -\mat{A}$ & & & \\
			\hline
			skew symmetric (over $\R$) $\mat{A}^\intercal = -\mat{A}$ & $0$
			& $0$ & $\begin{cases}
				\in \R, & \text{if} \ n \in 2k \\
				0, & \text{if} \ n \in 2k + 1 \\
			\end{cases}$ \\

			% \hline
			\Xhline{3\arrayrulewidth}
			positive definite $\dotp{\T(\vec{x})}{\vec{x}} > 0, \ \forall \vec{x} \neq \vec{0}$ & \multirow{2}{*}{$> 0$} 
			& \multirow{2}{*}{$> 0$} & \multirow{2}{*}{$> 0$} \\
			$\dotp{\mat{A}\vec{x}}{\vec{x}} = \vec{x}\herm\mat{A}\vec{x} > 0, \ \forall \vec{x} \neq \vec{0}$ & & & \\
			
			% \hline
			\Xhline{3\arrayrulewidth}
			positive semi-definite $\dotp{\T(\vec{x})}{\vec{x}} \ge 0, \ \forall \vec{x}$ & \multirow{2}{*}{$\ge 0$} 
			& \multirow{2}{*}{$\ge 0$} & \multirow{2}{*}{$\ge 0$} \\
			$\dotp{\mat{A}\vec{x}}{\vec{x}} = \vec{x}\herm\mat{A}\vec{x} \ge 0, \ \forall \vec{x}$ & & & \\
			
			% \hline
			\Xhline{3\arrayrulewidth}
			unitary (over $\C$) $\T^*\T = \mat{I}$ & \multirow{2}{*}{$|\lambda| = 1$} 
			& \multirow{2}{*}{$\texttimes$} & \multirow{2}{*}{$|\det(\mat{A})| = 1$} \\
			$\mat{A}\herm\mat{A} = \mat{I}$ & & & \\

			% \hline
			\Xhline{3\arrayrulewidth}
			orthogonal (over $\R$) $\T^*\T = \mat{I}$ & \multirow{2}{*}{$|\lambda| = 1$} 
			& \multirow{2}{*}{$\texttimes$} & \multirow{2}{*}{$\pm 1$} \\
			$\mat{A}^\intercal\mat{A} = \mat{I}$ & & & \\

			\hline
		\end{tabular}
		\renewcommand{\arraystretch}{1}
	\end{table}
	\begin{itemize}
		\item 皆為正規,相異特徵根對應的特徵向量皆正交。
		\item 若$\mat{A} \in \C^{n \times n}$為複數方陣時,$\mat{A}$為對稱矩陣即$\mat{A}^\intercal = \mat{A}$時,\textbf{不}保證$\mat{A}$為正規。
		\item 么正矩陣與正交矩陣的特徵值皆可能為\textbf{虛數}(並非$\pm 1$)。
		\item 正定(over $\R \lor \C$)、么正、正交矩陣皆可逆。
		\item 若$\mat{A}$為Hermitian矩陣即$\mat{A}\herm = \mat{A}$時,$\mat{A}$的相異特徵根對應的特徵向量必正交(比正規更鬆的條件)。
	\end{itemize}
\end{theorem}

\item \begin{theorem}{(8.27, 8.30)} (方陣,實數,複數)若$\T \in \L(\V, \V)$,則
	\begin{itemize}
		\item 若$\dotp{\T(\vec{u})}{\vec{v}} = 0, \ \forall \vec{v}, \vec{u} \in \V$ (over $\C$ or $\R$),則$\T = \mat{O}$。
		\item 若$\dotp{\T(\vec{v})}{\vec{v}} = 0, \ \forall \vec{v} \in \V$ (over $\C$),則$\T = \mat{O}$,\textbf{$\V$ (over $\R$)時不成立}。
		\item 若$\mat{A}, \mat{B} \in \F^{n \times n}$為方陣,則$\mat{A} = \mat{B}$$\iff$$\vec{y}\herm\mat{A}\vec{x} = \vec{y}\herm\mat{B}\vec{x}, \forall \vec{x}, \vec{y} \in \F^{n \times 1}$
		\item 若$\mat{A}, \mat{B} \in \C^{n \times n}$為複數方陣,則$\mat{A} = \mat{B}$$\iff$$\vec{x}\herm\mat{A}\vec{x} = \vec{x}\herm\mat{B}\vec{x}, \forall \vec{x} \in \C^{n \times 1}$。\textbf{在$\R$時不成立}。
	\end{itemize}
\end{theorem}

\item \begin{theorem}{(8.31, 8.32, 8.34, 8.35)} (方陣,正交,實數,複數)
	\begin{itemize}
		\item 若$\mat{A} \in \C^{n \times n}$為複數方陣,則以下等價
		\begin{itemize}
			\item $\mat{A}$么正。
			\item $\dotp{\mat{A}\vec{x}}{\mat{A}\vec{y}} = \dotp{\vec{x}}{\vec{y}}, \ \forall \vec{x}, \vec{y} \in \C^{n \times 1}$
			\item $||\mat{A}\vec{x}|| = ||\vec{x}||, \ \forall \vec{x} \in \C^{n \times 1}$
			\item $\cs(\mat{A})$單範正交。
			\item $\rs(\mat{A})$單範正交。
			\item 若$\beta = \{\vec{v}_1, \vec{v}_2, \ \cdots, \vec{v}_n\}$為$\C^{n \times 1}$單範正交基底,則$\{\mat{A}\vec{v}_1, \mat{A}\vec{v}_2, \ \cdots, \mat{A}\vec{v}_n\}$亦是單範正交基底。			\end{itemize}
		\item 若$\mat{A} \in \R^{n \times n}$為實方陣,則以下等價
		\begin{itemize}
			\item $\mat{A}$正交。
			\item $\dotp{\mat{A}\vec{x}}{\mat{A}\vec{y}} = \dotp{\vec{x}}{\vec{y}}, \ \forall \vec{x}, \vec{y} \in \R^{n \times 1}$
			\item $||\mat{A}\vec{x}|| = ||\vec{x}||, \ \forall \vec{x} \in \R^{n \times 1}$
			\item $\cs(\mat{A})$單範正交。
			\item $\rs(\mat{A})$單範正交。
		\end{itemize}
	\end{itemize}
\end{theorem}

\item \begin{theorem}{(8.28)} (實數,複數)Polar identity:
	\begin{itemize}
		\item 若$\V$ over $\R$,則$\dotp{\vec{u}}{\vec{v}} = \frac{1}{4}||\vec{u} + \vec{v}||^2 - \frac{1}{4}||\vec{u} - \vec{v}||^2$。
		\item 若$\V$ over $\C$,則$\dotp{\vec{u}}{\vec{v}} = \frac{1}{4}\sum_{k = 1}^{4}i^k||\vec{u} + i^k\vec{v}||^2$。
	\end{itemize}
\end{theorem}

\item \begin{theorem}{(8.50, 8.53)} (方陣,實數,複數)
	\begin{itemize}
		\item 若$\mat{A} \in \C^{n \times n}$為複數方陣,則
		\begin{itemize}
			\item $\mat{A}\herm = \mat{A}$$\iff$$\vec{x}\herm\mat{A}\vec{x} \in \R, \ \forall \vec{x} \in \C^{n \times 1}$。
			\item 若$\mat{A}\herm = \mat{A}$且$\vec{x}\herm\mat{A}\vec{x} > 0, \ \forall \vec{x} \neq \vec{0}$,則$\mat{A}$正定。
			\item 若$\mat{A}\herm = \mat{A}$且$\vec{x}\herm\mat{A}\vec{x} \ge 0, \ \forall \vec{x} \in \R^{n \times 1}$,則$\mat{A}$半正定。
		\end{itemize}
		\item 若$\mat{A} \in \R^{n \times n}$為實方陣,則
		\begin{itemize}
			\item 若$\mat{A}^\intercal = \mat{A}$且$\vec{x}^\intercal\mat{A}\vec{x} > 0, \ \forall \vec{x} \neq \vec{0}$,則$\mat{A}$正定。
			\item 若$\mat{A}^\intercal = \mat{A}$且$\vec{x}^\intercal\mat{A}\vec{x} \ge 0, \ \forall \vec{x} \in \R^{n \times 1}$,則$\mat{A}$半正定。
		\end{itemize}
	\end{itemize}
\end{theorem}

\item \begin{theorem}{(8.60, 8.61, 8.62, 8.63, 8.69)} (方陣,正交,相似,對角化,實數,複數)
	\begin{itemize}
		\item 么正或正交相似必定相似,反之不然。
		\item 可么正或正交對角化,則可對角化。
		\item 若$\mat{A}, \mat{B} \in \C^{n \times n}$為複數方陣,且$\mat{A}$與$\mat{B}$么正相似,則以下$\mat{A}$與$\mat{B}$的性質等價
			\begin{itemize}
				\item 正規
				\item Hermitian
				\item 斜Hermitian
				\item 正定
				\item 半正定
				\item 么正
			\end{itemize}
		\item 若$\mat{A}, \mat{B} \in \R^{n \times n}$為實方陣,且$\mat{A}$與$\mat{B}$正交相似,則以下$\mat{A}$與$\mat{B}$的性質等價
		\begin{itemize}
			\item 對稱
			\item 斜對稱
			\item 正交
		\end{itemize}
		\item 若$\mat{A} \in \C^{n \times n}$為複數方陣,則$\mat{A}$為正規方陣$\iff$$\mat{A}$可么正對角化。
		\item 若$\mat{A} \in \R^{n \times n}$為實方陣,則$\mat{A}$為對稱方陣$\iff$$\mat{A}$可正交對角化。
	\end{itemize}
\end{theorem}

\item \begin{theorem}{(8.89)} (方陣,分解,實數)Cholesky分解: \\
	若$\mat{A} \in \R^{n \times n}$對稱實方陣,且$\mat{A}$正定,則$\mat{A} = \mat{LL}^\intercal$,其中$\mat{L}$為下三角且對角線元素皆正。
\end{theorem}

\item \begin{theorem}{(8.83, 8.90, 8.91, 8.92, 8.93, 8.94, 8.95)} (方陣,實數,複數)
	\begin{itemize}
		\item 若$\mat{A} \in \C^{n \times n}$為複數方陣,則以下等價
			\begin{itemize}
				\item $\mat{A}$正定。
				\item $\exists \mat{B} \in \C^{m \times n}$為複數矩陣且\textbf{$\rnk(\mat{B}) = n$},使得$\mat{A} = \mat{B}\herm\mat{B}$。
				\item 存在$\mat{B}$為正定矩陣,使得$\mat{A} = \mat{B}^2$。
			\end{itemize}
		\item 若$\mat{A} \in \R^{n \times n}$為對稱實方陣,則以下等價
			\begin{itemize}
				\item $\mat{A}$正定。
				\item $\exists \mat{B} \in \R^{m \times n}$為實矩陣且\textbf{$\rnk(\mat{B}) = n$},使得$\mat{A} = \mat{B}^\intercal\mat{B}$。
				\item 主子行列式(principal minors)$\Delta_k(\mat{A}) > 0$(必須為對稱實方陣)。
				\item 存在$\mat{B}$為正定矩陣,使得$\mat{A} = \mat{B}^2$。
			\end{itemize}
		\item 若$\mat{A} \in \C^{n \times n}$為複數方陣,則以下等價
			\begin{itemize}
				\item $\mat{A}$正半定。
				\item $\exists \mat{B} \in \C^{m \times n}$為複數矩陣,使得$\mat{A} = \mat{B}\herm\mat{B}$。
				\item 存在$\mat{B}$為正半定矩陣,使得$\mat{A} = \mat{B}^2$。
			\end{itemize}
		\item 若$\mat{A} \in \R^{n \times n}$對稱實方陣,則以下等價
			\begin{itemize}
				\item $\mat{A}$正半定。
				\item $\exists \mat{B} \in \R^{m \times n}$為實矩陣,使得$\mat{A} = \mat{B}^\intercal\mat{B}$。
				\item 存在$\mat{B}$為正半定矩陣,使得$\mat{A} = \mat{B}^2$。
			\end{itemize}
		\item 若$\mat{A} \in \C^{m \times n}$為複數矩陣,則$\mat{A}\herm\mat{A}$與$\mat{A}\mat{A}\herm$皆正半定。
		\item 若$\mat{A} \in \R^{m \times n}$為實矩陣,則$\mat{A}^\intercal\mat{A}$與$\mat{A}\mat{A}^\intercal$皆正半定。
	\end{itemize}
\end{theorem}

\item \begin{theorem}{(8.102)} (方陣,特徵,實數,複數)主軸定理(Principal axis theorem):
	\begin{itemize}
		\item 若$\mat{A} \in \C^{n \times n}$為Hermitian複數方陣,則
		\begin{equation}
			Q(\vec{x}) = \vec{x}\herm\mat{A}\vec{x} = \lambda_1|y_1|^2 + \lambda_2|y_2|^2 + \cdots + \lambda_n|y_n|^2
		\end{equation}
		其中$\lambda_1, \lambda_2, \ \cdots, \lambda_n$為$\mat{A}$特徵根,$y_1, y_2, \ \cdots, y_n \in \C$。
		\item 若$\mat{A} \in \R^{n \times n}$為對稱實方陣,則
		\begin{equation}
			Q(\vec{x}) = \vec{x}^\intercal\mat{A}\vec{x} = \lambda_1y_1^2 + \lambda_2y_2^2 + \cdots + \lambda_ny_n^2
		\end{equation}
		其中$\lambda_1, \lambda_2, \ \cdots, \lambda_n$為$\mat{A}$特徵根,$y_1, y_2, \ \cdots, y_n \in \R$。
	\end{itemize}
\end{theorem}

\item \begin{theorem}{(8.108, 8.109, 8.111)} (方陣,特徵,實數,複數)若$\mat{A} \in \C^{n \times n}$為Hermitian複數方陣或$\mat{A} \in \R^{n \times n}$為對稱實方陣,$\lambda_1 \le \lambda_2 \le \cdots \le \lambda_n$為$\mat{A}$特徵根,則
	\begin{itemize}
		\item 
		\begin{equation}
			\rho(\vec{x}) = \frac{Q(\vec{x})}{||\vec{x}||^2} = \frac{\dotp{\mat{A}\vec{x}}{\vec{x}}}{\dotp{\vec{x}}{\vec{x}}}, \ \vec{x} \neq \vec{0}
		\end{equation}
		稱$\rho(\vec{x})$為$\vec{x}$的Rayleigh quotient。
		\item Rayleigh principle:
		\begin{itemize}
			\item $\lambda_1 \le \rho(\vec{x}) \le \lambda_n$
			\item $\max_{||\vec{x}|| = 1}\rho(\vec{x}) = \lambda_n$
			\item $\min_{||\vec{x}|| = 1}\rho(\vec{x}) = \lambda_1$
		\end{itemize}
	\end{itemize}
\end{theorem}

\item \begin{theorem}{(8.141, 8.142, 8.145)} (實數,方陣)若$\vec{u} \in \R^{n \times 1}, ||\vec{u}|| = 1$,則稱
	\begin{equation}
		\mat{H} = \mat{I} - 2\vec{uu}^\intercal
	\end{equation}
	為相對於$\vec{u}$的Householder矩陣或基本鏡射算子(elementary reflextor),則
	\begin{itemize}
		\item $\mat{H}^\intercal = \mat{H}$為對稱方陣。
		\item $\mat{H}^\intercal\mat{H} = \mat{I}$為正交方陣。
		\item $||\vec{x}|| = ||\vec{y}||, \vec{u} = \frac{\vec{x} - \vec{y}}{||\vec{x} - \vec{y}||}$,則
		\begin{itemize}
			\item $||\vec{x} - \vec{y}||^2 = 2(\vec{x} - \vec{y})^\intercal\vec{x}$
			\item $\mat{H}\vec{x} = \vec{y}$
		\end{itemize}
	\end{itemize}
\end{theorem}

\item \begin{theorem}{(8.154, 8.157, 8.160, 8.161, 8.163, 8.164)} (正交,方陣,特徵,可逆,解,實數,複數)
	\begin{itemize}
		\item 若$\mat{A} \in \C^{m \times n}$為複數矩陣,$s = \min\{m, n\}$,則稱$\mat{A} = \mat{U\Sigma V}\herm$為$\mat{A}$的奇異值分解(Singular Value Decomposition,SVD),
			其中$\mat{U} \in \C^{m \times m}$為複數方陣,$\mat{\Sigma} \in \C^{m \times n}$為複數矩陣,$\mat{V} \in \C^{n \times n}$為複數方陣。
			\begin{itemize}
				\item $\mat{U}, \mat{V}$為么正。
				\item $(\mat{\Sigma})_{ij} = 0, \ \forall i \neq j$
				\item $(\mat{\Sigma})_{ii} = \sigma_i, \ \forall i = 1, 2, \ \cdots, s$
				\item $\sigma_1 \ge \sigma_2 \ge \cdots \ge \sigma_s$稱為奇異值。
			\end{itemize}
		\item 若$\mat{A} \in \R^{m \times n}$為實矩陣,$s = \min\{m, n\}$,則稱$\mat{A} = \mat{U\Sigma V}^\intercal$為$\mat{A}$的奇異值分解(Singular Value Decomposition,SVD),
			其中$\mat{U} \in \R^{m \times m}$為實方陣,$\mat{\Sigma} \in \R^{m \times n}$為實矩陣,$\mat{V} \in \R^{n \times n}$為複數方陣。
			\begin{itemize}
				\item $\mat{U}, \mat{V}$為正交。
				\item $(\mat{\Sigma})_{ij} = 0, \ \forall i \neq j$
				\item $(\mat{\Sigma})_{ii} = \sigma_i, \ \forall i = 1, 2, \ \cdots, s$
				\item $\sigma_1 \ge \sigma_2 \ge \cdots \ge \sigma_s$稱為奇異值。
			\end{itemize}
		\item 奇異值必定唯一,但$\mat{U}, \mat{V}$未必唯一。
		\item 若$\rnk(\mat{A}) = r$為非零奇異值個數,則
			\begin{itemize}
				\item $\vec{v}_1, \vec{v}_2, \ \cdots, \vec{v}_r$為$\cs(\mat{A}\herm)$單範正交基底。
				\item $\vec{v}_{r + 1}, \ \cdots, \vec{v}_n$為$\Ker(\mat{A})$單範正交基底。
				\item $\vec{u}_1, \vec{u}_2, \ \cdots, \vec{u}_r$為$\cs(\mat{A})$單範正交基底。
				\item $\vec{u}_{r + 1}, \ \cdots, \vec{u}_m$為$\Ker(\mat{A}\herm)$單範正交基底。 
			\end{itemize}
		\item 若$\mat{A} = \mat{U\Sigma V}\herm$,則稱$\mat{A}^+ = \mat{V}\mat{\Sigma}^+\mat{U}\herm$為$\mat{A}$的虛反矩陣(pseudoinverse),則
			\begin{itemize}
				\item $(\mat{\Sigma}^+)_{ij} = 0, \ \forall i \neq j$。
				\item $(\mat{\Sigma}^+)_{ii} = \begin{cases}
					\frac{1}{\sigma_i}, & \ \text{if} \sigma_i \neq 0 \\
					0, & \ \text{if} \sigma_i = 0
				\end{cases}$
				\item $\mat{O}^+ = \mat{O}$
				\item 若$\mat{A}$可逆,則$\mat{A}^+ = \mat{A}\inv$。
				\item $\rnk(\mat{A}) = \rnk(\mat{A}^+)$
				\item $(c\mat{A})^+ = \frac{1}{c}\mat{A}^+$
				\item $(\mat{A}^+)^+ = \mat{A}$
				\item $(\mat{A}^\intercal)^+ = (\mat{A}^+)^\intercal$
				\item 若$\mat{A} \in \F^{m \times r}, \mat{B} \in \F^{r \times n}, (\mat{AB})^+ = \mat{B}^+\mat{A}^+$,其中$\rnk(\mat{A}) = \rnk(\mat{B}) = r$。
				\item $\vec{x}_0 = \mat{A}^+\vec{b}$為使$||\mat{A}\vec{x} - \vec{b}||_2$最小的$\vec{x}$,
				即$||\vec{x}_0||_2$為最小者。
				\item 若$\mat{A}$行獨立,則$\mat{A}^+ = (\mat{A}^\intercal\mat{A})\inv\mat{A}^\intercal$
				\item 若$\mat{A}$列獨立,則$\mat{A}^+ = \mat{A}^\intercal(\mat{A}\mat{A}^\intercal)\inv$
			\end{itemize}
	\end{itemize}
\end{theorem}
