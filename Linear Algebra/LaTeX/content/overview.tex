\section{Overview}

\begin{enumerate}
    \item 本文頁碼標記依照實體書\cite{1}\cite{2}\cite{3}的頁碼。
    \item TKB筆記\cite{4}章節頁碼:
    \begin{table}[H]
        \centering
        \begin{tabular}{|c|c|}
            \hline
            Chapter & Page No. \\
            \Xhline{2\arrayrulewidth}
            1 & 1 \\
            \hline
            2 & 33 \\
            \hline
            3 & 52 \\
            \hline
            4 & 80 \\
            \hline
            5 & 116 \\
            \hline
            6 & 154 \\
            \hline
            7 & 175 \\
            \hline
            8 & 206 \\
            \hline
        \end{tabular}
    \end{table}
    \item 第4章最重要,4.5超重要,6.6重要,7.4超重要。
    \item 參考TKB筆記\cite{4}內容,部分刪減8.6, 8.8, 8.10, 8.11內容。刪去3.6, 4.7, 5.7, 8.4, 8.9全部內容。
    \item 證明:(參考TKB筆記\cite{4}中頁碼)
    \begin{enumerate}
        \item 102
        \item 135
        \item 143
        \item 144
        \item 181
        \item 190
        \item 191
        \item 228
    \end{enumerate}
    \item 必考:(參考TKB筆記\cite{4}中頁碼)
    \begin{enumerate}
        \item 110
        \item 129
        \item 133
        \item 136
        \item 138
        \item 139
        \item 167
        \item 183
        \item 194
        \item 195
        \item 212
        \item 218
    \end{enumerate}
    \item 中英轉換:
    \begin{itemize}
        \item ellipsoid 橢球
        \item hyperbola 雙曲線
        \item ellipse 橢圓
        \item paraboloid 拋物面
        \item perpendicular 垂直的
        \item conic 圓錐
        \item parallelepiped 平行六面體
    \end{itemize}
    \item \begin{equation}
        \begin{aligned}
            \sinh x = \frac{e^x - e^{-x}}{2} \\
            \cosh x = \frac{e^x + e^{-x}}{2}
        \end{aligned}
    \end{equation}
    \item 三角函數:
    \begin{itemize}
        \item 和角公式: \begin{subequations}
                \begin{align}
                    \sin(\alpha + \beta) & = \sin\alpha\cos\beta + \cos\alpha\sin\beta \\
                    \sin(\alpha - \beta) & = \sin\alpha\cos\beta - \cos\alpha\sin\beta \\
                    \cos(\alpha + \beta) & = \cos\alpha\cos\beta - \sin\alpha\sin\beta \\
                    \cos(\alpha - \beta) & = \cos\alpha\cos\beta + \sin\alpha\sin\beta
                \end{align}
            \end{subequations}
        \item 和差化積: \begin{subequations}
            \begin{align}
                \sin\alpha + \sin\beta & = 2\sin\frac{\alpha + \beta}{2}\cos\frac{\alpha - \beta}{2} \\
                \sin\alpha - \sin\beta & = 2\cos\frac{\alpha + \beta}{2}\sin\frac{\alpha - \beta}{2} \\
                \cos\alpha + \cos\beta & = 2\cos\frac{\alpha + \beta}{2}\cos\frac{\alpha - \beta}{2} \\
                \cos\alpha - \cos\beta & = -2\sin\frac{\alpha + \beta}{2}\sin\frac{\alpha - \beta}{2}
            \end{align}
        \end{subequations}
        \item 積化和差: \begin{subequations}
            \begin{align}
                2\sin\alpha\cos\beta & = \sin(\alpha + \beta) + \sin(\alpha - \beta) \\
                2\cos\alpha\sin\beta & = \sin(\alpha + \beta) - \sin(\alpha - \beta) \\
                2\cos\alpha\cos\beta & = \cos(\alpha + \beta) + \cos(\alpha - \beta) \\
                2\sin\alpha\sin\beta & = \cos(\alpha + \beta) - \cos(\alpha - \beta)
            \end{align}
        \end{subequations}
    \end{itemize}
    \item 泰勒級數:\begin{subequations}
        \begin{align}
            (1 + x)^{\alpha} & = \sum_{n = 0}^{\infty}\binom{\alpha}{n}x^n, \ \forall |x| < 1, \alpha \in \C \\
            & = 1 + \alpha x + \frac{\alpha(\alpha - 1)}{2!}x^2 + \cdots + \frac{\alpha(\alpha - 1)\cdots(\alpha - n + 1)}{n!}x^n + \cdots \\
            \frac{1}{1 - x} & = \sum_{n = 0}^{\infty}x^n, \ \forall |x| < 1 \\
            e^x & = \sum_{n = 0}^{\infty}\frac{x^n}{n!}, \ \forall x \\
            \ln(1 - x) & = -\sum_{n = 1}^{\infty}\frac{x^n}{n}, \ \forall x \in [-1, 1) \\
            \sin x & = \sum_{n = 0}^{\infty}\frac{(-1)^n}{(2n + 1)!}x^{2n + 1}, \ \forall x \\
            \cos x & = \sum_{n = 0}^{\infty}\frac{(-1)^n}{(2n)!}x^{2n}, \ \forall x
        \end{align}
    \end{subequations}
    \item \begin{equation}
        e^{\ii x} = \cos x + \ii\sin x, \ |e^{\ii x}| = 1
    \end{equation}
\end{enumerate}

\pagebreak