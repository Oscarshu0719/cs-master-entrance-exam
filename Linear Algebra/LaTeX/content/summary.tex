\item \begin{theorem}{()} \quad\quad \begin{itemize}
        \item (\textbf{FALSE}) For any non-zero real \textbf{symmetric} matrix, its SVD can be the same as its eigenvalue decomposition.
        \item For any non-zero real matrix $\mat{A}$, $\mat{A}^\intercal\mat{A}$'s SVD can be the same as its eigenvalue decomposition. 
        因為$\mat{A}^\intercal\mat{A}$正半定,$\mat{A}$可么正對角化,$\exists \ \mat{P}$為么正矩陣,使得 \begin{equation}
            \begin{aligned}
                & \mat{P}^\intercal(\mat{A}^\intercal\mat{A})\mat{P} = \mat{D} \\
                \Rightarrow & \mat{A}^\intercal\mat{A} = \mat{P}\mat{D}\mat{P}^\intercal
            \end{aligned} 
        \end{equation} 為$\mat{A}^\intercal\mat{A}$的SVD。
        \item 可對角化\textbf{不}保證non-singular。
        \item If $\mat{A}$, $\mat{B}$ and $\mat{A + B}$ are non-singular square matrices, and $\mat{A}\inv + \mat{B}\inv$ is also non-singular.
        Since $\mat{A}(\mat{A} + \mat{B})\inv\mat{B}$ is invertible, \begin{equation}
            (\mat{A}(\mat{A} + \mat{B})\inv\mat{B})\inv = \mat{B}\inv(\mat{A} + \mat{B})\mat{A}\inv = \mat{A}\inv + \mat{B}\inv
        \end{equation}
        \item The transition matrix from one basis to another must be \textbf{non-singular}, but a linear transformation matrix can be singular.
        \item (\textbf{FALSE}) Let $\spc{S}$ be a subset of an inner product space, then $\spc{S} = (\spc{S}^{\bot})^{\bot}$.
        \item (\textbf{FALSE}) Let $\spc{S}_1, \spc{S}_2$ be subsets of an inner product space, and $\spc{S}_1^{\bot} = \spc{S}_2^{\bot}$, then $\spc{S}_1 = \spc{S}_2$.
        \item (\textbf{FALSE}) If $\spc{V}$ is orthogonal to $\spc{W}$, then $\spc{V}^\bot$ is orthogonal to $\spc{W}^\bot$.
        \item SVD中singular value遞減排序。
        \item $\mat{A}\vec{x} = \vec{b} \ (\vec{b} \neq \vec{0})$ is consistent, then solution set is \textbf{NOT} a subspace, since $\vec{0}$ is NOT included.
        \item If $\spc{W}$ is a subset of $\R^n$, but $\spc{W} \cup \spc{W}^\perp \neq \R^n$, since $\spc{W}^\perp$ is NOT a subset.
        \item If $\V \in \R^{m \times n}$, $<\mat{A}, \mat{B}> = \tr(\mat{B}^\intercal\mat{A})$ does \textbf{NOT} define an inner space in $\V$, since if $m \neq n$, $\mat{B}^\intercal\mat{A}$ may \textbf{NOT} exist.
        \item (101NTU-10) If \begin{equation}
            \mat{A} = 
            \begin{bmatrix} 
				0 & 1 & 0 & \cdots & 0 & 0 \\
                1 & 0 & 1 & \cdots & 0 & 0 \\
                0 & 1 & 0 & \cdots & 0 & 0 \\
                \vdots & \vdots & \vdots & \ddots & \vdots & \vdots \\
                0 & 0 & 0 & \cdots & 0 & 1 \\
                0 & 0 & 0 & \cdots & 1 & 0
			\end{bmatrix}
        \end{equation}, then \begin{equation}
            \lambda_{\mat{A}} = 2 \times \cos(\frac{k\pi}{n + 1}), \ k = 1,\ 2, \ \cdots, \ n
        \end{equation}
    \end{itemize}
\end{theorem}