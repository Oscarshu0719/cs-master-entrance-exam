\item \begin{theorem}{(7, 8, 10, 12)} System types:\begin{itemize}
        \item Multiprogramming system:\begin{itemize}
            \item Multiprogramming degree:存在系統內執行的process個數。
            \item 執行方式:\begin{itemize}
                \item Concurrent(並行):CPU在多個processes之間切換,同一時間只有一個process執行。
                \item Parallel(平行):同一時間多個processes在不同CPU上執行。
            \end{itemize}
        \end{itemize}
        \item Time-sharing (Multitasking) system:\begin{itemize}
            \item 屬於Multiprogramming的一種。
            \item CPU頻繁在任務間切換,response time短,適合interactive computing,經常採用Round-Robin scheduling。
            \item 採用virtual memory。
            \item 採用spooling (Simultaneous Peripheral Operations On-Line processing),將disk當作buffer,將輸出檔先儲存在disk,直到device取走,達到多個I/O devices的效果。
        \end{itemize}
        \item Multiprocessors (Tightly-coupled, Parallel) system:\begin{itemize}
            \item Processors溝通通常採用shared memory。
            \item Graceful degradation (Fail-soft):系統不會因為某些軟硬體元件故障而停頓。具備此能力的系統稱為Fault-tolerant。
            \item Multicores,即一晶片多核,相較multiprocessors速度快且省電。
            \item 類型:\begin{itemize}
                \item SMP (Symmetric MultiProcessors):所有processors能力與權利相同,troughput與reliability較ASMP好,但OS設計較ASMP複雜。
                \item ASMP (ASymmetric MultiProcessors):Master-slave架構,master負責工作與資源分配,其餘為slave。
            \end{itemize}
        \end{itemize}
        \item Distributed (Loosely-coupled) system:\begin{itemize}
            \item Processors的clock和OS未必相同。
            \item Processors溝通通常採用message passing,透過network連接。
            \item C/S model達到resources sharing;P2P (Peer-to-peer)不區分C/S。
        \end{itemize}
        \item Real-time system:\begin{itemize}
            \item 類型:\begin{itemize}
                \item Hard real-time system:\begin{itemize}
                    \item 若工作未在deadline前完成,即為失敗。
                    \item 處理時間過長或無法預測的設備少用,disk少用,virtual memory不用。
                    \item 不與time-sharing system並存。
                    \item 減少kernel介入時間。
                \end{itemize}
                \item Soft real-time sharing:\begin{itemize}
                    \item real-time processes必須持續保有最高priority直到結束。
                    \item 必須支援preemptive priority scheduling。
                    \item 不提供Aging。
                    \item 減少kernel的dispatch latency。
                    \item 可與virtual memory共存,但是real-time processes不能被swapped out直到完工。
                    \item 可與time-sharing並存。
                    \item 現行OS皆支援。
                \end{itemize}
            \end{itemize}
        \end{itemize}
        \item Batch system:非急迫或週期性的任務一起執行,提高資源使用率,不適用real-time及user-interactive。
    \end{itemize}
\end{theorem}
