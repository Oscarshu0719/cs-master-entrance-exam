\begin{theorem}{(7)} Spooling (Simultaneous Peripheral Operations On-Line processing):將disk當作buffer,將輸出檔先儲存在disk,直到device取走,達到多個I/O devices的效果。
\end{theorem}

\begin{theorem}{(10)} Real-time system:\begin{itemize}
        \item 類型:\begin{itemize}
            \item Hard real-time system:\begin{itemize}
                \item 若工作未在deadline前完成,即為失敗。
                \item 處理時間過長或無法預測的設備少用,disk少用,virtual memory不用。
                \item 不與time-sharing system並存。
                \item 減少kernel介入時間。
            \end{itemize}
            \item Soft real-time sharing:\begin{itemize}
                \item real-time processes必須持續保有最高priority直到結束。
                \item 必須支援preemptive priority scheduling。
                \item 不提供Aging。
                \item 減少kernel的dispatch latency。
                \item 可與virtual memory和time-sharing共存,但是real-time processes不能被swapped out直到完工。
                \item 現行OS皆支援。
            \end{itemize}
        \end{itemize}
    \end{itemize}
\end{theorem}
