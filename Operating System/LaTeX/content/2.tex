\begin{theorem}{(16)} Interrupted:\begin{itemize}
        \item CPU親自監督I/O完成。
        \item 若頻繁interrupt CPU使用率仍會很低,因此若I/O時間不長,polling反而可能較有利。
        \item 分類:\begin{itemize}
            \item External interrupt:CPU外的周邊設備發出,例如device controller發出I/O-completed。
            \item Internal interrupt:CPU執行process遭遇重大error,例如Divide-by-zero,執行priviledged instruction in user mode。
            \item Software interrupt:Process需要OS提供服務,呼叫system call。
            \item Trap:Software-generated。Catch arithmetic error,即CPU執行process遭遇重大error,例如Divide-by-zero。Process需要OS提供服務,會先發trap通知OS。
        \end{itemize}
        \item Interrupt要分priority。\begin{itemize}
            \item Non-maskable interrupt:通常是重大error引起,需要立即處理,例如internal interrupt。
            \item Maskable interrupt:可以忽略或是延後處理,例如software interrupt。
        \end{itemize}
    \end{itemize}
\end{theorem}

\begin{theorem}{(17)} DMA (Direct Memory Access):\begin{itemize}
        \item DMA controller代替CPU負責I/O與memory間的傳輸。
        \item 適用block-transfer,interrupt頻率較低,一個block才中斷一次,但設計較複雜。
        \item 造成resource contention(IF、MEM和WB週期),因此採用interleaving (cycle stealing),與CPU輪流使用memory與bus。
        \item DMA比CPU有更高priority (SJF)。
    \end{itemize}
\end{theorem}

\begin{theorem}{(18)} I/O: \begin{itemize}
        \item Non-blocking I/O與Asynchronous I/O差異:前者會有多少通知多少,後者會等I/O全部完成才通知process。
        \item 種類:\begin{itemize}
            \item Memory-mapped I/O:無專門I/O指令,但特別分一塊memory給I/O,寫入該塊memory視為I/O操作。
            \item Isolated I/O:有專門I/O指令。
        \end{itemize}
    \end{itemize}
\end{theorem}
