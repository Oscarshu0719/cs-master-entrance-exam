\item \begin{theorem}{(16, 17)} I/O運作方式:\begin{itemize}
        \item Polling (Busy-waiting, Programmed):CPU需不斷polling I/O device controller直到完成。
        \item Interrupted:\begin{itemize}
            \item I/O完成時,I/O device controller會interrupt,使CPU暫停目前process放到ready state queue。
            \item CPU親自監督I/O完成。
            \item 若頻繁interrupt CPU使用率仍會很低,因此若I/O時間不長,polling反而可能較有利。
            \item 分類:\begin{itemize}
                \item \begin{itemize}
                    \item External interrupt:CPU外的周邊設備發出,例如device controller發出I/O-completed。
                    \item Internal interrupt:CPU執行process遭遇重大error,例如Divide-by-zero,執行priviledged instruction in user mode。
                    \item Software interrupt:Process需要OS提供服務,呼叫system call。
                \end{itemize}
                \item \begin{itemize}
                    \item Interrupt:Hardware-generated,例如device controller發出I/O-completed。
                    \item Trap:Software-generated。Catch arithmetic error,即CPU執行process遭遇重大error,例如Divide-by-zero。Process需要OS提供服務,會先發trap通知OS。
                \end{itemize}
                \item Interrupt要分priority。\begin{itemize}
                    \item Non-maskable interrupt:通常是重大error引起,需要立即處理,例如internal interrupt。
                    \item Maskable interrupt:可以忽略或是延後處理,例如software interrupt。
                \end{itemize}
            \end{itemize}
        \end{itemize}
        \item DMA (Direct Memory Access):\begin{itemize}
            \item DMA controller代替CPU負責I/O與memory間的傳輸。
            \item 適用block-transfer。
            \item interrupt頻率較低,但設計較複雜。
            \item 造成resource contention(IF、MEM和WB週期),因此採用interleaving (cycle stealing),與CPU輪流使用memory與bus。
            \item DMA比CPU有更高priority (SJF)。
        \end{itemize}
    \end{itemize}
\end{theorem}

\item \begin{theorem}{(18)} \quad\quad \begin{itemize}
        \item Non-blocking I/O與Asynchronous I/O差異:前者會有多少通知多少,後者會等I/O全部完成才通知process。
        \item I/O種類:\begin{itemize}
            \item Memory-mapped I/O:無專門I/O指令,但特別分一塊memory給I/O,寫入該塊memory視為I/O操作。
            \item Isolated I/O:有專門I/O指令。
        \end{itemize}
    \end{itemize}
\end{theorem}

\item \begin{theorem}{(22)} CPU protection:利用timer,時間到timer發出timer-out interrupt通知OS,讓OS強制取回CPU。
\end{theorem}

