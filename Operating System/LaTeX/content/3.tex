\begin{theorem}{(34)} Virtual machine:\begin{itemize}
        \item Host:Underlying hardware system.
        \item Virtual Machine Manager (VMM, Hypervisor).
        \item Guest:Process provided by VM, for example, OS, applications.
        \item 不易開發VMM,因為要複製與底層host hardware一模一樣的VM非常困難,例如modes control and transition、資源調度和I/O device and controller之模擬。
        \item 效能比host hardware差。
        \item Implementations:\begin{itemize}
            \item Type 0:Hardware.
            \item Type 1:Kernel mode. \begin{itemize}
                \item OS-like software:只提供virtualization。
                \item General-purpose OS在kernel mode提供VMM services。
            \end{itemize}
            \item Type 2:User mode:Applications that provide VMM services.
        \end{itemize}
        \item Virtualization variations:\begin{itemize}
            \item Paravirtualization:Present guest similar but NOT identical to host hardware. Guest should be modified to run on paravirtualized hardware.
            \item Emulators:Applications to run on a different hardware environment.
            \item Application containment (container):Not virtualization at all, but provides segregating applications from the OS.
        \end{itemize}
        \item Java Virtual Machine (JVM):只提供規格(class loader,class verifier和Java interpreter),並非實現。
    \end{itemize}
\end{theorem}

\begin{theorem}{(52)} \quad\quad \begin{itemize}
        \item Software as a service(SaaS):Applications, e.g. Dropbox, Gmail.
        \item Platform as a service(PaaS):Software stack, i.e. APIs for softwares, e.g. DB server.
        \item Infrastructure as a service(IaaS):Servers or storage, e.g. storage for backup.
    \end{itemize}
\end{theorem}
