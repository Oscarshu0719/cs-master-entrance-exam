\begin{theorem}{(67)} Scheduler:\begin{itemize}
        \item Long-term (Job) scheduler:通常僅\textbf{batch system採用},從job queue中選jobs載入memory。執行頻率最低,可以調控multiprogramming degree與CPU-bound與I/O-bound jobs的比例。
        \item Short-term (CPU, process) scheduler:從ready queue選擇一個process分派給CPU執行。\textbf{所有系統都需要},執行頻率最高,\textbf{無法}調控multiprogramming degree與CPU-bound與I/O-bound jobs的比例。
        \item Medium-term scheduler:Memory space不足且有其他processes需要更多memory時執行,選擇Blocked或lower priority process swap out to disk。僅\textbf{Time-sharing system採用},batch和real-time systems不採用,可以調控multiprogramming degree與CPU-bound與I/O-bound jobs的比例。
    \end{itemize}
\end{theorem}

\begin{theorem}{(70)} Dispatcher: \begin{itemize}
        \item 將CPU真正分配給CPU scheduler選擇的process。
        \item Context switching.
        \item Switch mode to user mode.
        \item Jump to execution entry of user process.
    \end{itemize}
\end{theorem}

\begin{theorem}{(72, 78, 82, 84)} CPU scheduling: \begin{itemize}
        \item Non-preemptive SJF\textbf{不}適合用在\textbf{short-term} scheduler,因為很難在短時間算出next CPU burst;long-term scheduler較合適。
        \item MFQ雖然不公平,但\textbf{NO} starvation。
        \item Linux指定processes\textbf{不要}移轉到某些processors。
        \item Worst-case CPU utilization for scheduling $n$ processes using Rate-monotonic: \begin{equation}
            \begin{aligned}
                & 2 \times (2^{\frac{1}{n}} - 1) \\
                \Rightarrow & \ (n \rightarrow \infty) = 69\%
            \end{aligned}
        \end{equation}
        \item Dispatch latency:\begin{itemize}
            \item Conflict phase:preempts kernel,並且low-priority process releases needed resources for high-priority process。
            \item Dispatch phase:Context switching, change mode to user mode, and jump to the user process.
        \end{itemize}
    \end{itemize}
\end{theorem}