\begin{theorem}{(67)} Scheduler:\begin{itemize}
        \item Long-term (Job) scheduler:通常僅\textbf{batch system採用},從job queue中選jobs載入memory。執行頻率最低,可以調控multiprogramming degree與CPU-bound與I/O-bound jobs的比例。
        \item Short-term (CPU, process) scheduler:從ready queue選擇一個process分派給CPU執行。\textbf{所有系統都需要},執行頻率最高,\textbf{無法}調控multiprogramming degree與CPU-bound與I/O-bound jobs的比例。
        \item Medium-term scheduler:Memory space不足且有其他processes需要更多memory時執行,選擇Blocked或lower priority process swap out to disk。僅\textbf{Time-sharing system採用},batch和real-time systems不採用,可以調控multiprogramming degree與CPU-bound與I/O-bound jobs的比例。
    \end{itemize}
\end{theorem}

\begin{theorem}{(70)} Dispatcher: \begin{itemize}
        \item 將CPU真正分配給CPU scheduler選擇的process。
        \item Context switch.
        \item Switch mode to user mode.
        \item Jump to execution entry of process.
    \end{itemize}
\end{theorem}
