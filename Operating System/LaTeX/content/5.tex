\begin{theorem}{(136)} Deadlock:\begin{itemize}
        \item \textbf{必要}條件:\textbf{Mutual exclusion}, \textbf{hold and wait}, \textbf{no preemption}, and \textbf{circular wait}.
        \item 資源分配圖:\begin{itemize}
            \item No cycle一定no deadlock。
            \item 有cycle\textbf{不}一定deadlock。
            \item 若每一類resources皆single-instance,則有cycle一定deadlock。
        \end{itemize}
    \end{itemize}
\end{theorem}

\begin{theorem}{(139)} Deadlock prevention:\begin{itemize}
        \item 打破必要條件:\begin{itemize}
            \item Mutual exclusion:無法破除,這是與生俱來的性質。
            \item Hold and wait:除非可以一次獲得所有資源,否則不得持有任何資源;或可持有部分資源,但申請其他資源前,需先放掉所持有的所有資源。
            \item No preemption:改為preemption,但可能starvation。
            \item Circular wait:\begin{itemize}
                \item 每個資源有unique resource ID。
                \item process需依照resource ID依序\textbf{遞增}提出申請,及持有的resource ID,\textbf{不能大於}提出申請的resource ID。
            \end{itemize}
        \end{itemize}
    \end{itemize}
\end{theorem}

\begin{theorem}{(141)} Deadlock avoidance:\begin{itemize}
        \item Banker's algorithm:Unsafe未必deadlock。
        \item 若$n$ processes,$m$ resources(單一種類),若滿足\begin{equation} \label{eq:deadlock}
            \begin{aligned}
                1 \le \ & Max_i \le m \\
                & \sum_{i = 1}^{n} Max_i < n + m
            \end{aligned}
        \end{equation} 則NO deadlock。\\ \begin{proof}
            若所有資源都分配給processes,即\begin{equation}
                \sum_{i = 1}^{n} Allocation_i = m
            \end{equation} 又\begin{equation}
                \begin{aligned}
                    & \sum_{i = 1}^{n} Need_i = \sum_{i = 1}^{n} Max_i - \sum_{i = 1}^{n} Allocation_i \\
                    \rightarrow & \sum_{i = 1}^{n} Max_i = \sum_{i = 1}^{n} Need_i + m
                \end{aligned}
            \end{equation}
            根據第二條件,有\begin{equation}
                \begin{aligned}
                    & \sum_{i = 1}^{n} Max_i < n + m \\
                    \rightarrow & \sum_{i = 1}^{n} Need_i < n
                \end{aligned}
            \end{equation}表示至少一個process $P_i$,$Need_i = 0$,又\begin{equation}
                \begin{aligned}
                    & Max_i \ge 1 \land Need_i = 0 \\
                    \rightarrow & Allocation_i \ge 1
                \end{aligned}
            \end{equation}在$P_i$完工後,會產生$\ge 1$ resources給其他processes使用,又可以使$\ge 1$ processes $P_j$有$Need_j = 0$,依此類推,所有processes皆可完工。
        \end{proof}
    \end{itemize}
\end{theorem}

\begin{theorem}{()} \quad\quad \begin{table}[H]
        \centering
        \begin{tabular}{|c|c|c|c|c|}
            \hline
            Method & Deadlock & \makecell{Utilization \&\\throughput} & Starvation & Time complexity \\
            \Xhline{2\arrayrulewidth}
            \hline
            Deadlock prevention & & Low & \multirow{3}{*}{$\surd$} & \\
            \cline{1-3}\cline{5-5}
            Deadlock avoidance & & Low & & $O(n^2 \times m)$ \\
            \cline{1-3}\cline{5-5}
            \makecell{Deadlock detection\\and recovery} & $\surd$ & High & & \makecell{$O(n^2 \times m)$\\each time} \\
            \hline
        \end{tabular}
    \end{table}
\end{theorem}
