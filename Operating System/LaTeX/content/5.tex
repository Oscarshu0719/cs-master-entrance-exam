\subsection{Deadlock}

\begin{theorem}{(136)} Deadlock:\begin{itemize}
        \item \textbf{必要}條件:\begin{itemize}
            \item Mutual exclusion:同一個時間點資源只允許一個process持有。
            \item Hold and wait:持有部分資源又在等待其他process持有的資源。
            \item No preemption:process不可任意搶奪其他process持有的資源,必須等他自願釋放,才有機會取得。
            \item Circular wait:多個processes形成循環等待的關係。
        \end{itemize}
        \item deadlock throughput低,但starvation與效能差無必然關係,例如SRTF有starvation,但scheduling效益最佳。
        \item 資源分配圖:\begin{itemize}
            \item No cycle一定no deadlock。
            \item 有cycle不一定deadlock。
            \item 若每一類resources皆single-instance,則有cycle一定deadlock。
        \end{itemize}
    \end{itemize}
\end{theorem}

\begin{theorem}{(139)} Deadlock prevention:\begin{itemize}
        \item 保證no deadlock。
        \item Utilization和throughput較低。
        \item 可能starvation。
        \item 打破必要條件:\begin{itemize}
            \item Mutual exclusion:無法破除,這是與生俱來的性質。
            \item Hold and wait:除非可以一次獲得所有資源,否則不得持有任何資源;或可持有部分資源,但申請其他資源前,需先放掉所持有的資源。
            \item No preemption:改為preemption。\begin{itemize}
                \item 須防止race condition。
                \item Resource recovery.
                \item 可能starvation。
            \end{itemize}
            \item Circular wait:\begin{itemize}
                \item 每個資源有unique resource ID。
                \item process需依照resource ID依序遞增提出申請,及持有的resource ID,不能大於提出申請的resource ID。
            \end{itemize}
        \end{itemize}
    \end{itemize}
\end{theorem}

\begin{theorem}{(141)} Deadlock avoidance:\begin{itemize}
        \item 保證no deadlock。
        \item Utilization和throughput較低。
        \item 可能starvation。
        \item Banker's algorithm:\begin{itemize}
            \item $n$ processes, $m$ resources.
            \item $Request_i$: $P_i$提出的資源申請量。
            \item $Max$:processes完成工作所需最大的各種資源數量。
            \item $Allocation$:processes所持有的各種資源數量。
            \item $Need$:processes所缺少的各種資源數量。
            \item $Available$:目前可用的各種資源數量。
            \item Procedures:\begin{itemize}
                \item Check $Request_i \le Need_i$,若不成立直接終止。
                \item Check $Request_i \le Available$,若不成立須等到資源足夠。
                \item 暫時更新狀態,執行safety algorithm。
            \end{itemize}
            \item Safety algorithm:\begin{itemize}
                \item $Work$:即$Available$。
                \item $Finish$:表示是否可完工,初始值為$False$。
                \item Procedures:Recursively check所有未完工的processes,若滿足$Finish_i = False \land Need_i \le Work$,則$Finish_i = True; \ Work \mathrel{+}= Allocation_i$;否則若$Finish$皆為$True$,返回$safe$,反之返回$unsafe$。
            \end{itemize}
            \item Time complexity: $O(n^2 \times m)$.
        \end{itemize}
        \item 若$n$ processes,$m$ resources(單一種類),若滿足\begin{equation} \label{eq:deadlock}
            \begin{aligned}
                1 \le \ & Max_i \le m \\
                & \sum_{i = 1}^{n} Max_i < n + m
            \end{aligned}
        \end{equation} 則保證不會deadlock。證明:若所有資源都分配給processes,即$\sum_{i = 1}^{n} Allocation_i = m$,因為$\sum_{i = 1}^{n} Need_i = \sum_{i = 1}^{n} Max_i - \sum_{i = 1}^{n} Allocation_i \rightarrow \sum_{i = 1}^{n} Max_i = \sum_{i = 1}^{n} Need_i + m$,
        根據第二條件,$\sum_{i = 1}^{n} Max_i < n + m$,有$\sum_{i = 1}^{n} Need_i < n$,表示至少一個process $P_i$,$Need_i = 0$,因為$Max_i \ge 1 \land Need_i = 0 \rightarrow Allocation_i \ge 1$,在$P_i$完工後,會產生$\ge 1$ resources給其他processes使用,又可以使$\ge 1$ processes $P_j$有$Need_j = 0$,依此類推,所有processes皆可完工。
        \item 若所有類型資源皆為single-instance,可用畫圖簡化procedure。\begin{itemize}
            \item claim edge:未來process $P_i$會對resource $R_j$提出申請。
            \item Procedures:\begin{itemize}
                \item 檢查是否有claim edge,若無終止$P_i$。
                \item 檢查$R_j$是否available,若不成立須等到資源足夠。
                \item 暫時將claim edge改為allocation edge,檢查是否有cycle存在,若無表示safe;若有則unsafe,並改回claim edge。
            \end{itemize}
        \end{itemize}
        \item Unsafe未必deadlock。
    \end{itemize}
\end{theorem}

\begin{theorem}{(145, 148)} Deadlock detection and recovery:\begin{itemize}
        \item 不保證no deadlock。
        \item Utilization和throughput較高。
        \item 偵測系統\textbf{目前}是否deadlock,若有則recovery。
        \item Deadlock detection algorithm:\begin{itemize}
            \item $n$ processes,$m$ resources. 
            \item $Allocation$, $Request$, $Available$, $Work$, $Finish$
            \item Procedures:\begin{itemize}
                \item 設定初始值:\begin{equation}
                    \begin{aligned}
                        Work & = Available \\
                        Finish_i & = \begin{cases}
                            False &, Allocation_i \neq 0 \\
                            True &, Allocation_i = 0
                        \end{cases}
                    \end{aligned}
                \end{equation}
                \item Recursively check $Finish_i = False \land Request_i \le Work$,則$Finish_i = True; Work \mathrel{+}= Allocation_i$;否則 check $Finish$ 皆為 $True$,則no deadlock,否則$Finish_i = False$,$P_i$ deadlock。
            \end{itemize}
            \item Detection一次 time complexity $O(n^2 \times m)$,另位還要考慮偵測頻率,因此cost極高。
        \end{itemize}
        \item 若所有類型資源皆為single-instance,可用畫圖簡化procedure。\begin{itemize}
            \item 使用Wait-for graph,省略resources。
            \item No cycle, no deadlock; 有cycles,deadlock。
        \end{itemize}
        \item Recovery:\begin{itemize}
            \item Process termination:\begin{itemize}
                \item Abort all deadlocked processes:cost高,先前工作白費。
                \item Abort one process at a time until deadlock is eliminated:每終止一個deadlocked process,都要執行一次deadlock detection alogrithm,cost high。
            \end{itemize}
            \item Resource preemption:\begin{itemize}
                \item Selecting a victim:Minimizing cost,cost高,注意starvation。
                \item Rolling back:Difficult, returning to safe state.
            \end{itemize}
        \end{itemize}
    \end{itemize}
\end{theorem}
