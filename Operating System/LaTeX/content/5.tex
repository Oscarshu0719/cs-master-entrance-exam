\begin{theorem}{(141)} Deadlock avoidance:\begin{itemize}
        \item 若$n$ processes,$m$ resources(單一種類),若滿足\begin{equation} \label{eq:deadlock}
            \begin{aligned}
                1 \le \ & Max_i \le m \\
                & \sum_{i = 1}^{n} Max_i < n + m
            \end{aligned}
        \end{equation} 則NO deadlock。\\ \begin{proof}
            若所有資源都分配給processes,即\begin{equation}
                \sum_{i = 1}^{n} Allocation_i = m
            \end{equation} 又\begin{equation}
                \begin{aligned}
                    & \sum_{i = 1}^{n} Need_i = \sum_{i = 1}^{n} Max_i - \sum_{i = 1}^{n} Allocation_i \\
                    \rightarrow & \sum_{i = 1}^{n} Max_i = \sum_{i = 1}^{n} Need_i + m
                \end{aligned}
            \end{equation}
            根據第二條件,有\begin{equation}
                \begin{aligned}
                    & \sum_{i = 1}^{n} Max_i < n + m \\
                    \rightarrow & \sum_{i = 1}^{n} Need_i < n
                \end{aligned}
            \end{equation}$\exists$ process $P_i$,$Need_i = 0$,又\begin{equation}
                \begin{aligned}
                    & Max_i \ge 1 \land Need_i = 0 \\
                    \rightarrow & Allocation_i \ge 1
                \end{aligned}
            \end{equation}在$P_i$完工後,會產生$\ge 1$ resources給其他processes使用,又可以使$\ge 1$ processes $P_j$有$Need_j = 0$,依此類推,所有processes皆可完工。
        \end{proof}
    \end{itemize}
\end{theorem}
