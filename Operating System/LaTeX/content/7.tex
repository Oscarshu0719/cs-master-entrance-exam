\begin{theorem}{(223)} \quad\quad \begin{itemize}
        \item Dynamic loading:不需要OS協助,programmer責任。
        \item Dynamic linking:需要OS協助,因為processes間無法access彼此的memory。
    \end{itemize}
\end{theorem}

\begin{theorem}{(229, 230, 238)} Contiguous allocation:\begin{itemize}
        \item External fragmentation:\begin{itemize}
            \item Compaction:\begin{itemize}
                \item 不易制定optimal policy。
                \item process必須是dynamic binding才行在execution time移動memory blocks。
            \end{itemize}
            \item Page memory management:\begin{itemize}
                \item 但有internal fragmentation problem,page size越大越嚴重。
                \item Support memory sharing, memory protection, dynamic loading and dynamic linking.
                \item Effective Memory Access Time (EMAT) 較久。
                \item \begin{equation}
                    \begin{aligned}
                        \text{EMAT} & = p \times (\text{TLB time} + \text{Memory access time}) \\
                        & + (1 - p) \times (\text{TLB time} + 2 \times \text{Memory access time})
                    \end{aligned}
                \end{equation} $p$ is TLB hit rate.
            \end{itemize}
        \end{itemize}
        \item Internal fragmentation:\begin{itemize}
            \item Segment memory management:\begin{itemize}
                \item 但有external fragmentation problem。
                \item Support memory sharing and memory protection, 但比paging容易實施。
                \item EMAT更久,因為多了checking $offset < limit$。
            \end{itemize}
        \end{itemize}
        \item Paged segment memory management:\begin{itemize}
            \item 先segment再page。
            \item Segment table紀錄limit和page table address,且每個segment都有各自page table。
            \item 解決external fragmentation problem,但有internal fragmentation problem。
            \item EMAT最長,很佔空間。
        \end{itemize}
    \end{itemize}
\end{theorem}
