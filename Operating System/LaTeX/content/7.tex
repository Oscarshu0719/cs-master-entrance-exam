\subsection{Memory management}

\begin{theorem}{(222)} Binding:\begin{itemize}
        \item 決定在process執行在memory起始address。
        \item 時間點:\begin{itemize}
            \item Compiling time by compiler. 通常只有OS commands採用,e.g. COM file。
            \item Loading/Linking time by linking loader.
            \item Execution time by OS, so called dynamic binding. MMU (Memory Management Unit):Translation from logical (virtual) to physical address.
        \end{itemize}
    \end{itemize}
\end{theorem}

\begin{theorem}{(223)} Dynamic loading:\begin{itemize}
        \item Load-on-call,execution time需要才load。
        \item Process execution time較久,且牽扯到I/O。
        \item 不需要OS協助,programmer責任。
    \end{itemize}
\end{theorem}

\begin{theorem}{(223)} Dynamic linking:\begin{itemize}
        \item Modules code在execution time需要才link。
        \item 節省object code space。
        \item Shared library,助於library更新。
        \item 需要OS協助,因為processes間無法access彼此的memory。
    \end{itemize}
\end{theorem}

\begin{theorem}{(229, 230)} Contiguous allocation:\begin{itemize}
        \item 被占用的space也稱partition,通常是variable,也稱variable partition。Partititon數量也是process數量,即multiprogramming degree。
        \item Process利用linked list管理free memory blocks (holes),稱作Available list (AV-list)。
        \item External fragmentation:\begin{itemize}
            \item Free memory blocks總和夠大,但各自不夠。
            \item Compaction:移動執行中的process,使不連續memory blocks變連續。不易制定optimal policy,且process必須是dynamic binding才行在execution time移動memory blocks。
            \item Page memory management:
        \end{itemize}
        \item Internal fragmentation:配給process的space過多。
    \end{itemize}
\end{theorem}
