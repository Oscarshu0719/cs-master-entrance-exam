\begin{theorem}{(242)} Virtual memory:\begin{itemize}
        \item Less I/O time,但總體I/O time上升,因為次數提升。
        \item \begin{equation}
                \text{EMAT} = (1 - p) \times \text{Memory Access time} + p \times \text{Page fault time}
        \end{equation} $p$ is page fault rate.
    \end{itemize}
\end{theorem}

\begin{theorem}{(246)} Page replacement:\begin{itemize}
        \item LRU近似:\begin{itemize}
            \item Second chance (Clock):先用FIFO挑出一個page,若同時reference bit為$0$,則為victim page,但若為$1$,則reset reference bit,loading time改為現在時間,重新FIFO找page。
            \item Enchanced second chance:選擇$<reference, dirty>$最小者,若多個pages相同,則採用FIFO。
        \end{itemize}
        \item 所有replacement algorithms沒有最差,只有最佳。
        \item Belady amonaly:分給process的frames增加,但page fualt rate不降反升。
        \item Stack property:$n$ frames所包含的page set必定是$n + 1$ frames所包含的page set的子集合。且具有stack property的法則,不會發生belady anomaly。只有\textbf{OPT}和\textbf{LRU}有。
        \item Page buffering:\begin{itemize}
            \item Free frames pool:將frames分為\begin{itemize}
                \item Resident frames分配給process。
                \item Free frames pool,OS keep,讓miss pages先行載入,process即可恢復執行,且加入resident frames,完成後再將victim page write back to disk,並歸還free frames pool。
            \end{itemize} 
            \item Keep modification list紀錄dirty bit為$1$的所有page no.,等到OS空閒再將list中的pages write back to disk and reset dirty bits。
        \end{itemize}
    \end{itemize}
\end{theorem}

\begin{theorem}{(253)} Frame allocation:\begin{itemize}
        \item Process可分配frames數量由hardware決定,最多為physical memory size,最少須讓任一machine code完成,即週期中最多可能memory access數量,e.g. $IF$, $MEM$, $WB$共三次。
        \item 解決thrashing:\begin{itemize}
            \item 若已發生,只能降低multiprogramming degree。
            \item 利用page fault frequency control防止thrashing,OS訂定page fault rate合理的上下限,thrashing理當不會發生。若大於上限,應該多分配frames;若小於下限,應該取走一些frames。
        \end{itemize}
        \item Working-set model:\begin{itemize}
            \item 若符合locality,則可降低page fault rate;違反locality:linked list, hashing, binary search, jump, indirect addressing mode。
            \item 可預防thrashing,對於prepaging也有益。
        \end{itemize}
        \item \begin{itemize}
            \item Bigger paging disk沒幫助。
            \item \textbf{Faster paging disk有幫助},因為decrease page fault time。
            \item \textbf{Increase page size有幫助}。
            \item Decrease page size沒幫助。
            \item \textbf{Local repalcement有幫助},因為thrashing不至於擴散。
            \item \textbf{Prepaging有幫助},若猜測夠準,decrease page fault rate。
        \end{itemize}
    \end{itemize}
\end{theorem}

\begin{theorem}{(258)} Page size越小:\begin{itemize}
        \item Page fault rate增加。
        \item Page table size增加。
        \item I/O次數增加。
        \item Internal fragmentation輕微。
        \item I/O transfer time較小。
        \item Locality較佳。
        \item 趨勢:大page size。
    \end{itemize} 
\end{theorem}

\begin{theorem}{(261)} Copy-on-write:\begin{itemize}
        \item \code{fork()} without Copy-on-write:Child and parent processes占用不同空間,且複製parent process content給child process,大幅增加memory space需求,且process creation較慢。
        \item \code{fork()} with Copy-on-write:Child process共享parent process memory space,不allocate new frames,降低memory space需求,且不須複製parent process content,process creation較快,但在write時,
        allocate new frame給child process,並複製內容,修改child process的page table指向new frame,最後才write。
        \item \code{vfork()} (Virtual memory \code{fork()}):Child process共享parent process memory space,不allocate new frames,但不提供Copy-on-write,因此在write時,另一方會受影響。適用於child process create後馬上\code{execlp()},e.g. UNIX shell command interpreter。
    \end{itemize}
\end{theorem}
