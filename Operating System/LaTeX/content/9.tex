\begin{theorem}{()} Seek time:Head移到track time,通常最長。Latency (Rotation) time:Head移到sector time。
\end{theorem}

\begin{theorem}{(302)} Disk free space management:\begin{itemize}
        \item Grouping:Linked list,同時記錄更多free blocks的no.。
        \item Counting:Linked list,適用於contiguous allocation and free,同時記錄接著的contiguous free blocks。
    \end{itemize}
\end{theorem}

\begin{theorem}{(304)} File allocation methods:\begin{itemize}
        \item Physical directory紀錄。
        \item Contiguous allocation:\begin{itemize}
            \item Seek time較短,因為多數落在同一個cylinder。
            \item Support random (direct) access和sequential access且後者較linked allocation快。
            \item Reliablity較linked allocation高。
            \item External fragmentation problem,disk中通過repack解決,類似compaction。
            \item 檔案大小不易擴增,需事先宣告大小。
        \end{itemize}
        \item Linked allocation:\begin{itemize}
            \item 優缺點與contiguous allocation相反。
            \item 無法support random access。
            \item FAT (File Allocation Table):將linking info存在FAT (memory) 中,不存在allocation disk,加速access。
        \end{itemize}
        \item Index allocation:\begin{itemize}
            \item 不一定連續,需額外allocate index block記錄所有allocated blocks no.。
            \item Support random access和sequential access。
            \item 可動態擴充,不事先宣告大小。
            \item Linking info比linked allocation大很多。
            \item 若檔案大到一個index block無法紀錄。解決:\begin{itemize}
                \item 多個index blocks用linking串連,I/O次數大增。
                \item Multilevel index structure:I/O次數固定,但對小型檔案極不合適,index block甚至比data block還多。
                \item UNIX的I-Node:對所有大小檔案皆適合。
            \end{itemize}
        \end{itemize}
        \item Internal fragmentation problem所有allocation皆有,可以視而不見。
    \end{itemize}
\end{theorem}

\begin{theorem}{(309)} Disk scheduling:\begin{itemize}
        \item FCFS:公平,no starvation,效果不好。
        \item SSTF (Shortest Seek-time Track First):不公平,可能starvation,效果不錯。
        \item SCAN:效果尚可,適用大量tracks request,獲得較均勻等待時間,不盡公平,但NO starvation。
        \item C-SCAN (Circular SCAN):只提供單向服務,回程不服務。
        \item Look:類似SCAN,服務完該方向最後一個就折返。
        \item C-Look:只提供單向服務的Look。
        \item 無最差也無最佳。
    \end{itemize}
\end{theorem}

\begin{theorem}{(313)} \quad\quad \begin{itemize}
        \item Raw I/O:將disk視為大型,不支援file system service,performance佳,但user不方便使用。
        \item Bootstrap loader:只用在開機時將OS的object code從disk載入RAM,放在disk中固定的boot blocks。
    \end{itemize}
\end{theorem}

\begin{theorem}{(314)} Bad sectors處理:\begin{itemize}
        \item Spare sectors (sector sparing, forwarding):\begin{itemize}
            \item e.g. SCSI.
            \item 在low-level formatting完成,OS看不到。
            \item 破壞disk scheduling效能。解法:把spare sectors分散到每個cyclinder上,bad sectors利用鄰近的spare sector。
        \end{itemize}
        \item Sector slipping:移動其他sectors,空出下一個sector取代bad sector。
    \end{itemize}
\end{theorem}
