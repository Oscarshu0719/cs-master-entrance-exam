\begin{theorem}{()} Storage: \begin{itemize}
        \item Smartphones normally do \textbf{NOT} have HDDs.
        \item Secondary storage is normally \textbf{non-volatile}.
        \item Wearable devices are normally equipped with \textbf{hard disks} to increase its storage space.
    \end{itemize}
\end{theorem}

\begin{theorem}{()} Disk: \begin{itemize}
        \item \textbf{High-level} formatting creates a file system on a disk partition.
        \item A disk sector contains a header, a data area, and a trailer.
        \item In UNIX, disk scheduling algorithm is performed in the \textbf{disk driver}.
        \item A file system can be created across \textbf{multiple disk partitions}.
        \item \textbf{Disk device driver} can \textbf{NOT} be paged out, but page tables, memory-mapped files, shared memory can.
        \item Moving files between directories on the \textbf{same} disk partition and \textbf{deleting} files on a hard disk cause little overhead, but moving files between directories on \textbf{different} disk partitions cause much.
        \item The variation of disk I/O \textbf{latencies} under SSTF can be very high.
    \end{itemize}
\end{theorem}

\begin{theorem}{()} Cybersecurity: \begin{itemize}
        \item Trojan Horse is a code segment that \textbf{misuses} its environment.
        \item Installing antivirus software is \textbf{NOT} an example of least privileges.
        \item Many routers are equipped with \textbf{firewall} and \textbf{VPN} functions.
        \item Via HTTPS, ISPs can know the browsing website, but can \textbf{NOT} know the content.
    \end{itemize}
\end{theorem}

\begin{theorem}{()} Cryptography: \begin{itemize}
        \item Public-key (asymmetric) cryptography提供digital signature功能。
        \item AES:Symmetric, block cipher.
        \item DES:Symmetric, block cipher.
        \item RC4:Symmetric, stream cipher.
        \item RSA:Asymmetric,只要鑰匙夠長,沒有任何可靠的攻擊方法。\begin{itemize}
            \item Authentication:將message與hash過再用private key加密的message串接。e.g. $M || \{h(M)\}_{K_{sa}}$.
            \item Confidentiality:將用one-time AES key加密的message與用public key加密的one-time AES key串接。e.g. $\{M\}_{K_{da}} || \{K_{da}\}_{K_{pb}}$.
            \item Confidentiality and authentication:將authentication的內容用one-time AES key加密,再與用public key加密的one-time AES key串接。e.g. $\{M || \{h(M)\}_{K_{sa}}\}_{K_{da}} || \{K_{da}\}_{K_{pb}}$.
        \end{itemize}
        \item Digital certificate contains \textbf{private key} signed by the user.
    \end{itemize}
\end{theorem}

\begin{theorem}{()} Kernel:\begin{itemize}
        \item Monolithic:UNIX, UNIX-like, Windows 9x, Android.
        \item Microkernel:Mach.
        \item Hybrid:Windows NT, Windows XP, macOS.
        \item Kernel processes are \textbf{NOT} allocated through paging and virtual memory interface.
        \item A \textbf{non-preemptive} kernel is free from race conditions on kernel data structures.
        \item \textbf{Preemptive} kernel design can \textbf{NOT} prevent the deadlock problem with kernel data structures from occurring in the kernel.
        \item Linux kernel is a \textbf{preemptive} kernel and a process running in a kernel mode could \textbf{NOT} be preempted.
    \end{itemize}
\end{theorem}

\begin{theorem}{()} UID:\begin{itemize}
        \item Real UID: identify the real owner of the process and affect the permissions for sending signals.
        \item Effective UID: used for most access checks, including creating and accessing to a file.
        \item Saved UID: used when a program running with elevated privileges needs to do some unprivileged work temporarily.
    \end{itemize}
\end{theorem}

\begin{theorem}{()} I/O:\begin{itemize}
        \item Buffered I/O: Read one block to cache when R/W, then copy from cache and return to reduce number of system call. Totally 2 copy operations.
        \item Unbuffered I/O: Directly transfer from disk without caching. Caching is conducted by the application. Number of copy operations is determined by the transfering method, and it's only 1 copy operation for block-transfering. 
        \item A program using asynchronous I/O system calls in \textbf{NOT} simpler to write than using synchronous I/O system calls.
    \end{itemize}
\end{theorem}

\begin{theorem}{()} File system: \begin{itemize}
        \item devfs:\textbf{Virtual} fs。一個file一個device,但該device未必存在,\textbf{不確定device mapping}。
        \item sysfs:\textbf{Virutal} fs。將real connected devices組織成\textbf{分階層}的file directory,每個device有\textbf{唯一}對應的directory。
        \item Device tree:每個node用key對應value方式紀錄device properties,其中value可為空。
    \end{itemize}
\end{theorem}

\begin{theorem}{()} GCD (Grand Central Dispatch): \begin{itemize}
        \item 自動利用更多CPU cores。
        \item 自動管理thread life cycles。
        \item Move thread pool out of hand of developers and closer to OS.
        \item Dispatch tasks時,可分在相同或不同queues,分別稱作serial和concurrent。Queues間可分為sync和async,前者同時間只允許一個queue執行,後者允許多個queues執行。
    \end{itemize}
\end{theorem}

\begin{theorem}{()} Container: \begin{itemize}
        \item 所有containers共用host OS。
        \item 相較VM,不須打包OS就能執行,速度較快且空間小。
    \end{itemize}
\end{theorem}

\begin{theorem}{()} MBR, BIOS, GPT, UEFI: \begin{itemize}
        \item BIOS無法辨識GPT(GUID Partition Table)。
        \item UEFI用來定義OS和firmware間的software interface。
        \item UEFI是用模組化,動態連結的形式構建的系統,較BIOS而言更易於實現,容錯和糾錯特性更強,縮短了系統研發的時間。
        \item UEFI(Unified Extensible Firmware Interface)預啟動時就load OS,且可以同時識別MBR和GPT。
        \item GPT使用LBA(Logical Block Address)取代早期CHS(Cylinder-head-sector)定址方式。
        \item GPT的分割區表的位置資訊儲存在GPT header中,但第一個磁區仍然用作MBR,之後才是GPT header。
    \end{itemize}
\end{theorem}

\begin{theorem}{()} Thread: \begin{itemize}
        \item \textbf{Native Windows threads} cause a user-mode to kernel-mode.
        \item Hyper-threading is \textbf{superscalar} and it can speedup \textbf{context switching}.
        \item Each thread of the program receives a \textbf{larger} CPU time with \textbf{many-to-one} thread model.
        \item Most operating systems \textbf{downgrade} the thread priority when it runs out of time quantum, but \textbf{boost} the priority when it returns from an I/O request.
    \end{itemize}
\end{theorem}

\begin{theorem}{()} Cache: \begin{itemize}
        \item Physical caches do NOT flush at \textbf{context switching}.
        \item The TLB cache may require a flush after a page table update.
        \item Cache memories are usually hardware controlled, and OS may \textbf{NOT} even need to know their existence.
    \end{itemize}
\end{theorem}

\begin{theorem}{()} Allocation: \begin{itemize}
        \item There is \textbf{NO} optimum solution to allocate contiguous memory from free holes.
        \item Extent allocation uses \textbf{contiguous physical} blocks, and it also needs defragmentation.
        \item Contiguous allocation offers the best R/W performance for \textbf{large} files.
    \end{itemize}
\end{theorem}

\begin{theorem}{()} Page table: \begin{itemize}
        \item (\textbf{FALSE}) Use of shared memory can reduce the number of page table entries.
        \item (\textbf{FALSE}) The page table of Linux process is managed by the C runtime library (.so) in the process.
        \item For the \textbf{unused regions} in the virtual address space, the space overhead of the corresponding \textbf{page table entries} can be negligible.
    \end{itemize}
\end{theorem}

\begin{theorem}{()} CPU scheduling: \begin{itemize}
        \item FIFO can outperform LRU.
        \item FIFO may have Convoy effect, which causes low \textbf{I/O} utilization.
        \item After making system calls, the process is still in running state.
        \item (\textbf{FALSE}) In a time-sharing system, a process does \textbf{NOT} leave running state unless it terminates or is preempted through a timer interrupt. 
    \end{itemize}
\end{theorem}

\begin{theorem}{()} Synchronization: \begin{itemize}
        \item \textsc{Test-and-Set} still wastes cycles when a process can \textbf{NOT} acquire a lock. 
        \item To use shared memory, several system calls have to be invoked.
        \item \textsc{Test-and-Set} can be implementated in \textbf{user space}, provided that the lock variable is in a shared memory region.
        \item \textbf{Two-phase locking protocol (2PL)} ensures \textbf{conflict serializability}, but it may result in \textbf{deadlock}.
        \item OS does \textbf{NOT} need to estimate $MAX$ when a process enters ready queue.
    \end{itemize}
\end{theorem}

\begin{theorem}{()} \quad\quad \begin{itemize}
        \item Data fault: Access invalid data memory, which is signaled by \textbf{MMU}.
        \item NUMA is intrinsic in Von Neumann's computer model.
        \item \code{kmalloc}: physically contiguous; \code{vmalloc}: virtually contiguous; \code{malloc}: no constraints.
        \item \code{strncpy}相較\code{strcpy}安全,且需要\textbf{預留一格},可防止buffer overflow。
        \item Java \textbf{interprets} Java bytecode operations \textbf{one at a time}.
        \item CLR, which is the implementation of .NET VM, \textbf{compiles} Microsoft intermediate language instructions \textbf{one at a time}.
        \item Normal instructions for the VM can execute \textbf{directly on the hardware} and \textbf{only the privileged instructions} must be simulated.
        \item Named pipes are referred to as \textbf{FIFOs} in UNIX systems. Once created, they appear as typical \textbf{files} in the file systems. 
        \item Permission bits are stored at \textbf{inodes}.
        \item Five classic components: datapath, control unit, memory, input, and output.
        \item Data center cares more about \textbf{throughput} than response time.
        \item Memory blocks on the \textbf{stacks} can \textbf{NOT} be freed at any time, but \textbf{heaps} can.
        \item \textbf{Stack} is good for locality.
        \item (\textbf{FALSE}) Programs written in different assmebly languages can ONLY be executed on specific hardware.
        \item Computer system can be divided into four components including hardware, OS, application programs, and users.  
        \item Normal instructions for the virtual machines can execute directly on the hardware and ONLY the privileged instructions must be simulated.
        \item Bitmap is NOT a file.
        \item data section存global和static variables。
        \item When the block size is very large, the \textbf{spatial locality} within the block is lower.
    \end{itemize}
\end{theorem}
