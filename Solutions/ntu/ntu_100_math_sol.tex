\documentclass[a4paper,12pt]{article}

\usepackage{../header}
\newcommand{\school}{ntu}
\newcommand{\subject}{math}
\renewcommand{\year}{100}
\newcommand{\titlename}{\MakeUppercase{\school} \subject \ \year}

\fancypagestyle{mainmatter}{\rhead{\titlename}}
\pagestyle{mainmatter}
\CenterWallPaper{.50}{img/logo_ntu_recolor.jpg}
\newcommand{\ver}{\textsc{Version} 1.0} % Version number.

\begin{document}

\title{\LARGE{\textbf{Solutions}} \\
	\Huge{\textbf{\titlename}} \\
	\normalsize{\ver}
}
\author{}
\date{}

\maketitle

% Start of solutions.

\begin{enumerate}
	\item \begin{answer}{$\dag$}\begin{equation}
            1
        \end{equation}
    \end{answer}
    \item \begin{answer}{$\dag$}\begin{equation}
            \begin{bmatrix}
                5 \\
                1 \\
                3
            \end{bmatrix}
        \end{equation}
    \end{answer}
    \item We have \begin{equation}
        \mat{B}^3 = \begin{bmatrix}
            17 & 6 \\
            18 & -1
        \end{bmatrix} = 6 \times \mat{B} + 5 \times \mat{I}
    \end{equation}
    \begin{answer}{$\dag$}\begin{equation}
            (6, \ 5)
        \end{equation}
    \end{answer}
    \item We have \begin{equation}
        \det\left(\begin{bmatrix}
            1 & 0 & 0 \\
            1 & 1 & 0 \\
            2 & 3 & 2
        \end{bmatrix})\right) = 2, \ \det\left(\begin{bmatrix}
            1 & 5 & 1 \\
            1 & 1 & 2 \\
            -2 & 1 & 3
        \end{bmatrix})\right) = -31, \ \det\left(\begin{bmatrix}
            1 & 4 & 1 \\
            5 & 2 & 0 \\
            1 & 0 & 0
        \end{bmatrix})\right) = -2
    \end{equation}
    \begin{answer}{$\dag$}\begin{equation}
            2 \times \frac{1}{-31} \times (-2) = \frac{4}{31}
        \end{equation}
    \end{answer}
    \item We have \begin{equation}
        \tr(\mat{A}^2) = (1 + (-2) + 3) + ((-2) + 9 + a) + (3 + a + 0) = 5
    \end{equation}
    \begin{answer}{$\dag$}\begin{equation}
            -\frac{7}{2}
        \end{equation}
    \end{answer}
    \item We have \begin{equation}
        \begin{aligned}
            \mat{A}\vec{w} & = \vec{w} + \alpha\vec{w}\vec{w}^\intercal\vec{w} \\
            & = \vec{w} + 10 \times \alpha \vec{w} \\
            & = (10 \times \alpha + 1)\vec{w} \\
            \Rightarrow & \ 10 \times \alpha + 1 = 0 (\because \mat{A} \ \text{is singular.})
        \end{aligned}
    \end{equation} And, we have \begin{equation}
        \rnk(\vec{w}\vec{w}^\intercal) = 1
    \end{equation} Then, we have eigenvalues of $\vec{w}\vec{w}^\intercal$ \begin{equation}
        0, \ 0, \ 0, \ 0, \ 10
    \end{equation} Then, we have eigenvalues of $\mat{A}$ \begin{equation}
        1, \ 1, \ 1, \ 1, \ 0 \ (\because \ 1 + 10 \times \alpha)
    \end{equation} Then, we have $\rnk(\mat{A}) = 4$.
    \begin{answer}{$\dag$}\begin{equation}
            (-\frac{1}{10}, \ 4)
        \end{equation}
    \end{answer}
    \item Suppose \begin{equation}
        \begin{aligned}
            \mat{D} & = \begin{bmatrix}
                \alpha & 0 & 0 \\
                0 & \beta & 0 \\
                0 & 0 & \gamma
            \end{bmatrix} \\
            \Rightarrow \mat{D}^{-1} & = \begin{bmatrix}
                \frac{1}{\alpha} & 0 & 0 \\
                0 & \frac{1}{\beta} & 0 \\
                0 & 0 & \frac{1}{\gamma}
            \end{bmatrix}
        \end{aligned}
    \end{equation} Then, we have \begin{equation}
        \mat{D}^{-1}\mat{A}\mat{D} = \begin{bmatrix}
            \frac{1}{\alpha} & 0 & 0 \\
            0 & \frac{1}{\beta} & 0 \\
            0 & 0 & \frac{1}{\gamma}
        \end{bmatrix} \begin{bmatrix}
            a_{11} & a_{12} & a_{13} \\
            a_{21} & a_{22} & a_{23} \\
            a_{31} & a_{32} & a_{33} 
        \end{bmatrix} \begin{bmatrix}
            \alpha & 0 & 0 \\
            0 & \beta & 0 \\
            0 & 0 & \gamma
        \end{bmatrix} = \begin{bmatrix}
            a_{11} & \frac{1}{2} \times a_{12} & \frac{1}{4} \times a_{13} \\
            2 \times a_{21} & a_{22} & \frac{1}{2} \times a_{23} \\
            4 \times a_{31} & 2 \times a_{32} & a_{33}
        \end{bmatrix}
    \end{equation}
    \begin{answer}{$\dag$}\begin{equation}
            \begin{bmatrix}
                4 \times \alpha & 0 & 0 \\
                0 & 2 \times \alpha & 0 \\
                0 & 0 & \alpha
            \end{bmatrix}, \ \forall \ \alpha \in \R
        \end{equation}
    \end{answer}
    \item Let \begin{equation}
        \begin{aligned}
            & \mat{A} = \mat{I} + \begin{bmatrix}
                \vec{u} & \vec{v}
            \end{bmatrix} \begin{bmatrix}
                \vec{u} \\
                \vec{v}
            \end{bmatrix} \\
            \Rightarrow & \ \mat{A} = \mat{I} + \mat{X}\mat{Y}
        \end{aligned}
    \end{equation} Since, $\mat{XY} = \mat{YX}$ have same eigenvalues, we have \begin{equation}
        \mat{YX} = \begin{bmatrix}
            35 & 55 \\
            55 & 35
        \end{bmatrix}
    \end{equation} Then, $\mat{YX}$ has eigenvalues $-20, \ 90$, so $\mat{XY}$ has eigenvalues \begin{equation}
        0, \ 0, \ 0, \ -20, \ 90
    \end{equation} Then, $\mat{A}$ has eigenvalues \begin{equation}
        1 + 0, \ 1 + 0, \ 1 + 0, \ 1 + (-20), \ 1 + 90
    \end{equation}
    \begin{answer}{$\dag$}\begin{equation}
            1, \ 1, \ 1, \ -19, \ 91
        \end{equation}
    \end{answer}
    \item We have \begin{flalign}
        & (\because \ r_{n(n - 1)}^{-1}, \ r_{n(n - 2)}^{-1}, \ \cdots, \ r_{n1}^{-1})& \nonumber \\
        & = \begin{vmatrix}
            \frac{n - 1}{2 \times (n + 1)} & \frac{n - 1}{3 \times (n + 2)} & \cdots & \frac{n - 1}{2n \times (n + 1)} \\
            \frac{n - 2}{3 \times (n + 1)} & \frac{n - 2}{4 \times (n + 2)} & \cdots & \frac{n - 2}{2n \times (n + 2)} \\
            \vdots & \vdots & \ddots & \vdots \\
            \frac{1}{n \times (n + 1)} & \frac{1}{(n + 1) \times (n + 2)} & \cdots & \frac{1}{2n \times (2n - 1)} \\
            \frac{1}{n + 1} & \frac{1}{n + 2} & \cdots & \frac{1}{2n} \\
        \end{vmatrix}_{n \times n}& \nonumber \\
        & (\because \ r_{1}^{\frac{1}{n - 1}}, \ r_{2}^{\frac{1}{n - 2}}, \ \cdots, \ r_{n - 2}^{\frac{1}{2}}, \ c_{1}^{n + 1}, \ c_{2}^{n + 2}, \ \cdots, \ c_{n}^{2n})& \nonumber \\
        & = (\displaystyle\prod_{i = 1}^{n - 1} i)(\displaystyle\prod_{j = 1}^{n} \frac{1}{n + j})\begin{vmatrix}
            \frac{1}{2} & \frac{1}{3} & \cdots & \frac{1}{n + 1} \\
            \frac{1}{3} & \frac{1}{4} & \cdots & \frac{1}{n + 2} \\
            \vdots & \vdots & \ddots & \vdots \\
            \frac{1}{n} & \frac{1}{n + 1} & \cdots & \frac{1}{2n - 1} \\
            1 & 1 & \cdots & 1 \\
        \end{vmatrix}_{n \times n}& \nonumber \\
        & (\because \ c_{n(n - 1)}^{-1}, \ c_{n(n - 2)}^{-1}, \ \cdots, \ c_{n1}^{-1})& \nonumber \\
        & = \frac{n!(n - 1)!}{(2n)!}\begin{vmatrix}
            \frac{n - 1}{2 \times (n + 1)} & \frac{n - 2}{3 \times (n + 1)} & \cdots & \frac{1}{n \times (n + 1)} & \frac{1}{n + 1} \\
            \frac{n - 1}{3 \times (n + 2)} & \frac{n - 2}{4 \times (n + 2)} & \cdots & \frac{1}{(n + 1) \times (n + 2)} & \frac{1}{n + 2} \\
            \vdots & \vdots & \ddots & \vdots & \vdots \\
            \frac{n - 1}{n \times (2n - 1)} & \frac{n - 2}{(n + 1) \times (2n - 1)} & \cdots & \frac{1}{(2n - 2) \times (2n - 1)} & \frac{1}{2n - 1} \\
            0 & 0 & \cdots & 0 & 1 \\
        \end{vmatrix}_{n \times n}& \\
        & (\because \ c_{1}^{\frac{1}{n - 1}}, \ c_{2}^{\frac{1}{n - 2}}, \ \cdots, \ r_{n - 2}^{\frac{1}{2}}, \ r_{1}^{n + 1}, \ r_{2}^{n + 2}, \ \cdots, \ r_{n - 1}^{2n - 1})& \nonumber \\
        & = \frac{n!(n - 1)!}{(2n)!}(\displaystyle\prod_{i = 1}^{n - 1} i)(\displaystyle\prod_{j = 1}^{n} \frac{1}{n + j})\begin{vmatrix}
            \frac{1}{2} & \frac{1}{3} & \cdots & \frac{1}{n} & 1 \\
            \frac{1}{3} & \frac{1}{4} & \cdots & \frac{1}{n + 1} & 1 \\
            \vdots & \vdots & \ddots & \vdots & \vdots \\
            \frac{1}{n} & \frac{1}{n + 1} & \cdots & \frac{1}{2n - 2} & 1 \\
            0 & 0 & \cdots & 0 & 1 \\
        \end{vmatrix}_{(n - 1) \times (n - 1)}& \nonumber \\
        & = \frac{n!(n - 1)!}{(2n)!}(\displaystyle\prod_{i = 1}^{n - 1} i)(\displaystyle\prod_{j = 1}^{n} \frac{1}{n + j})\begin{vmatrix}
            \frac{1}{2} & \frac{1}{3} & \cdots & \frac{1}{n} \\
            \frac{1}{3} & \frac{1}{4} & \cdots & \frac{1}{n + 1}\\
            \vdots & \vdots & \ddots & \vdots \\
            \frac{1}{n} & \frac{1}{n + 1} & \cdots & \frac{1}{2n - 2}
        \end{vmatrix}_{(n - 1) \times (n - 1)}& \nonumber \\
        & = \frac{(n!)^2[(n - 1)!]^2 \times 2n}{[(2n)!]^2} \times P_{n - 1}& \nonumber
    \end{flalign}
    % \begin{equation}
    %     \begin{aligned}
    %         & \overset{r_{n(n - 1)}^{-1}, \ r_{n(n - 2)}^{-1}, \ \cdots, \ r_{n1}^{-1}}= \begin{vmatrix}
    %             \frac{n - 1}{2 \times (n + 1)} & \frac{n - 1}{3 \times (n + 2)} & \cdots & \frac{n - 1}{2n \times (n + 1)} \\
    %             \frac{n - 2}{3 \times (n + 1)} & \frac{n - 2}{4 \times (n + 2)} & \cdots & \frac{n - 2}{2n \times (n + 2)} \\
    %             \vdots & \vdots & \ddots & \vdots \\
    %             \frac{1}{n \times (n + 1)} & \frac{1}{(n + 1) \times (n + 2)} & \cdots & \frac{1}{2n \times (2n - 1)} \\
    %             \frac{1}{n + 1} & \frac{1}{n + 2} & \cdots & \frac{1}{2n} \\
    %         \end{vmatrix}_{n \times n} \\
    %         & \overset{r_{1}^{\frac{1}{n - 1}}, \ r_{2}^{\frac{1}{n - 2}}, \ \cdots, \ r_{n - 2}^{\frac{1}{2}}, \ c_{1}^{n + 1}, \ c_{2}^{n + 2}, \ \cdots, \ c_{n}^{2n}}= (\displaystyle\prod_{i = 1}^{n - 1} i)(\displaystyle\prod_{j = 1}^{n} \frac{1}{n + j})\begin{vmatrix}
    %             \frac{1}{2} & \frac{1}{3} & \cdots & \frac{1}{n + 1} \\
    %             \frac{1}{3} & \frac{1}{4} & \cdots & \frac{1}{n + 2} \\
    %             \vdots & \vdots & \ddots & \vdots \\
    %             \frac{1}{n} & \frac{1}{n + 1} & \cdots & \frac{1}{2n - 1} \\
    %             1 & 1 & \cdots & 1 \\
    %         \end{vmatrix}_{n \times n} \\
    %         & \overset{c_{n(n - 1)}^{-1}, \ c_{n(n - 2)}^{-1}, \ \cdots, \ c_{n1}^{-1}}= \frac{n!(n - 1)!}{(2n)!}\begin{vmatrix}
    %             \frac{n - 1}{2 \times (n + 1)} & \frac{n - 2}{3 \times (n + 1)} & \cdots & \frac{1}{n \times (n + 1)} & \frac{1}{n + 1} \\
    %             \frac{n - 1}{3 \times (n + 2)} & \frac{n - 2}{4 \times (n + 2)} & \cdots & \frac{1}{(n + 1) \times (n + 2)} & \frac{1}{n + 2} \\
    %             \vdots & \vdots & \ddots & \vdots & \vdots \\
    %             \frac{n - 1}{n \times (2n - 1)} & \frac{n - 2}{(n + 1) \times (2n - 1)} & \cdots & \frac{1}{(2n - 2) \times (2n - 1)} & \frac{1}{2n - 1} \\
    %             0 & 0 & \cdots & 0 & 1 \\
    %         \end{vmatrix}_{n \times n} \\
    %         & \overset{c_{1}^{\frac{1}{n - 1}}, \ c_{2}^{\frac{1}{n - 2}}, \ \cdots, \ r_{n - 2}^{\frac{1}{2}}, \ r_{1}^{n + 1}, \ r_{2}^{n + 2}, \ \cdots, \ c_{n - 1}^{2n - 1}}= \frac{n!(n - 1)!}{(2n)!}(\displaystyle\prod_{i = 1}^{n - 1} i)(\displaystyle\prod_{j = 1}^{n} \frac{1}{n + j})\begin{vmatrix}
    %             \frac{1}{2} & \frac{1}{3} & \cdots & \frac{1}{n} & 1 \\
    %             \frac{1}{3} & \frac{1}{4} & \cdots & \frac{1}{n + 1} & 1 \\
    %             \vdots & \vdots & \ddots & \vdots & \vdots & \vdots \\
    %             \frac{1}{n} & \frac{1}{n + 1} & \cdots & \frac{1}{2n - 2} & 1 \\
    %             0 & 0 & \cdots & 0 & 1 \\
    %         \end{vmatrix}_{(n - 1) \times (n - 1)} \\
    %         & = \frac{n!(n - 1)!}{(2n)!}(\displaystyle\prod_{i = 1}^{n - 1} i)(\displaystyle\prod_{j = 1}^{n} \frac{1}{n + j})\begin{vmatrix}
    %             \frac{1}{2} & \frac{1}{3} & \cdots & \frac{1}{n} \\
    %             \frac{1}{3} & \frac{1}{4} & \cdots & \frac{1}{n + 1}\\
    %             \vdots & \vdots & \ddots & \vdots \\
    %             \frac{1}{n} & \frac{1}{n + 1} & \cdots & \frac{1}{2n - 2}
    %         \end{vmatrix}_{(n - 1) \times (n - 1)} \\
    %         & = \frac{(n!)^2[(n - 1)!]^2 \times 2n}{[(2n)!]^2} \times P_{n - 1}
    %     \end{aligned}
    % \end{equation}
    And, we have \begin{equation}
        \frac{\det(P_{n + 1})}{\det(P_{n})} = \frac{[(n + 1)!]^2(n!)^2 \times (2n + 2)}{[(2n + 2)!]^2}
    \end{equation}
    \begin{answer}{$\dag$}\begin{equation}
            \frac{\det(P_{n + 1})}{\det(P_{n})} = \frac{[(n + 1)!]^2(n!)^2 \times (2n + 2)}{[(2n + 2)!]^2}
        \end{equation}
    \end{answer}
    \item We have \begin{equation}
        \begin{cases}
            \vec{u}\herm\vec{u} = \vec{v}\herm\vec{v} \\
            \vec{u} \neq \vec{v} \\
            \mat{A}\herm\mat{A} = \mat{I} \\
            \mat{A}\vec{u} = \vec{v}
        \end{cases}
    \end{equation} Let $\vec{w} = \vec{u} - \vec{v}$, and we assume \begin{equation}
        \mat{A} = \mat{I} - \frac{1}{\vec{w}\herm\vec{u}}\vec{w}\vec{w}\herm
    \end{equation} Then, we have \begin{equation}
        \begin{aligned}
            \mat{A}\vec{u} & = \mat{I}\vec{u} - \frac{1}{\vec{w}\herm\vec{u}}\vec{w}\vec{w}\herm\vec{u} = \vec{u} - \vec{w} = \vec{v} \\
            \mat{A}\herm\mat{A} & = (\mat{I} - \frac{1}{\vec{w}\herm\vec{u}}\vec{w}\vec{w}\herm)\herm(\mat{I} - \frac{1}{\vec{w}\herm\vec{u}}\vec{w}\vec{w}\herm) \\
            & = (\mat{I} - \frac{1}{\vec{w}\herm\vec{u}}\vec{w}\vec{w}\herm)(\mat{I} - \frac{1}{\vec{w}\herm\vec{u}}\vec{w}\vec{w}\herm) \\
            & = \mat{I} - \frac{1}{\vec{u}\herm\vec{w}}\vec{w}\vec{w}\herm - \frac{1}{\vec{w}\herm\vec{u}}\vec{w}\vec{w}\herm + \frac{1}{(\vec{u}\herm\vec{w})(\vec{w}\herm\vec{u})}\vec{w}\vec{w}\herm\vec{w}\vec{w}\herm \\
            & = \mat{I} - [\frac{1}{\vec{u}\herm\vec{w}} - \frac{1}{\vec{w}\herm\vec{u}} + \frac{\vec{w}\herm\vec{w}}{(\vec{u}\herm\vec{w})(\vec{w}\herm\vec{u})}]\vec{w}\vec{w}\herm \\
            & = \mat{I} - \frac{\vec{w}\herm(\vec{u} - \vec{w}) + \vec{u}\herm\vec{w}}{(\vec{u}\herm\vec{w})(\vec{w}\herm\vec{u})}\vec{w}\vec{w}\herm \\
            & = \mat{I} - \frac{(\vec{u} - \vec{v})\herm\vec{v} + \vec{u}\herm(\vec{u} - \vec{v})}{(\vec{u}\herm\vec{w})(\vec{w}\herm\vec{u})}\vec{w}\vec{w}\herm \\
            & = \mat{I}
        \end{aligned}
    \end{equation}
    \begin{answer}{$\dag$}\begin{equation}
            \mat{A}(\vec{u}, \ \vec{v}) = \mat{I} - \frac{1}{\vec{w}\herm\vec{u}}\vec{w}\vec{w}\herm
        \end{equation}
    \end{answer}
    \item We have \begin{enumerate}[label=(\alph*)]
        \item \begin{answer}{$\dag$}\begin{equation}
                6 \times 6 = 36
            \end{equation}
        \end{answer} Since $K_{6, 6}$ has the maximal edges.
        \item \begin{answer}{$\dag$}\begin{equation}
                e \le 3 \times 5 - 6 = 9
            \end{equation}
        \end{answer}
    \end{enumerate}
    \item We have \begin{equation}
        \begin{aligned}
            \Rightarrow & \ (\alpha - 2)(\alpha - 3) = \alpha^2 - 5 \times \alpha + 6 = 0 \\ 
            \Rightarrow & \ a_{n + 2} - 5 \times a_{n + 1} + 6 \times a_n = q_1 \times n + q_2
        \end{aligned}
    \end{equation} And, we have \begin{equation}
        (n + 2 - 7) - 5 \times (n + 1 - 7) + 6 \times (n - 7) = 2 \times n - 17
    \end{equation}
    \begin{enumerate}[label=(\alph*)]
        \item \begin{answer}{$\dag$}\begin{equation}
                p_1 = -5, \ p_2 = 6
            \end{equation}
        \end{answer}
        \item \begin{answer}{$\dag$}\begin{equation}
                q_1 = 2, \ q_2 = -17
            \end{equation}
        \end{answer}
    \end{enumerate}
    \item \begin{enumerate}[label=(\alph*)]
        \item \begin{answer}{$\dag$}\begin{equation}
                2^{4^2 - 4} = 4096
            \end{equation}
        \end{answer}
        \item \begin{answer}{$\dag$}\begin{equation}
                2^{\frac{4^2 - 4}{2} - 1} \times 2^4 = 512
            \end{equation}
        \end{answer}
    \end{enumerate}
    \item \begin{answer}{$\dag$} Suppose \begin{equation}
            x < 50 \land y < 50 \rightarrow x + y < 100
        \end{equation} contradiction, so \begin{equation}
            x \ge 50 \land y \ge 50
        \end{equation}
    \end{answer}
    \item \begin{answer}{$\dag$} Let $*$ be operator of $G$ and $\cdot$ be operator of $H$. \begin{equation}
            \begin{aligned}
                & \because \ f \ \text{is onto} \\
                & \therefore \ \forall \ y_1, \ y_2 \in H, \ \exists \ x_1, \ x_2, \ \text{s.t.} \ f(x_1) = y_1, \ f(x_2) = y_2 \\
                & \Rightarrow \ y_1 \cdot y_2 = f(x_1) \cdot f(x_2) = f(x_1 * x_2) = f(x_2 * x_1) = f(x_2) \cdot f(x_1) = y_2 \cdot y_1
            \end{aligned}
        \end{equation} Then, if $G$ is abelian, then $H$ is abelian.
    \end{answer}
\end{enumerate}

% End of solutions.

\end{document}

