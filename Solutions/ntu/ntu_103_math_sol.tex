\documentclass[a4paper,12pt]{article}

\usepackage{../header}
\newcommand{\school}{ntu}
\newcommand{\subject}{math}
\renewcommand{\year}{103}
\newcommand{\titlename}{\MakeUppercase{\school} \subject \ \year}

\fancypagestyle{mainmatter}{\rhead{\titlename}}
\pagestyle{mainmatter}
\CenterWallPaper{.50}{img/logo_ntu_recolor.jpg}
\newcommand{\ver}{\textsc{Version} 1.0} % Version number.

\begin{document}

\title{\LARGE{\textbf{Solutions}} \\
	\Huge{\textbf{\titlename}} \\
	\normalsize{\ver}
}
\author{}
\date{}

\maketitle

% Start of solutions.

\begin{enumerate}
	\item \begin{answer}{$\dag$} There are $3$ As, so we first \textbf{permute} other $4$ characters, and then \textbf{insert} $3$ As in the $5$ spaces. \begin{equation}
            \frac{4!}{2!} \times \binom{5}{3}    
        \end{equation}
    \end{answer}
    \item We have \begin{equation}
        \begin{aligned}
            & x_1 + x_2 + \cdots + x_n = r, \ \forall \ x_i > 0, \ 1 \le i \le n \\
            \Rightarrow & \ y_1 + y_2 + \cdots + y_n = r - n, \ \forall \ y_i \ge 0, \ 1 \le i \le n \\
        \end{aligned}
    \end{equation}
    \begin{answer}{$\dag$}\begin{equation}
            \binom{n + (r - n) - 1}{r - n}
        \end{equation}
    \end{answer}
    \item \begin{answer}{$\dag$}\begin{equation}
            (2^2)^{(2^m)} = 4^{(2^m)}
        \end{equation}
    \end{answer}
    \item We have \begin{equation}
        \begin{aligned}
            \sum_{n = 1}^{\infty}\sum_{i = 1}^{n}\frac{1}{i}x^n & = x + (1 + \frac{1}{2})x^2 + (1 + \frac{1}{2} + \frac{1}{3})x^3 + \cdots + (1 + \frac{1}{2} + \frac{1}{3} + \cdots + \frac{1}{n})x^n + \cdots \\
            & = 1 \times (x + x^2 + x^3 + \cdots + x^n + \cdots) + \frac{1}{2} \times (x^2 + x^3 + \cdots + x^n + \cdots) + \cdots \\
            & + \frac{1}{n} \times (x^n + x^{n + 1} + \cdots) + \cdots \\
            & = \frac{x}{1 - x} + \frac{1}{2} \times \frac{x^2}{1 - x} + \cdots + \frac{1}{n} \times \frac{x^n}{1 - x} + \cdots \\
            & = \sum_{n = 1}^{\infty}\frac{1}{n}\frac{1}{1 - x}x^n = \frac{1}{1 - x}\sum_{n = 1}^{\infty}\frac{1}{n}x^n
        \end{aligned}
    \end{equation} And, we have \begin{equation}
        \begin{aligned}
            & \sum_{n = 1}^{\infty}\frac{1}{n}x^n \overset{\text{derivative}}= \sum_{n = 1}^{\infty}x^{n - 1} = \frac{1}{1 - x} \\
            \Rightarrow & \ \frac{1}{1 - x} \overset{\text{integral}}= -\ln (1 - x)
        \end{aligned}
    \end{equation} Then, we have \begin{equation}
        \frac{1}{1 - x}\sum_{n = 1}^{\infty}\frac{1}{n}x^n = \frac{-\ln (1 - x)}{1 - x}
    \end{equation}
    \begin{answer}{$\dag$}\begin{equation}
            \frac{-\ln (1 - x)}{1 - x}
        \end{equation}
    \end{answer}
    \item We have \begin{equation}
        \begin{aligned}
            \Rightarrow & \ \alpha^2 = \alpha + 2 \\
            \Rightarrow & \ \alpha = 2 \lor \alpha = -1 \\
            \Rightarrow & \ a_n = c \times 2^n + d \times (-1)^n \\
        \end{aligned}
    \end{equation} And, we have \begin{equation}
        \begin{aligned}
            & \begin{cases}
                a_0 = 0 = c + d \\
                a_1 = 1 = 2 \times c - d
            \end{cases} \\
            \Rightarrow & \ \begin{cases}
                c = \frac{1}{3} \\
                d = -\frac{1}{3}
            \end{cases}
        \end{aligned}
    \end{equation}
    \begin{answer}{$\dag$}\begin{equation}
            a_n = \frac{1}{3} \times 2^n - \frac{1}{3} \times (-1)^n
        \end{equation}
    \end{answer}
    \item\begin{answer}{$\dag$}\begin{equation}
            cfjgda
        \end{equation}
    \end{answer}
    \item \begin{answer}{$\dag$} \quad \begin{enumerate}[label=(\alph*)]
            \item If $\spc{S} = \emptyset$, $\spn(\spc{S}) = \{\vec{0}\} \subseteq \spc{V}$. \\ 
            Otherwise, if $\spc{S} \neq \emptyset$, $\vec{0} \in \spn(\spc{S})$, and $\forall \ \vec{x}, \ \vec{y} \in \spn(\spc{S})$, let \begin{equation}
                    \begin{aligned}
                        & \begin{cases}
                            \spn(\spc{S}) & = \spn\{v_1, \ v_2, \cdots, \ v_n\} \\
                            \vec{x} & = a_1v_1 + a_2v_2 + \cdots + a_nv_n \\
                            \vec{y} & = b_1v_1 + b_2v_2 + \cdots + b_nv_n
                        \end{cases} \\
                        \Rightarrow & \ \forall \ \alpha, \ \beta \in \R, \ \alpha\vec{x} + \beta\vec{y} = \\
                        & \ (\alpha a_1 + \beta b_1)v_1 + (\alpha a_2 + \beta b_2)v_2 + \cdots + (\alpha a_n + \beta b_n)v_n \in \spn(\spc{S}) \\
                        \Rightarrow & \ \spn(\spc{S}) \subseteq \V
                    \end{aligned}
                \end{equation}
            \item \begin{equation}
                \begin{aligned}
                    & \spc{S} \subseteq \spc{U}, \ \forall \ \vec{x} = \{x_1, \ x_2, \cdots, \ x_n\} \in \spc{S} \\
                    \Rightarrow & \ \spn(\spc{S}) = \{\alpha x_1 + \alpha x_2 + \cdots + \alpha x_n\} \subseteq \spc{U}
                \end{aligned}
            \end{equation}
            \item Suppose \begin{equation}
                \exists \ \spc{T} \subseteq \spc{V}, \ \text{s.t.} \ \spc{T} \subseteq \spc{U}
            \end{equation} And, we have \begin{equation}
                \begin{aligned}
                    & \spc{S} \subseteq \spn(\spc{S}), \ \spn(\spc{S}) \subseteq \spc{V} \\
                    \Rightarrow & \ \spc{T} \subseteq \spn(\spc{S})
                \end{aligned}
            \end{equation} And, we have \begin{equation}
                \begin{aligned}
                    & \spc{S} \subseteq \spc{T}, \ \spn(\spc{S}) \subseteq \spc{T} \\
                    \Rightarrow & \ \spc{T} = \spn(\spc{S})
                \end{aligned}
            \end{equation}
        \end{enumerate}
    \end{answer}
    \item \quad \begin{enumerate}[label=(\alph*)]
        \item \begin{answer}{$\dag$}\begin{equation}
                \nul(\T) + \rnk(\T) = \dim(\V)
            \end{equation}
        \end{answer}
        \item We have \begin{equation}
            \begin{bmatrix}[ccc|c]
                1 & 1 & 0 & 5 \\
                1 & 0 & 1 & 3 \\
                0 & 1 & 1 & 0
            \end{bmatrix} \overset{\text{rref}}= \begin{bmatrix}[ccc|c]
                1 & 0 & 0 & 4 \\
                0 & 1 & 0 & 1 \\
                0 & 0 & 1 & -1
            \end{bmatrix}
        \end{equation}
        \begin{answer}{$\dag$}\begin{equation}
                4 \times (2, \ 0, \ 1) + 1 \times (2, \ 1, \ -1) - 1 \times (2, \ -1, \ 0) = (8, \ 2, \ 3)
            \end{equation}
        \end{answer}
        \item \begin{answer}{$\dag$}\begin{equation}
                7 \times 4 = 28
            \end{equation}
        \end{answer}
        \item \begin{answer}{$\dag$}\begin{equation}
                0, \ 1, \ 2, \ 3, \ 4, \ 5
            \end{equation} Since $\spc{U}$ and $\spc{V}$ are \textbf{distinct}, $\spc{U} = \spc{V} = \spc{W}$ does NOT exist.
        \end{answer}
    \end{enumerate}
    \item \begin{answer}{$\dag$} We have $\mat{A}^2 = \mat{I}$, so \begin{equation}
            \begin{aligned}
                \mat{A}^{-100} & = (\mat{A}^2)^{-50} = \mat{I} \\
                \mat{A}^{101} & = (\mat{A}^2)^{50} \times \mat{A} = \mat{A}
            \end{aligned}
        \end{equation}
    \end{answer}
    \item Find the minimal solution. We have \begin{equation}
        \mat{A} = \begin{bmatrix}
            2 & 1 & 1 \\
            4 & 2 & 2 \\
            5 & 1 & 0
        \end{bmatrix}, \ \vec{b} = \begin{bmatrix}
            4 \\
            8 \\
            19
        \end{bmatrix}, \ \vec{u} = \begin{bmatrix}
            a \\
            b \\
            c
        \end{bmatrix}
    \end{equation} Then, we have \begin{equation}
        \begin{aligned}
            & (\mat{A}\mat{A}\herm)\vec{u} = \vec{b} \\
            \Rightarrow & \ \begin{bmatrix}
                2 & 1 & 1 \\
                4 & 2 & 2 \\
                5 & 1 & 0
            \end{bmatrix} \begin{bmatrix}
                2 & 4 & 5 \\
                1 & 2 & 1 \\
                1 & 2 & 0
            \end{bmatrix} \begin{bmatrix}
                a \\
                b \\
                c
            \end{bmatrix} = \begin{bmatrix}
                4 \\
                8 \\
                19
            \end{bmatrix} \\
            \Rightarrow & \ \vec{u} = \begin{bmatrix}
                a \\
                b \\
                c
            \end{bmatrix} = \begin{bmatrix}
                -1 \\
                -1 \\
                2
            \end{bmatrix}
        \end{aligned}
    \end{equation} Then, we have \begin{equation}
        \mat{A}\herm\vec{u} = \begin{bmatrix}
            2 & 4 & 5 \\
            1 & 2 & 1 \\
            1 & 2 & 0
        \end{bmatrix} \begin{bmatrix}
            -1 \\
            -1 \\
            2
        \end{bmatrix} = \begin{bmatrix}
            4 \\
            -1 \\
            -3
        \end{bmatrix}
    \end{equation}
    \begin{answer}{$\dag$}\begin{equation}
            \begin{bmatrix}
                4 \\
                -1 \\
                -3
            \end{bmatrix}
        \end{equation}
    \end{answer}
    \item \begin{answer}{$\dag$}\begin{equation}
            -1, \ 1, \ 2, \ 3
        \end{equation}
    \end{answer}
\end{enumerate}

% End of solutions.

\end{document}
