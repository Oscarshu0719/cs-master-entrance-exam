\documentclass[a4paper,12pt]{article}

\usepackage{../header}
\newcommand{\school}{ntu}
\newcommand{\subject}{math}
\renewcommand{\year}{105}
\newcommand{\titlename}{\MakeUppercase{\school} \subject \ \year}

\fancypagestyle{mainmatter}{\rhead{\titlename}}
\pagestyle{mainmatter}
\CenterWallPaper{.50}{img/logo_ntu_recolor.jpg}
\newcommand{\ver}{\textsc{Version} 1.0} % Version number.

\begin{document}

\title{\LARGE{\textbf{Solutions}} \\
	\Huge{\textbf{\titlename}} \\
	\normalsize{\ver}
}
\author{}
\date{}

\maketitle

% Start of solutions.

\begin{enumerate}
	\item We have \begin{enumerate}[label=(\alph*)]
        \item True.
        \item False.
        \item True. $\tr(\mat{AB}) = \tr(\mat{BA}) \rightarrow \tr(\mat{B}^{-1}\mat{AB}) = \tr(\mat{B}^{-1}\mat{BA}) = \tr(\mat{A})$.
        \item False.
        \item True.
    \end{enumerate}
    \begin{answer}{$\dag$}\begin{equation}
            ace
        \end{equation}
    \end{answer}
    \item We have \begin{enumerate}[label=(\alph*)]
        \item True. $\det(\mat{AB}) = \det(\mat{A}) \times \det(\mat{B}) = \det(\mat{BA})$
        \item True.
        \item True. $\det(\mat{B}^{-1}\mat{AB}) = \det(\mat{B}^{-1}) \times \det(\mat{A}) \times \det(\mat{B}) = \frac{1}{\det(\mat{B})} \times \det(\mat{A}) \times \det(\mat{B}) = \det(\mat{A})$
        \item True.
        \item False.
    \end{enumerate}
    \begin{answer}{$\dag$}\begin{equation}
            abcd       
        \end{equation}
    \end{answer}
    \item We have \begin{enumerate}[label=(\alph*)]
        \item False.
        \item False, if $\mat{R} = \mat{O}$ is rectangular, but $\mat{R}^\intercal\mat{R} = \mat{O}$ is NOT positive definite.
        \item False, the orthogonal set contains $\vec{0}$, it's NOT linearly independent.
        \item True.
        \item True.
    \end{enumerate}
    \begin{answer}{$\dag$}\begin{equation}
            de       
        \end{equation}
    \end{answer}
    \item We have \begin{enumerate}[label=(\alph*)]
        \item False, since it does NOT contain $\vec{0}$.
        \item False, since it does NOT contain $\vec{0}$.
        \item False, since it does NOT contain $\vec{0}$.
        \item True.
        \item False, since it does NOT contain $\vec{0}$.
    \end{enumerate}
    \begin{answer}{$\dag$}\begin{equation}
            d        
        \end{equation}
    \end{answer}
    \item Suppose \begin{equation}
        f(x) = 1 - x^k = (1 - x)(1 + x + x^2 + \cdots + x^{k - 1})
    \end{equation} Then, we have \begin{equation}
        f(-\mat{N}) = \mat{I} + \mat{N}^k = (\mat{I} + \mat{N})(\mat{I} + (-\mat{N}) + (-\mat{N})^2 - \cdots + (-\mat{N})^{k - 1}) = \mat{I}
    \end{equation}
    \begin{answer}{$\dag$}\begin{equation}
            (\mat{I} + \mat{N})^{-1} = \mat{I} + (-\mat{N}) + (-\mat{N})^2 - \cdots + (-\mat{N})^{k - 1}
        \end{equation}
    \end{answer}
    \item \begin{answer}{$\dag$}\begin{equation}
            \frac{1}{4} \times x^2 + \frac{1}{9} \times y^2 = 1    
        \end{equation}
    \end{answer}
    \item We have characteristic polynomial \begin{equation}
        p_{\mat{I} + \mat{A}}(x) = (x - 2)(x - 7) 
    \end{equation} Then, we have eigenspaces \begin{equation}
        \begin{cases}
            \spc{V}(2) & = \left\{\begin{bmatrix}
                1 \\
                -1
            \end{bmatrix}\right\} \\
            \spc{V}(7) & = \left\{\begin{bmatrix}
                1 \\
                -2
            \end{bmatrix}\right\} \\
        \end{cases}
    \end{equation}
    \begin{answer}{$\dag$}\begin{equation}
            \begin{bmatrix}
                1 \\
                -1
            \end{bmatrix}, \ \begin{bmatrix}
                1 \\
                -2
            \end{bmatrix}
        \end{equation}
    \end{answer} Since the eigenvectors of $(\mat{I} + \mat{A})$ are the same as $(\mat{I} + \mat{A})^{100}$.
    \item Obviously, $1$ is an eigenvalue, which $\gm(1) = n - 1$. Then, we have \begin{equation}
        \tr(\mat{A}) = \sum_{i = 1}^{n}(1 + x_i) = n + \sum_{i = 1}^{n}x_i
    \end{equation} Then, we have the n-th eigenvalue \begin{equation}
        n + (\sum_{i = 1}^{n}x_i) - (n - 1) \times 1 = 1 + \sum_{i = 1}^{n}x_i
    \end{equation}
    \begin{answer}{$\dag$}\begin{equation}
            \det(\mat{A}) = 1^{n - 1} \times (1 + \sum_{i = 1}^{n}x_i) = 1 + \sum_{i = 1}^{n}x_i
        \end{equation}        
    \end{answer}
    \item We have new problem \begin{equation}
        \begin{aligned}
            & x_1 + x_2 + x_3 + x_4 < 8, \ \forall \ x_i \ge 0, \ 1 \le i \le 4 \\
            & (x_5 = 8 - (x_1 + x_2 + x_3 + x_4), \ x_5 > 0) \\
            \Rightarrow & \ x_1 + x_2 + x_3 + x_4 + y_5 = 8 - 1, \ \forall \ x_i \ge 0, \ 1 \le i \le 4, \ y_5 \ge 0 
        \end{aligned}
    \end{equation}
    \begin{answer}{$\dag$}\begin{equation}
            \binom{5 + (8 - 1) - 1}{8 - 1} = 330      
        \end{equation}
    \end{answer}
    \item \begin{answer}{$\dag$}\begin{equation}
            (1, \ 2, \ 6) \circ (3, \ 5) \circ (4, \ 8) \circ (7)            
        \end{equation}
    \end{answer}
    \item We have \begin{equation}
        \begin{aligned}
            \Rightarrow & \ \alpha = 2 \\
            \Rightarrow & \ \begin{cases}
                a_n^{(h)} = c \times 2^n \\
                a_n^{(p)} = d \times n + e
            \end{cases}
        \end{aligned}
    \end{equation} Then, we have \begin{equation}
        \begin{aligned}
            & d \times n + e = 2 \times (d \times (n - 1) + e) + n \\
            \Rightarrow & \ d = -1, \ e = -2 \\
            \Rightarrow & \ a_n = c \times 2^n - n - 2
        \end{aligned}
    \end{equation} Then, we have \begin{equation}
        \begin{aligned}
            & a_0 = 4 = c - 0 - 2 \\
            \Rightarrow & \ c = 6
        \end{aligned}
    \end{equation}
    \begin{answer}{$\dag$}\begin{equation}
            a_n = 6 \times 2^n - n - 2            
        \end{equation}
    \end{answer}
    \item \begin{equation}
		\Rightarrow \sum_{i = 1}^n a_n x^n = 2 \times \sum_{i = 1}^n a_{n - 1}x^n + \sum_{i = 1}^n nx^n
	\end{equation} We have \begin{equation}
		\sum_{i = 1}^n nx^n = x\sum_{i = 1}^n nx^{n - 1}
	\end{equation} Then, we have \begin{equation}
		\begin{aligned}
			\Rightarrow & \ \sum_{i = 1}^n nx^{n - 1} \overset{\text{integral}}= \sum_{i = 1}^n x^n = \frac{x}{1 - x} \\
			\Rightarrow & \ \frac{x}{1 - x} \overset{\text{derivative}}= \frac{1}{(1 - x)^2} \\ 
			\Rightarrow & \ \sum_{i = 1}^n nx^n = \frac{x}{(1 - x)^2}
		\end{aligned}
	\end{equation} We have the new generating function \begin{equation}
		\begin{aligned}
			& A(x) - a_0 = 2x \times A(x) + \frac{x}{(1 - x)^2} \\
			\Rightarrow & \ A(x) = \frac{4 \times x^2 - 7 \times x + 4}{(1 - 2x)(1 - x)^2} \\
			\Rightarrow & \ A(x) = 6 \times \frac{1}{1 - 2x} - 2 \times \frac{1}{1 - x} - \frac{x}{(1 - x)^2} \\
			\Rightarrow & \ A(x) = 6 \times \frac{1}{1 - 2x} - \frac{1}{1 - x} - (\frac{1}{1 - x} + \frac{x}{(1 - x)^2}) \\
			\Rightarrow & \ A(x) = 6 \times \frac{1}{1 - 2x} - \frac{1}{1 - x} - \frac{1}{(1 - x)^2} \\
		\end{aligned}
	\end{equation}
	\begin{answer}{$\dag$} \begin{equation}
            6 \times \frac{1}{1 - 2x} - \frac{1}{1 - x} - \frac{1}{(1 - x)^2}
		\end{equation}
	\end{answer}
    \item We have \begin{equation}
        (1 + x + x^2 + \cdots)(1 + (x^2) + (x^2)^2 + \cdots)(1 + (x^3) + (x^3)^2 + \cdots)\cdots
    \end{equation} Since each number can be repeated.
    \begin{answer}{$\dag$}\begin{equation}
            \displaystyle\prod_{i = 1}^{n}\frac{1}{1 - x^i}            
        \end{equation} 
    \end{answer}
    \item \begin{answer}{$\dag$}\begin{equation}
            2^{2^{m - 1}}            
        \end{equation}
    \end{answer} Since the base means $0$ and $1$ two values of the codomain, and the index means that if we know the $01$-sequence then we know the opposite.
    \item We have \begin{equation}
        n \le 2 \times i + 1
    \end{equation}
    \begin{answer}{$\dag$}\begin{equation}
            i \ge \floor{\frac{n - 1}{2}}
        \end{equation}
    \end{answer}
\end{enumerate}

% End of solutions.

\end{document}
